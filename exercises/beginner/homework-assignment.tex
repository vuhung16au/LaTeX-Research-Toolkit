% Homework Assignment Template
% Demonstrates key LaTeX techniques for homework:
% - Math mode with \[ ... \] for display equations
% - enumerate environment for numbered problem lists
% - align environment for multi-step derivations
% - Tables with [h] positioning
% - Cross-referencing with \label and \ref
% - Including figures
\documentclass[11pt]{article}
\usepackage[a4paper,margin=1in]{geometry}
\usepackage{amsmath,amssymb}
\usepackage{graphicx}
\usepackage{enumitem}
\usepackage{hyperref}

\title{Homework \#1}
\author{Your Name \\ Course \& Section \\ Date}
\date{}

% Customize list labels: "Problem 1:", "Problem 2:", etc.
\setlist[enumerate,1]{label=Problem~\arabic*:}
% Sub-parts use letters: a), b), c)
\setlist[enumerate,2]{label=\alph*)}

\begin{document}
\maketitle

\section*{Instructions}
Show all work. When aligning multi-step derivations, use the \texttt{align} environment.

\section*{Problems}
\begin{enumerate}
  % Problem 1: Riemann Integral - demonstrates \int, subscripts, superscripts, \lim, \sum
  \item Write out the definition of the Riemann integral using the limit of Riemann sums:
  \[
  \int_{a}^{b} f(x) \, dx = \lim_{||\Delta x|| \to 0} \sum_{k=1}^{n} f(x_k^*) \Delta x_k
  \]
  
  % Problem 2: Derivative derivation - demonstrates align environment with & for alignment
  \item Derive the derivative of $f(x)=x^2$ using the limit definition. Use the \texttt{align} environment to align all equals signs:
\begin{align}
  f'(x) &= \lim_{h \to 0} \frac{(x+h)^2 - x^2}{h} \label{eq:step1} \\
        &= \lim_{h \to 0} \frac{(x^2 + 2xh + h^2) - x^2}{h} \label{eq:step2} \\
        &= \lim_{h \to 0} \frac{2xh + h^2}{h} \label{eq:step3} \\
        &= \lim_{h \to 0} (2x + h) \label{eq:step4} \\
        &= 2x \label{eq:final}
\end{align}
  Notice how the ampersand (\&) before each equals sign aligns them vertically. We can reference equation \ref{eq:final} for the final result.
  
  % Problem 3: Table example - demonstrates table environment with [h] positioning
  \item Create a table showing the values of $f(x) = x^2$ for several values of $x$. The table should appear exactly here (not floating to top/bottom of page):
  \begin{table}[h]
    \centering
    \begin{tabular}{|c|c|}
      \hline
      $x$ & $f(x) = x^2$ \\
      \hline
      0 & 0 \\
      1 & 1 \\
      2 & 4 \\
      3 & 9 \\
      \hline
    \end{tabular}
    \caption{Values of $f(x) = x^2$}
    \label{tab:squares}
  \end{table}
  We can reference Table~\ref{tab:squares} using the \verb|\ref| command.
  
  % Problem 4: Nested enumerate - demonstrates sub-parts with automatic letter labels
  \item This problem has multiple sub-parts:
  \begin{enumerate}
    \item Evaluate $\int_{0}^{1} x^2\,dx$.
    \item Explain why this integral represents the area under the parabola.
    \item Sketch the region and label it as Figure~\ref{fig:parabola}.
  \end{enumerate}
  
  % Problem 5: Figure inclusion - demonstrates figure environment with [h] positioning
  \item Insert a figure showing the parabola $y = x^2$:
  \begin{figure}[h]
    \centering
    \fbox{\rule{0pt}{1.2in}\rule{1.8in}{0pt}}
    \caption{Graph of the parabola $y = x^2$ (placeholder figure).}
    \label{fig:parabola}
  \end{figure}
  The \verb|[h]| parameter tells LaTeX to try to place the figure \textbf{h}ere, preventing it from floating to the top or bottom of the page.
  
\end{enumerate}

\end{document}
