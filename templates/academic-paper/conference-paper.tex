\documentclass[10pt,twocolumn]{article}
\usepackage[utf8]{inputenc}
\usepackage{amsmath,amsfonts,amssymb}
\usepackage{graphicx}
\usepackage{hyperref}
\usepackage{geometry}
\usepackage{natbib}
\usepackage{balance}
\usepackage{url}
\usepackage{setspace}

% Page setup for conference paper
\geometry{letterpaper, margin=0.75in}
\onehalfspacing

% Conference-specific formatting
\setlength{\columnsep}{0.25in}

% Title and author information
\title{Your Conference Paper Title: A Concise Description}
\author{
    Your Name$^1$ \and 
    Co-author Name$^2$ \and 
    Another Author$^1$ \\
    $^1$Your Institution, City, Country \\
    $^2$Co-author Institution, City, Country \\
    \texttt{\{your.email, coauthor.email\}@institution.edu}
}
\date{\today}

\begin{document}

\maketitle

\begin{abstract}
This abstract should be 150-200 words and clearly state the problem, approach, results, and contributions. Conference abstracts are read by reviewers and attendees, so make it compelling and clear.
\end{abstract}

\section{Introduction}
Conference papers have limited space, so your introduction should be:
\begin{itemize}
    \item Direct and to the point
    \item Clearly state the problem
    \item Highlight your contribution
    \item Motivate the work
\end{itemize}

\section{Related Work}
Keep this section focused and relevant. Conference papers typically have shorter literature reviews than journal papers.

\section{Methodology}
Describe your approach clearly. Include:
\begin{itemize}
    \item Problem formulation
    \item Your solution approach
    \item Key algorithms or techniques
    \item Implementation details (if relevant)
\end{itemize}

\section{Experimental Setup}
Describe your experimental setup:
\begin{itemize}
    \item Datasets used
    \item Experimental parameters
    \item Evaluation metrics
    \item Baseline comparisons
\end{itemize}

\section{Results and Analysis}
Present your results with clear figures and tables:

\begin{table}[h]
\centering
\begin{tabular}{|l|c|c|}
\hline
Method & Accuracy & Time (s) \\
\hline
Baseline & 85.2\% & 10.5 \\
Our Method & 92.1\% & 12.3 \\
\hline
\end{tabular}
\caption{Performance comparison}
\label{tab:results}
\end{table}

\section{Discussion}
Discuss the significance of your results and their implications.

\section{Conclusion and Future Work}
Summarize your contributions and outline future research directions.

\section*{Acknowledgments}
Thank your funding sources, collaborators, and reviewers.

\bibliographystyle{abbrv}
\bibliography{references}

\balance
\end{document}
