\documentclass[10pt,twocolumn]{article}
\usepackage[utf8]{inputenc}
\usepackage{amsmath,amsfonts,amssymb}
\usepackage{graphicx}
\usepackage{hyperref}
\usepackage{geometry}
\usepackage{natbib}
\usepackage{balance}
\usepackage{setspace}

% Page setup
\geometry{letterpaper, margin=0.75in}
\onehalfspacing

% Title and author information
\title{Your Conference Paper Title}
\author{Your Name$^1$, Co-author Name$^2$ \\
        $^1$Your Institution \\
        $^2$Co-author Institution \\
        \texttt{your.email@institution.edu}}
\date{\today}

\begin{document}

\maketitle

\begin{abstract}
This is your abstract for a conference paper. It should be concise and highlight the key contributions. Conference abstracts are typically shorter than journal abstracts.
\end{abstract}

\section{Introduction}
Your introduction for a conference paper should be more focused and direct than a journal paper. Get to the point quickly and clearly state your contribution.

\section{Related Work}
Briefly review the most relevant work. In conference papers, this section is often shorter than in journal papers.

\section{Methodology}
Describe your approach clearly and concisely. Conference papers have limited space, so focus on the most important aspects.

\section{Results}
Present your results with clear figures and tables. Make sure they are readable in the two-column format.

\begin{figure}[h]
\centering
% Use placeholder if figure.png is not present
\IfFileExists{figure.png}{%
  \includegraphics[width=\columnwidth]{figure.png}%
}{%
  \fbox{\parbox{0.9\columnwidth}{Placeholder figure: add `figure.png` next to this .tex}}%
}
\caption{Your figure caption here}
\label{fig:example}
\end{figure}

\section{Discussion}
Discuss the implications of your results and their significance.

\section{Conclusion}
Summarize your contributions and suggest future work.

\section*{Acknowledgments}
Thank your funding sources and collaborators.

\bibliographystyle{abbrv}
\bibliography{references}

\balance
\end{document}
