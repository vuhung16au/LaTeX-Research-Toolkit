\documentclass[12pt]{report}
\usepackage[utf8]{inputenc}
\usepackage{amsmath,amsfonts,amssymb}
\usepackage{graphicx}
\usepackage{hyperref}
\usepackage{geometry}
\usepackage[backend=biber,style=numeric-comp]{biblatex}
\addbibresource{../../references/references.bib}
\usepackage{tocloft}
\usepackage{setspace}
\usepackage{fancyhdr}

% Page setup
\geometry{letterpaper, margin=1in}
\onehalfspacing

% Headers and footers
\pagestyle{fancy}
\fancyhf{}
\rhead{Your Name}
\lhead{PhD Thesis}
\cfoot{\thepage}

% Title page information
\title{Your PhD Thesis Title: A Comprehensive Investigation}
\author{Your Name}
\date{\today}

\begin{document}

% Title page
\begin{titlepage}
\centering
\vspace*{2cm}

{\LARGE \textbf{Your PhD Thesis Title: A Comprehensive Investigation}}

\vspace{1.5cm}

{\large by}

\vspace{0.5cm}

{\large Your Name}

\vspace{1.5cm}

A dissertation submitted in partial fulfillment of the requirements for the degree of Doctor of Philosophy in [Your Field]

\vspace{1cm}

Your Institution Name

Your Department

\vspace{1cm}

\date{\today}
\end{titlepage}

% Abstract
\chapter*{Abstract}
Your PhD abstract should be comprehensive (500-800 words) and include:
\begin{itemize}
    \item Problem statement and motivation
    \item Research objectives and questions
    \item Methodology and approach
    \item Key findings and contributions
    \item Implications and future work
\end{itemize}

\chapter*{Acknowledgments}
Thank your advisor, committee members, collaborators, funding sources, family, and friends.

% Table of contents
\tableofcontents
\listoffigures
\listoftables

% Main content
\chapter{Introduction}
Your PhD introduction should be comprehensive and include:
\begin{itemize}
    \item Background and motivation
    \item Problem statement and research questions
    \item Objectives and scope
    \item Significance and contributions
    \item Thesis organization
\end{itemize}

\chapter{Literature Review}
This chapter should provide a comprehensive review:
\begin{itemize}
    \item Theoretical foundation
    \item Historical development of the field
    \item Current state of knowledge
    \item Research gaps and opportunities
    \item Positioning of your work
\end{itemize}

\chapter{Theoretical Framework}
Present your theoretical framework:
\begin{itemize}
    \item Theoretical foundations
    \item Conceptual models
    \item Hypotheses or propositions
    \item Theoretical contributions
\end{itemize}

\chapter{Research Methodology}
Describe your research methodology in detail:
\begin{itemize}
    \item Research design and approach
    \item Data collection methods
    \item Analysis techniques
    \item Validation and reliability
    \item Ethical considerations
\end{itemize}

\chapter{Results}
Present your findings systematically:
\begin{itemize}
    \item Descriptive results
    \item Analytical results
    \item Statistical analysis
    \item Figures and tables
\end{itemize}

\chapter{Discussion}
Discuss the implications of your results:
\begin{itemize}
    \item Interpretation of findings
    \item Comparison with literature
    \item Theoretical implications
    \item Practical implications
    \item Limitations
\end{itemize}

\chapter{Conclusion}
Summarize your contributions and suggest future research directions.

% Bibliography
\printbibliography

% Appendices
\appendix
\chapter{Appendix A: Research Instruments}
Include surveys, interview guides, or other research instruments.

\chapter{Appendix B: Additional Results}
Include supplementary results, tables, or figures.

\chapter{Appendix C: Code and Data}
Include relevant code, algorithms, or data samples.

\end{document}
