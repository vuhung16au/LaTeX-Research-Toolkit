\documentclass[tikz,border=0.1mm]{standalone}
\usetikzlibrary{arrows,calc,patterns,angles,shadows,quotes}
\usepackage{amsmath, amssymb}
\usepackage[utf8]{vietnam}
\usepackage{xcolor}

% Define theme colors
\definecolor{ThemePurple}{RGB}{60, 16, 83}      % #3C1053
\definecolor{ThemeRed}{RGB}{242, 18, 12}       % #F2120C
\definecolor{ThemePurpleRed}{RGB}{181, 24, 37} % #B51825
\definecolor{WarmStone}{RGB}{145, 139, 131}    % #918B83
\definecolor{DeepCharcoal}{RGB}{48, 44, 42}    % #302C2A
\definecolor{SoftIvory}{RGB}{242, 239, 235}    % #F2EFEB

\def\tacgia{Vu}
\begin{document}
\mathversion{bold}
% Define trigonometric functions for plotting
% These macros convert angles to radians for TikZ calculations
\def\fsin(#1){sin(#1 r)}
\def\fcos(#1){cos(#1 r)}
\pgfmathsetmacro{\xmin}{-8.0}
\pgfmathsetmacro{\xmax}{8.0}
\pgfmathsetmacro{\ymin}{-4.5}
\pgfmathsetmacro{\ymax}{4.5}
\pgfmathsetmacro{\trucxmin}{-1}
\pgfmathsetmacro{\trucxmax}{1}
\pgfmathsetmacro{\trucymin}{-1}
\pgfmathsetmacro{\trucymax}{1}
\pgfmathsetmacro{\g}{pi/2}
\pgfmathsetmacro{\hamxmin}{-0.5}
\pgfmathsetmacro{\hamxmax}{6.5}

% Animation loop: Demonstrate how sin(alpha) and cos(alpha) change as the angle rotates
% This shows the relationship: cos(alpha) = sin(alpha + pi/2)
% Creates 361 frames (0° to 360°) showing a complete rotation around the unit circle
\foreach \m in {0,...,360}{
% Base angle: alpha = 60 degrees (used as reference angle)
\def\rgoc{60}
% Total angle after rotation: alpha + rotation angle
\pgfmathsetmacro{\rgocm}{int(\rgoc+\m)}
\begin{tikzpicture}[join=round,cap=round, >=stealth,font=\footnotesize]
\clip(\xmin,\ymin) rectangle (\xmax,\ymax);
\fill[fill=SoftIvory] (\xmin,\ymin) rectangle (\xmax,\ymax);

% Title - moved down to ensure visibility
\node[DeepCharcoal,above,font=\large\bfseries,align=center] at (0,3.8) 
    {Unit Circle: Understanding $\sin(\alpha)$ and $\cos(\alpha)$};
    
% Progress indicator - moved up to ensure visibility
\node[WarmStone,below,font=\small] at (0,-3.8) 
    {Rotation: $\m^\circ$ of $360^\circ$};

% Unit circle visualization - scaled 3x for better visibility
% This scope contains the main trigonometric circle with axes, grid, and geometric elements
\begin{scope}[scale=3,font=\footnotesize,shift={(0,0)}]
    % Light grid for reference
    \draw[WarmStone!15,very thin] (-1,-1) grid[step=0.5] (1,1);
    
    % Axes
    \draw[->,color=black,thick] (\trucxmin,0.) -- (\trucxmax,0.)
        node[below,xshift=-4pt] {$x$};
    \draw[->,color=black,thick] (0.,\trucymin) -- (0.,\trucymax)
        node[below right] {$y$};
    
    % Unit circle with radius label
    \draw[ThemeRed,thick] (0,0) circle (1cm);
    \node[ThemeRed,above right,font=\tiny] at (0.7,0.7) {$r=1$};
    
    % Circle markers
    \draw
        [fill=ThemeRed](0,0)circle(0.5pt)node[shift={(-45:7pt)}] {$O$}
        (1,0)node[shift={(-45:7pt)}] {$1$}
        (-1,0)node[shift={(-135:7pt)}] {$-1$}
        (0,1)node[shift={(45:7pt)}] {$1$}
        (0,-1)node[shift={(-45:7pt)}] {$-1$};
    
    % Radius vector (defined first for triangle)
    \draw[ultra thick,ThemeRed,->,rotate=\rgoc] (0,0) -- (1.0,0) coordinate(A);
    
    % Initial angle components (base angle alpha = 60°)
    % These represent sin(alpha) and cos(alpha) before rotation
    % Purple-red line: sin component (vertical projection)
    \draw[ultra thick,ThemePurpleRed](0,0)--(0,0|-\rgoc:1cm);
    % Purple line: cos component (horizontal projection)
    \draw[ultra thick,ThemePurple] (0,0)--(\rgoc:1.0cm|-0,0);
    % Dotted lines: projections from point on circle to axes
    \draw[densely dotted,ThemePurpleRed](\rgoc:1.0cm)--(\rgoc:1.0cm-|0,0)coordinate(hsin);
    \draw[densely dotted,ThemePurple](\rgoc:1.0cm)--(\rgoc:1.0cm|-0,0)coordinate(hcos);
    
    % Right triangle visualization: Shows the relationship between angle, sin, and cos
    % The triangle has sides: opposite (sin), adjacent (cos), and hypotenuse (radius = 1)
    \draw[fill=SoftIvory!80,draw=ThemePurple,very thin,opacity=0.3] (0,0) -- (A) -- (hsin) -- cycle;
    
    % Angle arc with label
    \draw[thick,DeepCharcoal,->] (0.2cm,0) arc(0:\rgoc:0.2cm);
    \path({0.5*\rgoc}:3pt) node[rotate=\rgoc,text=DeepCharcoal,font=\small]{$\alpha$};
    
    % Rotating scope: Rotate the entire trigonometric diagram by angle \m
    % This creates the animation effect showing how sin/cos values change with rotation
    % \m ranges from 0° to 360° to complete a full circle
    \begin{scope}[rotate=\m]
    % Rotated angle components: sin(alpha + m) and cos(alpha + m)
    \draw[ultra thick,ThemePurpleRed](0,0)--(0,0|-\rgoc:1cm);
    \draw[ultra thick,ThemePurple] (0,0)--(\rgoc:1.0cm|-0,0);
    % Dotted lines: projections for the rotated angle
    \draw[densely dotted,ThemePurpleRed](\rgoc:1.0cm)--(\rgoc:1.0cm-|0,0)coordinate(hcosm);
    \draw[densely dotted,ThemePurple](\rgoc:1.0cm)--(\rgoc:1.0cm|-0,0)coordinate(hsinm) ;
    % Moving radius vector (yellow): rotates around the circle
    \draw[ultra thick,yellow,->,rotate=\rgoc] (0,0) -- (1.0,0);
    \fill[draw=black,fill=ThemeRed]
        (\rgoc:1.0cm) circle(0.5pt) (hcosm) circle(0.5pt) (hsinm) circle(0.5pt);
    \end{scope}
    
    % Rotation arc
    \draw[thick,WarmStone,->] (\rgoc:0.2cm) arc(\rgoc:\rgocm:0.2cm);
    \path(0.75*\rgocm:8pt) node[text=WarmStone,font=\small]{\m$^\circ$};
    
    % Projections for rotated angle: Show sin(alpha+m) and cos(alpha+m) on axes
    % These lines connect the rotated point to the x and y axes
    \draw[densely dotted,ThemePurple](hcosm)--(hcosm-|0,0);
    \draw[densely dotted,ThemePurpleRed](hsinm)--(hsinm|-0,0);
    % Thick lines: The actual sin and cos values as lengths on the axes
    \draw[ultra thick,ThemePurple](0,0)--(hcosm-|0,0);
    \draw[ultra thick,ThemePurpleRed] (0,0)--(hsinm|-0,0);
    \fill[draw=black,fill=ThemeRed](hcosm-|0,0) circle(0.5pt) (hsinm|-0,0) circle(0.5pt);
    
    % Label triangle sides - adjusted positions to avoid overlap
    \node[ThemePurpleRed,left,font=\tiny] at ($(hsin)!0.5!(0,0|-\rgoc:1cm) + (-0.15,0)$) {opp};
    \node[ThemePurple,below,font=\tiny] at ($(hcos)!0.5!(\rgoc:1.0cm|-0,0) + (0,-0.15)$) {adj};
    \node[ThemeRed,above left,font=\tiny] at ($(A)!0.5!(0,0) + (0.1,0.1)$) {hyp=1};
    
    \end{scope}

% Display calculated values for the rotated angle
% Shows angle in degrees and radians, plus computed sin and cos values
\node[DeepCharcoal,right,font=\normalsize] at (-7,-2.2)
            {$\alpha+ \m^{\circ} = \rgocm^{\circ}  = \pgfmathparse{rad(\rgocm)} \pgfmathresult$ rad};
\node[ThemePurple,right,font=\normalsize] at (-7,-2.7)
            {$\sin \left( \alpha+ \m^{\circ}\right)     = \pgfmathsin{\rgocm}\pgfmathresult$};
\node[ThemePurpleRed,right,font=\normalsize] at (-7,-3.2)
            {$\cos \left( \alpha+ \m^{\circ}\right)   = \pgfmathcos{\rgocm}\pgfmathresult$};

% Point labels
\draw[fill=ThemeRed] (A) circle(1.5pt) node[shift={(\rgoc:7pt)},rotate=0,font=\small]{$A$}
(hsin)circle(1.5pt) node[shift={(180:6pt)},font=\small]{$H$}
(hcos)circle(1.5pt) node[shift={(-45:6pt)},font=\small]{$K$};

% Legend - moved to middle-right edge of screen (no box, just items)
\draw[ThemePurpleRed,ultra thick] (6.5,1.0) -- (6.8,1.0);
\node[DeepCharcoal,right,font=\tiny] at (6.8,1.0) {$\sin$};
\draw[ThemePurple,ultra thick] (6.5,0.6) -- (6.8,0.6);
\node[DeepCharcoal,right,font=\tiny] at (6.8,0.6) {$\cos$};
\draw[yellow,ultra thick] (6.5,0.2) -- (6.8,0.2);
\node[DeepCharcoal,right,font=\tiny] at (6.8,0.2) {radius};

% Author - moved to ensure visibility
\node[DeepCharcoal,above right,font=\tiny] at (7.5,3.5){\tacgia};
\end{tikzpicture}
}
\end{document}
