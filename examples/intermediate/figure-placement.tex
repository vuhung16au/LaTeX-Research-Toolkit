\documentclass{article}
\usepackage[utf8]{inputenc}
\usepackage{graphicx}
\usepackage{float}
\usepackage{caption}
\usepackage{subcaption}

\title{Figure Placement Examples in LaTeX}
\author{Your Name}
\date{\today}

\begin{document}

\maketitle

\section{Basic Figure Placement}
Here's a basic figure with automatic placement:

\begin{figure}[h]
\centering
\includegraphics[width=0.6\textwidth]{sample-figure.png}
\caption{A sample figure with automatic placement}
\label{fig:sample}
\end{figure}

\section{Figure Placement Options}
LaTeX provides several placement options for figures:

\begin{itemize}
    \item \texttt{[h]} - here (if possible)
    \item \texttt{[t]} - top of page
    \item \texttt{[b]} - bottom of page
    \item \texttt{[p]} - page of floats
    \item \texttt{[H]} - HERE (requires float package)
\end{itemize}

\section{Figure with HERE Placement}
This figure uses the [H] option to force placement exactly here:

\begin{figure}[H]
\centering
\includegraphics[width=0.5\textwidth]{sample-figure.png}
\caption{A figure forced to appear here}
\label{fig:here}
\end{figure}

\section{Multiple Figures}
You can include multiple figures in a single figure environment:

\begin{figure}[h]
\centering
\begin{subfigure}{0.45\textwidth}
    \includegraphics[width=\textwidth]{figure1.png}
    \caption{First subfigure}
    \label{fig:sub1}
\end{subfigure}
\hfill
\begin{subfigure}{0.45\textwidth}
    \includegraphics[width=\textwidth]{figure2.png}
    \caption{Second subfigure}
    \label{fig:sub2}
\end{subfigure}
\caption{Two related figures}
\label{fig:multiple}
\end{figure}

\section{Figure Sizing Options}
Different ways to size figures:

\begin{figure}[h]
\centering
\includegraphics[width=0.8\textwidth]{sample-figure.png}
\caption{Figure sized to 80\% of text width}
\label{fig:width}
\end{figure}

\begin{figure}[h]
\centering
\includegraphics[height=3cm]{sample-figure.png}
\caption{Figure with fixed height}
\label{fig:height}
\end{figure}

\begin{figure}[h]
\centering
\includegraphics[scale=0.5]{sample-figure.png}
\caption{Figure scaled to 50\% of original size}
\label{fig:scale}
\end{figure}

\section{Figure References}
As shown in Figure~\ref{fig:sample}, the basic placement works well for most cases.

Figure~\ref{fig:here} demonstrates the use of forced placement.

The multiple figures in Figure~\ref{fig:multiple} show how to combine related images.

\section{Figure Placement Best Practices}
\begin{itemize}
    \item Use \texttt{[h]} for most cases
    \item Use \texttt{[H]} sparingly and only when necessary
    \item Avoid \texttt{[p]} unless you have many figures
    \item Consider using \texttt{[t]} or \texttt{[b]} for better page layout
    \item Always include captions and labels
    \item Reference figures in your text
\end{itemize}

\end{document}
