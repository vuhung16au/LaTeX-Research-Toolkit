\documentclass{article}
\usepackage[utf8]{inputenc}
\usepackage{amsmath,amsfonts,amssymb}
\usepackage{graphicx}
\usepackage{xcolor}

\title{Custom Commands in LaTeX}
\author{Your Name}
\date{\today}

% Custom commands
\newcommand{\mycommand}{This is a custom command}
\newcommand{\greeting}[1]{Hello, #1!}
\newcommand{\highlight}[1]{\textcolor{red}{\textbf{#1}}}
\newcommand{\important}[1]{\textcolor{blue}{\textbf{#1}}}

% Commands with optional parameters
\newcommand{\person}[2][Dr.]{#1 #2}
\newcommand{\email}[1]{\texttt{#1}}

% Math commands
\newcommand{\R}{\mathbb{R}}
\newcommand{\N}{\mathbb{N}}
\newcommand{\Z}{\mathbb{Z}}
\newcommand{\C}{\mathbb{C}}

% Custom environments
\newenvironment{mybox}
{\begin{center}\begin{tabular}{|p{0.8\textwidth}|}\hline}
{\\\hline\end{tabular}\end{center}}

% Custom counters
\newcounter{mycounter}
\newcommand{\myitem}{\stepcounter{mycounter}\arabic{mycounter}. }

\begin{document}

\maketitle

\section{Basic Custom Commands}
Here are examples of basic custom commands:

\mycommand

\greeting{World}

\greeting{LaTeX}

\section{Commands with Parameters}
Commands can take parameters:

\person{Smith} % Uses default "Dr."
\person[Prof.]{Jones} % Uses "Prof."

\email{your.email@institution.edu}

\section{Formatting Commands}
Use custom commands for consistent formatting:

This is \highlight{important text} that stands out.

This is \important{another important point}.

\section{Mathematical Commands}
Custom math commands for common sets:

The real numbers: $\R$

The natural numbers: $\N$

The integers: $\Z$

The complex numbers: $\C$

\section{Custom Environments}
Here's a custom box environment:

\begin{mybox}
This is content inside a custom box environment.
It can contain multiple lines of text.
\end{mybox}

\section{Custom Counters}
Here's a custom numbered list:

\myitem First item
\myitem Second item
\myitem Third item

\section{Advanced Custom Commands}
More complex commands:

\newcommand{\citeauthor}[1]{\cite{#1}}
\newcommand{\fullcite}[1]{\citeauthor{#1} (\citeyear{#1})}

\newcommand{\todo}[1]{\textcolor{red}{\textbf{[TODO: #1]}}}

\todo{Add more examples here}

\section{Best Practices for Custom Commands}
\begin{itemize}
    \item Use descriptive names
    \item Document your commands
    \item Keep commands simple
    \item Use parameters when appropriate
    \item Test commands thoroughly
    \item Consider using packages instead of complex custom commands
\end{itemize}

\section{Common Custom Commands}
Here are some commonly useful custom commands:

\begin{verbatim}
% For consistent formatting
\newcommand{\code}[1]{\texttt{#1}}
\newcommand{\file}[1]{\texttt{#1}}
\newcommand{\url}[1]{\texttt{#1}}

% For mathematical notation
\newcommand{\abs}[1]{\left|#1\right|}
\newcommand{\norm}[1]{\left\|#1\right\|}
\newcommand{\inner}[2]{\langle #1, #2 \rangle}

% For consistent spacing
\newcommand{\bigskip}{\vspace{1cm}}
\newcommand{\medskip}{\vspace{0.5cm}}
\newcommand{\smallskip}{\vspace{0.25cm}}
\end{verbatim}

\end{document}
