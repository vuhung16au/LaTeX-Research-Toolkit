% Deep Learning 101 - Main Book File
% Author: Vu Hung Nguyen
% License: Creative Commons License Version 4.0 (CC BY 4.0)

\documentclass[10pt,a5paper,twoside,openright]{book}

% --- Page Layout ---
\usepackage[a5paper, top=2cm, bottom=2cm, inner=1.8cm, outer=1.5cm]{geometry}
\usepackage{setspace}
\onehalfspacing

% --- Essential Packages ---
\usepackage[utf8]{inputenc}
\usepackage[T1]{fontenc}
\usepackage{lmodern}
\usepackage{microtype}

% --- Math Packages ---
\usepackage{amsmath,amssymb,amsfonts,amsthm}
\usepackage{mathtools}

% --- Graphics and Colors ---
\usepackage{graphicx}
\usepackage{xcolor}
\usepackage{tikz}

% --- Tables and Lists ---
\usepackage{booktabs}
\usepackage{array}
\usepackage{enumitem}

% --- References and Links ---
\usepackage{hyperref}
\hypersetup{
    colorlinks=true,
    linkcolor=blue!60!black,
    citecolor=green!60!black,
    urlcolor=blue!60!black,
    bookmarksopen=true,
}
\usepackage[capitalise,noabbrev]{cleveref}

% --- Bibliography ---
\usepackage[backend=biber,style=alphabetic,sorting=nyt]{biblatex}
\addbibresource{references.bib}

% --- Glossary and Index ---
\usepackage{makeidx}
\usepackage{glossaries}
\makeindex
\makeglossaries
\loadglsentries{chapters/glossary}

% --- Theorem Environments ---
\newtheorem{theorem}{Theorem}[chapter]
\newtheorem{lemma}[theorem]{Lemma}
\newtheorem{proposition}[theorem]{Proposition}
\newtheorem{corollary}[theorem]{Corollary}
\theoremstyle{definition}
\newtheorem{definition}[theorem]{Definition}
\newtheorem{example}[theorem]{Example}
\theoremstyle{remark}
\newtheorem{remark}[theorem]{Remark}

% --- Custom Commands ---
\newcommand{\vect}[1]{\boldsymbol{#1}}
\newcommand{\mat}[1]{\boldsymbol{#1}}
\newcommand{\transpose}{^\top}
\newcommand{\norm}[1]{\left\lVert#1\right\rVert}
\newcommand{\abs}[1]{\left|#1\right|}

% --- Difficulty Level Commands ---
\newcommand{\difficulty}[1]{%
    \ifstrequal{#1}{beginner}{%
        \textcolor{green!70!black}{\textbf{[Beginner]}}%
    }{%
        \ifstrequal{#1}{intermediate}{%
            \textcolor{orange!70!black}{\textbf{[Intermediate]}}%
        }{%
            \ifstrequal{#1}{advanced}{%
                \textcolor{red!70!black}{\textbf{[Advanced]}}%
            }{%
                \textbf{[#1]}%
            }%
        }%
    }%
}

% --- Title and Author Information ---
\title{\textbf{Deep Learning 101}}
\author{Vu Hung Nguyen}
\date{\today}

% ======================================================================
% BEGIN DOCUMENT
% ======================================================================

\begin{document}

% --- Front Matter ---
\frontmatter

% Title Page
\begin{titlepage}
    \centering
    \vspace*{2cm}
    
    {\Huge\bfseries Deep Learning 101\par}
    \vspace{1.5cm}
    
    {\Large Vu Hung Nguyen\par}
    \vspace{2cm}
    
    {\large A Comprehensive Guide to\\
    Deep Learning Theory and Practice\par}
    
    \vfill
    
    {\large \today\par}
    \vspace{0.5cm}
    
    {\small License: Creative Commons Attribution 4.0 International (CC BY 4.0)\par}
\end{titlepage}

% Copyright Page
\newpage
\thispagestyle{empty}
\vspace*{\fill}
\begin{flushleft}
\textbf{Deep Learning 101}\\
Copyright \copyright~\the\year~Vu Hung Nguyen\\[0.5em]
This work is licensed under the Creative Commons Attribution 4.0 International License.\\
To view a copy of this license, visit \url{http://creativecommons.org/licenses/by/4.0/}\\[0.5em]
First Edition: \today
\end{flushleft}
\vspace*{\fill}
\clearpage

% Table of Contents
\tableofcontents

% Acknowledgements
\chapter*{Acknowledgements}
\addcontentsline{toc}{chapter}{Acknowledgements}
I would like to express my deepest gratitude to the deep learning community for their invaluable contributions to this rapidly evolving field. Special thanks to the pioneers whose groundbreaking work laid the foundation for modern deep learning.

I am grateful to my colleagues, students, and collaborators who have provided feedback, suggestions, and encouragement throughout the writing of this book. Your insights have been instrumental in shaping the content and presentation of this material.

This book draws inspiration from the excellent resources available in the deep learning community, including the seminal work by Goodfellow, Bengio, and Courville, as well as the emerging literature on understanding deep learning.

I would also like to thank my family for their unwavering support and patience during the many hours spent writing and revising this manuscript.

Finally, I am grateful to all readers who engage with this material and contribute to the advancement of deep learning through their research, applications, and education.

Any errors or omissions in this book are entirely my own responsibility.

\vspace{1cm}
\noindent Vu Hung Nguyen\\
\today


% Notation
\chapter*{Notation}
\addcontentsline{toc}{chapter}{Notation}
This book uses the following notation throughout:

\section*{General Notation}

\begin{itemize}[leftmargin=2em]
    \item $a, b, c$ --- scalars (lowercase italic letters)
    \item $\vect{a}, \vect{b}, \vect{c}$ --- vectors (bold lowercase letters)
    \item $\mat{A}, \mat{B}, \mat{C}$ --- matrices (bold uppercase letters)
    \item $\mathcal{A}, \mathcal{B}, \mathcal{C}$ --- sets (calligraphic uppercase)
    \item $a_i$ --- the $i$-th element of vector $\vect{a}$
    \item $A_{ij}$ or $\mat{A}_{ij}$ --- element at row $i$, column $j$ of matrix $\mat{A}$
    \item $\mat{A}\transpose$ --- transpose of matrix $\mat{A}$
    \item $\mat{A}^{-1}$ --- inverse of matrix $\mat{A}$
    \item $\norm{\vect{x}}$ --- norm of vector $\vect{x}$ (typically $L^2$ norm)
    \item $\norm{\vect{x}}_p$ --- $L^p$ norm of vector $\vect{x}$
    \item $\abs{x}$ --- absolute value of scalar $x$
    \item $\mathbb{R}$ --- set of real numbers
    \item $\mathbb{R}^n$ --- $n$-dimensional real vector space
    \item $\mathbb{R}^{m \times n}$ --- set of real $m \times n$ matrices
\end{itemize}

\section*{Probability and Statistics}

\begin{itemize}[leftmargin=2em]
    \item $P(X)$ --- probability distribution over discrete variable $X$
    \item $p(x)$ --- probability density function over continuous variable $x$
    \item $P(X=x)$ or $P(x)$ --- probability that $X$ takes value $x$
    \item $P(X|Y)$ --- conditional probability of $X$ given $Y$
    \item $\mathbb{E}_{x \sim P}[f(x)]$ --- expectation of $f(x)$ with respect to distribution $P$
    \item $\text{Var}(X)$ --- variance of random variable $X$
    \item $\text{Cov}(X, Y)$ --- covariance of random variables $X$ and $Y$
    \item $\mathcal{N}(\mu, \sigma^2)$ --- Gaussian distribution with mean $\mu$ and variance $\sigma^2$
\end{itemize}

\section*{Calculus and Optimization}

\begin{itemize}[leftmargin=2em]
    \item $\frac{dy}{dx}$ or $\frac{\partial y}{\partial x}$ --- derivative of $y$ with respect to $x$
    \item $\nabla_{\vect{x}} f$ --- gradient of function $f$ with respect to $\vect{x}$
    \item $\nabla^2 f$ or $\mat{H}$ --- Hessian matrix (matrix of second derivatives)
    \item $\arg\min_x f(x)$ --- value of $x$ that minimizes $f(x)$
    \item $\arg\max_x f(x)$ --- value of $x$ that maximizes $f(x)$
\end{itemize}

\section*{Machine Learning}

\begin{itemize}[leftmargin=2em]
    \item $\mathcal{D}$ --- dataset
    \item $\mathcal{D}_{\text{train}}$ --- training dataset
    \item $\mathcal{D}_{\text{val}}$ --- validation dataset
    \item $\mathcal{D}_{\text{test}}$ --- test dataset
    \item $n$ --- number of examples in dataset
    \item $m$ --- mini-batch size
    \item $\vect{x}$ --- input vector or feature vector
    \item $y$ --- target or label
    \item $\hat{y}$ --- prediction or estimated output
    \item $\vect{\theta}$ or $\vect{w}$ --- parameters or weights
    \item $\mathcal{L}$ --- loss function
    \item $J$ --- cost function (sum or average of losses)
    \item $\alpha$ or $\eta$ --- learning rate
    \item $\lambda$ --- regularization coefficient
\end{itemize}

\section*{Neural Networks}

\begin{itemize}[leftmargin=2em]
    \item $L$ --- number of layers in a neural network
    \item $n^{[l]}$ --- number of units in layer $l$
    \item $\vect{a}^{[l]}$ --- activations of layer $l$
    \item $\vect{z}^{[l]}$ --- pre-activation values of layer $l$
    \item $\mat{W}^{[l]}$ --- weight matrix for layer $l$
    \item $\vect{b}^{[l]}$ --- bias vector for layer $l$
    \item $f$ or $\sigma$ --- activation function
    \item $g$ --- general function or transformation
\end{itemize}


% --- Main Matter ---
\mainmatter

% Chapter 1: Introduction
% Chapter 1: Introduction

\chapter{Introduction}
\label{chap:introduction}

% Chapter 1, Section 1: What is Deep Learning?

\section{What is Deep Learning?}
\label{sec:what-is-dl}
\difficulty{beginner}

Deep learning is a subfield of machine learning that focuses on learning hierarchical representations of data through artificial \gls{neural-network}s with multiple layers. These networks, inspired by the structure and function of the human brain, have revolutionized numerous fields including \gls{computer-vision}, \gls{natural-language-processing}, speech recognition, and many others.

\subsection{The Rise of Deep Learning}

The resurgence of neural networks, now known as deep learning, can be attributed to several key factors:

\begin{enumerate}
    \item \textbf{Availability of Large Datasets:} The digital age has produced massive amounts of data, providing the fuel needed to train complex models effectively.
    
    \item \textbf{Computational Power:} The advent of Graphics Processing Units (GPUs) and specialized hardware has enabled the training of much larger networks than was previously possible.
    
    \item \textbf{Algorithmic Innovations:} Improvements in optimization algorithms, regularization techniques, and network architectures have made it possible to train very deep networks.
    
    \item \textbf{Open-Source Software:} Frameworks like TensorFlow, PyTorch, and others have democratized access to deep learning tools.
\end{enumerate}

\subsection{Key Characteristics}

Deep learning differs from traditional machine learning in several important ways:

\begin{itemize}
    \item \textbf{Automatic Feature Learning:} Unlike traditional approaches that require manual feature engineering, deep learning models automatically learn relevant features from raw data.
    
    \item \textbf{Hierarchical Representations:} Deep networks learn multiple levels of representation, from low-level features (e.g., edges in images) to high-level concepts (e.g., object categories).
    
    \item \textbf{End-to-End Learning:} Deep learning often enables end-to-end learning, where the entire system is trained jointly rather than in separate stages.
    
    \item \textbf{Scalability:} Deep learning models can continue to improve with more data and computational resources.
\end{itemize}

\subsection{Applications}

Deep learning has achieved remarkable success in numerous domains:

\begin{description}
    \item[Computer Vision:] Image classification, object detection, semantic segmentation, facial recognition, and image generation.
    
    \item[Natural Language Processing:] Machine translation, sentiment analysis, question answering, text generation, and language understanding.
    
    \item[Speech and Audio:] Speech recognition, speaker identification, music generation, and audio synthesis.
    
    \item[Healthcare:] Medical image analysis, drug discovery, disease prediction, and personalized medicine.
    
    \item[Robotics:] Autonomous navigation, manipulation, and decision-making.
    
    \item[Game Playing:] Achieving superhuman performance in complex games like Go, Chess, and video games.
\end{description}

The impact of deep learning extends far beyond these applications, touching virtually every aspect of modern technology and scientific research.

% Index entries
\index{deep learning!introduction}
\index{machine learning!deep learning}
\index{neural networks!deep learning}
\index{artificial intelligence!deep learning}

% Chapter 1, Section 2: Historical Context

\section{Historical Context}
\label{sec:historical-context}

The history of deep learning is intertwined with the broader history of artificial intelligence and neural networks. Understanding this context helps us appreciate the current state of the field and its future directions.

\subsection{The Perceptron Era (1940s-1960s)}

The foundations of neural networks were laid in the 1940s with the work of Warren McCulloch and Walter Pitts, who created a computational model of a neuron. In 1958, Frank Rosenblatt invented the Perceptron, an algorithm for learning a binary classifier.

The Perceptron showed promise but faced significant limitations. In 1969, Marvin Minsky and Seymour Papert's book \emph{Perceptrons} demonstrated that single-layer perceptrons could not solve non-linearly separable problems like XOR, leading to the first ``AI winter.''

\subsection{The Backpropagation Revolution (1980s)}

The field was revitalized in the 1980s with the rediscovery and popularization of the backpropagation algorithm by David Rumelhart, Geoffrey Hinton, and Ronald Williams. This algorithm enabled the training of multi-layer networks, overcoming the limitations of single-layer perceptrons.

Key developments during this period include:
\begin{itemize}
    \item Convolutional Neural Networks (CNNs) by Yann LeCun
    \item Recurrent Neural Networks (RNNs) for sequential data
    \item Improved optimization techniques
\end{itemize}

\subsection{The Second AI Winter (1990s-2000s)}

Despite theoretical advances, neural networks fell out of favor in the 1990s due to:
\begin{itemize}
    \item Limited computational resources
    \item Difficulty training deep networks (vanishing gradient problem)
    \item Success of alternative methods like Support Vector Machines (SVMs)
    \item Lack of large labeled datasets
\end{itemize}

During this period, the term ``deep learning'' was coined to distinguish multi-layer neural networks from shallow architectures.

\subsection{The Deep Learning Renaissance (2006-Present)}

The modern era of deep learning began around 2006 with several breakthrough papers:

\begin{enumerate}
    \item \textbf{2006:} Geoffrey Hinton and colleagues introduced Deep Belief Networks (DBNs) and showed that deep networks could be trained using layer-wise pretraining.
    
    \item \textbf{2009:} Large-scale GPU computing for neural networks became practical, dramatically reducing training times.
    
    \item \textbf{2012:} AlexNet won the ImageNet competition by a large margin, demonstrating the power of deep CNNs trained on GPUs.
    
    \item \textbf{2014-2016:} Sequence-to-sequence models and attention mechanisms revolutionized NLP.
    
    \item \textbf{2017-Present:} Transformer architectures and large language models like GPT and BERT have achieved unprecedented performance.
\end{enumerate}

\subsection{Key Milestones}

\begin{table}[h]
\centering
\caption{Major milestones in deep learning history}
\label{tab:milestones}
\small
\begin{tabular}{@{}ll@{}}
\toprule
\textbf{Year} & \textbf{Milestone} \\
\midrule
1943 & McCulloch-Pitts neuron model \\
1958 & Rosenblatt's Perceptron \\
1986 & Backpropagation popularized \\
1989 & LeCun's CNN for handwritten digits \\
1997 & LSTM networks introduced \\
2006 & Deep Belief Networks \\
2012 & AlexNet wins ImageNet \\
2014 & Generative Adversarial Networks (GANs) \\
2017 & Transformer architecture \\
2018 & BERT for NLP \\
2020 & GPT-3 and large language models \\
\bottomrule
\end{tabular}
\end{table}

This historical perspective shows that deep learning is built on decades of research, with periods of both enthusiasm and skepticism. The current success is the result of persistent research, technological advances, and the convergence of multiple enabling factors.

% Chapter 1, Section 3: Fundamental Concepts

\section{Fundamental Concepts}
\label{sec:fundamental-concepts}

Before diving into the technical details, it is essential to understand several fundamental concepts that underpin deep learning.

\subsection{Learning from Data}

At its core, deep learning is about learning from data. Given a dataset $\mathcal{D} = \{(\vect{x}_1, y_1), (\vect{x}_2, y_2), \ldots, (\vect{x}_n, y_n)\}$, where $\vect{x}_i$ represents input features and $y_i$ represents corresponding targets, the goal is to learn a function $f: \mathcal{X} \rightarrow \mathcal{Y}$ that maps inputs to outputs.

\begin{definition}[Supervised Learning]
In supervised learning, we have access to labeled examples where both inputs and desired outputs are known. The model learns to predict outputs for new, unseen inputs.
\end{definition}

\begin{definition}[Unsupervised Learning]
In unsupervised learning, we only have inputs without explicit labels. The model learns to discover patterns, structure, or representations in the data.
\end{definition}

\begin{definition}[Reinforcement Learning]
In reinforcement learning, an agent learns to make decisions by interacting with an environment and receiving rewards or penalties.
\end{definition}

\subsection{The Learning Process}

The learning process in deep learning typically involves:

\begin{enumerate}
    \item \textbf{Model Definition:} Specify the architecture of the neural network, including the number of layers, types of layers, and activation functions.
    
    \item \textbf{Loss Function:} Define a loss function $\mathcal{L}(\hat{y}, y)$ that measures the discrepancy between predictions $\hat{y}$ and true targets $y$.
    
    \item \textbf{Optimization:} Use an optimization algorithm (typically gradient descent variants) to adjust the model parameters $\vect{\theta}$ to minimize the loss:
    \begin{equation}
        \vect{\theta}^* = \arg\min_{\vect{\theta}} \frac{1}{n} \sum_{i=1}^{n} \mathcal{L}(f(\vect{x}_i; \vect{\theta}), y_i)
    \end{equation}
    
    \item \textbf{Evaluation:} Assess the model's performance on held-out test data to estimate generalization.
\end{enumerate}

\subsection{Neural Networks as Universal Approximators}

One of the remarkable properties of neural networks is their ability to approximate a wide range of functions.

\begin{theorem}[Universal Approximation Theorem (informal)]
A neural network with a single hidden layer containing a sufficient number of neurons can approximate any continuous function on a compact subset of $\mathbb{R}^n$ to arbitrary accuracy.
\end{theorem}

While this theorem provides theoretical justification for using neural networks, in practice, deep networks with multiple layers are often more efficient and effective than shallow but wide networks.

\subsection{Representation Learning}

A key advantage of deep learning is automatic feature learning, also known as representation learning.

\begin{itemize}
    \item \textbf{Lower Layers:} Learn simple, general features (e.g., edges, textures in images)
    \item \textbf{Middle Layers:} Combine simple features into more complex patterns (e.g., object parts)
    \item \textbf{Higher Layers:} Learn abstract, task-specific representations (e.g., object categories)
\end{itemize}

This hierarchical feature learning is what makes deep networks particularly powerful for complex tasks.

\subsection{Generalization and Overfitting}

A critical challenge in machine learning is ensuring that models generalize well to new data.

\begin{definition}[Overfitting]
Overfitting occurs when a model learns the training data too well, including noise and spurious patterns, leading to poor performance on new data.
\end{definition}

\begin{definition}[Underfitting]
Underfitting occurs when a model is too simple to capture the underlying patterns in the data, resulting in poor performance on both training and test data.
\end{definition}

The goal is to find the right balance between model complexity and generalization ability, often visualized by the bias-variance tradeoff:

\begin{equation}
    \text{Expected Error} = \text{Bias}^2 + \text{Variance} + \text{Irreducible Error}
\end{equation}

Understanding these fundamental concepts provides a solid foundation for exploring the technical details of deep learning in subsequent chapters.

% Chapter 1, Section 4: Structure of This Book

\section{Structure of This Book}
\label{sec:book-structure}

This book is organized into three main parts, each building upon the previous one to provide a comprehensive understanding of deep learning.

\subsection{Part I: Basic Math and Machine Learning Foundation}

The first part establishes the mathematical and machine learning foundations necessary for understanding deep learning:

\begin{description}
    \item[Chapter 2: Linear Algebra] Covers vectors, matrices, and operations essential for understanding neural network computations.
    
    \item[Chapter 3: Probability and Information Theory] Introduces probability distributions, expectation, information theory concepts, and their relevance to machine learning.
    
    \item[Chapter 4: Numerical Computation] Discusses numerical optimization, gradient-based optimization, and computational considerations.
    
    \item[Chapter 5: Classical Machine Learning Algorithms] Reviews traditional machine learning methods that provide context and motivation for deep learning approaches.
\end{description}

\subsection{Part II: Practical Deep Networks}

The second part focuses on practical aspects of designing, training, and deploying deep neural networks:

\begin{description}
    \item[Chapter 6: Deep Feedforward Networks] Introduces the fundamental building blocks of deep learning, including multilayer perceptrons and activation functions.
    
    \item[Chapter 7: Regularization for Deep Learning] Explores techniques to improve generalization and prevent overfitting.
    
    \item[Chapter 8: Optimization for Training Deep Models] Covers modern optimization algorithms and training strategies.
    
    \item[Chapter 9: Convolutional Networks] Details architectures specifically designed for processing grid-structured data like images.
    
    \item[Chapter 10: Sequence Modeling] Examines recurrent and recursive networks for sequential and temporal data.
    
    \item[Chapter 11: Practical Methodology] Provides guidelines for successfully applying deep learning to real-world problems.
    
    \item[Chapter 12: Applications] Showcases deep learning applications across various domains.
\end{description}

\subsection{Part III: Deep Learning Research}

The third part delves into advanced topics and current research directions:

\begin{description}
    \item[Chapter 13: Linear Factor Models] Introduces probabilistic models with linear structure.
    
    \item[Chapter 14: Autoencoders] Explores unsupervised learning through reconstruction-based models.
    
    \item[Chapter 15: Representation Learning] Discusses learning meaningful representations from data.
    
    \item[Chapter 16: Structured Probabilistic Models] Covers graphical models and their integration with deep learning.
    
    \item[Chapter 17: Monte Carlo Methods] Introduces sampling-based approaches for probabilistic inference.
    
    \item[Chapter 18: Confronting the Partition Function] Addresses computational challenges in probabilistic models.
    
    \item[Chapter 19: Approximate Inference] Explores methods for tractable inference in complex models.
    
    \item[Chapter 20: Deep Generative Models] Examines modern approaches to generating new data samples.
\end{description}

\subsection{How to Use This Book}

This book is designed to accommodate different learning paths:

\begin{itemize}
    \item \textbf{For Beginners:} Start with Part I to build a strong foundation, then proceed sequentially through Part II.
    
    \item \textbf{For Practitioners:} If you have a solid mathematical background, you may skip or skim Part I and focus on Parts II and III.
    
    \item \textbf{For Researchers:} Part III provides advanced material relevant to current research directions in deep learning.
    
    \item \textbf{For Specific Topics:} Each chapter is relatively self-contained, allowing you to focus on topics most relevant to your interests or needs.
\end{itemize}

Throughout the book, we balance theoretical rigor with practical insights, providing both mathematical foundations and intuitive explanations. Code examples and exercises (when available) help reinforce concepts and develop practical skills.

% Chapter 1, Section 5: Prerequisites and Resources

\section{Prerequisites and Resources}
\label{sec:prerequisites}

To get the most out of this book, certain prerequisites are helpful, though not absolutely necessary. This section outlines the assumed background and provides resources for filling any gaps.

\subsection{Mathematical Prerequisites}

While we introduce key concepts in Part I, familiarity with the following topics will be beneficial:

\begin{itemize}
    \item \textbf{Linear Algebra:} Vectors, matrices, eigenvalues, and eigenvectors
    \item \textbf{Calculus:} Derivatives, partial derivatives, chain rule, and basic optimization
    \item \textbf{Probability:} Basic probability theory, random variables, and common distributions
    \item \textbf{Statistics:} Mean, variance, covariance, and basic statistical inference
\end{itemize}

For readers needing to review these topics, we recommend:
\begin{itemize}
    \item ``Linear Algebra Done Right'' by Sheldon Axler
    \item ``All of Statistics'' by Larry Wasserman
    \item Online resources: Khan Academy, MIT OpenCourseWare
\end{itemize}

\subsection{Programming and Machine Learning Background}

Basic programming knowledge is helpful for implementing and experimenting with the concepts:

\begin{itemize}
    \item \textbf{Python Programming:} Understanding of basic syntax, data structures, and functions
    \item \textbf{NumPy:} Familiarity with array operations is useful
    \item \textbf{Machine Learning Basics:} General understanding of supervised learning, training/test splits, and evaluation metrics
\end{itemize}

Recommended resources:
\begin{itemize}
    \item ``Python for Data Analysis'' by Wes McKinney
    \item ``Hands-On Machine Learning'' by Aurélien Géron
    \item scikit-learn documentation and tutorials
\end{itemize}

\subsection{Deep Learning Frameworks}

While this book focuses on concepts rather than specific implementations, familiarity with a deep learning framework is valuable:

\begin{itemize}
    \item \textbf{PyTorch:} Popular for research and prototyping
    \item \textbf{TensorFlow/Keras:} Widely used in industry
    \item \textbf{JAX:} Emerging framework for research
\end{itemize}

Official documentation and tutorials for these frameworks provide excellent hands-on learning opportunities.

\subsection{Additional Resources}

To complement this book, consider exploring:

\begin{description}
    \item[Classic Textbooks:]
    \begin{itemize}
        \item ``Deep Learning'' by Goodfellow, Bengio, and Courville
        \item ``Pattern Recognition and Machine Learning'' by Christopher Bishop
    \end{itemize}
    
    \item[Online Courses:]
    \begin{itemize}
        \item Coursera: Deep Learning Specialization (Andrew Ng)
        \item Fast.ai: Practical Deep Learning for Coders
        \item Stanford CS231n, CS224n (lecture notes and videos)
    \end{itemize}
    
    \item[Research Papers:]
    \begin{itemize}
        \item ArXiv.org for latest research
        \item Papers with Code for implementations
        \item Conference proceedings: NeurIPS, ICML, ICLR, CVPR
    \end{itemize}
    
    \item[Community Resources:]
    \begin{itemize}
        \item Distill.pub for interactive explanations
        \item Towards Data Science (Medium)
        \item Reddit: r/MachineLearning
        \item Twitter/X: Follow leading researchers
    \end{itemize}
\end{description}

\subsection{A Note on Exercises}

Throughout this book, we include exercises at the end of chapters (when available) to help reinforce understanding. We encourage readers to:

\begin{enumerate}
    \item Work through exercises actively rather than just reading solutions
    \item Implement concepts in code to deepen understanding
    \item Experiment with variations to explore the behavior of models
    \item Collaborate with others and discuss concepts
\end{enumerate}

Remember that deep learning is best learned through a combination of theoretical understanding and practical experience. Don't be discouraged if some concepts take time to fully grasp---this is normal and part of the learning process.

\subsection{Getting Help}

If you encounter difficulties or have questions:
\begin{itemize}
    \item Review the notation section and relevant earlier chapters
    \item Consult the bibliography for additional perspectives
    \item Engage with online communities for discussions
    \item Implement toy examples to build intuition
    \item Be patient---deep learning is a rapidly evolving field with many subtleties
\end{itemize}

With these prerequisites and resources in mind, you are well-equipped to begin your deep learning journey. Let us now proceed to the mathematical foundations in Part I.



% ======================================================================
% PART I: Basic Math and Machine Learning Foundation
% ======================================================================
\part{Basic Math and Machine Learning Foundation}

% Chapter 2: Linear Algebra
% Chapter 2: Linear Algebra

\chapter{Linear Algebra}
\label{chap:linear-algebra}

% Chapter 2, Section 1: Scalars, Vectors, Matrices, and Tensors

\section{Scalars, Vectors, Matrices, and Tensors}
\label{sec:scalars-vectors-matrices-tensors}

Linear algebra provides the mathematical framework for understanding and implementing deep learning algorithms. We begin with the basic objects that form the foundation of this framework.

\subsection{Scalars}

A \emph{scalar} is a single number, in contrast to objects that contain multiple numbers. We typically denote scalars with lowercase italic letters, such as $a$, $n$, or $x$.

\begin{example}
The learning rate $\alpha = 0.01$ is a scalar. The number of training examples $n = 1000$ is also a scalar.
\end{example}

In deep learning, scalars are often real numbers ($a \in \mathbb{R}$), but they can also be integers, complex numbers, or elements of other fields depending on the context.

\subsection{Vectors}

A \emph{vector} is an array of numbers arranged in order. We identify each individual number in the vector by its position in the ordering. We denote vectors with bold lowercase letters, such as $\vect{x}$, $\vect{y}$, or $\vect{w}$.

\begin{definition}[Vector]
A vector $\vect{x} \in \mathbb{R}^n$ is an ordered collection of $n$ real numbers:
\begin{equation}
    \vect{x} = \begin{bmatrix} x_1 \\ x_2 \\ \vdots \\ x_n \end{bmatrix}
\end{equation}
where $x_i$ denotes the $i$-th element of $\vect{x}$.
\end{definition}

\begin{example}
A feature vector for a house might be:
\begin{equation}
    \vect{x} = \begin{bmatrix} 2000 \\ 3 \\ 2 \\ 50 \end{bmatrix}
\end{equation}
representing square footage, number of bedrooms, number of bathrooms, and age in years.
\end{example}

\subsection{Matrices}

A \emph{matrix} is a 2-D array of numbers, where each element is identified by two indices. We denote matrices with bold uppercase letters such as $\mat{A}$, $\mat{W}$, or $\mat{X}$.

\begin{definition}[Matrix]
A matrix $\mat{A} \in \mathbb{R}^{m \times n}$ is a rectangular array of real numbers with $m$ rows and $n$ columns:
\begin{equation}
    \mat{A} = \begin{bmatrix}
        A_{11} & A_{12} & \cdots & A_{1n} \\
        A_{21} & A_{22} & \cdots & A_{2n} \\
        \vdots & \vdots & \ddots & \vdots \\
        A_{m1} & A_{m2} & \cdots & A_{mn}
    \end{bmatrix}
\end{equation}
where $A_{ij}$ denotes the element at row $i$ and column $j$.
\end{definition}

\begin{example}
A matrix of training examples where each row is a feature vector:
\begin{equation}
    \mat{X} = \begin{bmatrix}
        x_{11} & x_{12} & x_{13} \\
        x_{21} & x_{22} & x_{23} \\
        x_{31} & x_{32} & x_{33}
    \end{bmatrix}
\end{equation}
Here, $\mat{X} \in \mathbb{R}^{3 \times 3}$ contains 3 examples with 3 features each.
\end{example}

\subsection{Tensors}

A \emph{tensor} is an array with more than two axes. While scalars are 0-D tensors, vectors are 1-D tensors, and matrices are 2-D tensors, we typically reserve the term ``tensor'' for arrays with three or more dimensions.

\begin{definition}[Tensor]
A tensor $\mathcal{A} \in \mathbb{R}^{n_1 \times n_2 \times \cdots \times n_k}$ is a $k$-dimensional array where elements are identified by $k$ indices: $\mathcal{A}_{i_1, i_2, \ldots, i_k}$.
\end{definition}

\begin{example}
A batch of color images can be represented as a 4-D tensor:
\begin{equation}
    \mathcal{X} \in \mathbb{R}^{B \times H \times W \times C}
\end{equation}
where $B$ is the batch size, $H$ and $W$ are height and width, and $C$ is the number of color channels (e.g., 3 for RGB).
\end{example}

\subsection{Notation Conventions}

Throughout this book, we adopt the following conventions:
\begin{itemize}
    \item Scalars: lowercase italic ($a$, $b$, $x$)
    \item Vectors: bold lowercase ($\vect{a}$, $\vect{x}$, $\vect{w}$)
    \item Matrices: bold uppercase ($\mat{A}$, $\mat{X}$, $\mat{W}$)
    \item Tensors: calligraphic uppercase ($\mathcal{A}$, $\mathcal{X}$)
\end{itemize}

Understanding these fundamental objects and their properties is essential for working with the mathematical formulations of deep learning algorithms.

% Chapter 2, Section 2: Matrix Operations

\section{Matrix Operations}
\label{sec:matrix-operations}

Matrix operations form the computational backbone of neural networks. Understanding these operations is crucial for implementing and analyzing deep learning algorithms.

\subsection{Matrix Addition and Scalar Multiplication}

Matrices of the same dimensions can be added element-wise:

\begin{definition}[Matrix Addition]
Given $\mat{A}, \mat{B} \in \mathbb{R}^{m \times n}$, their sum $\mat{C} = \mat{A} + \mat{B}$ is defined as:
\begin{equation}
    C_{ij} = A_{ij} + B_{ij}
\end{equation}
for all $i = 1, \ldots, m$ and $j = 1, \ldots, n$.
\end{definition}

\begin{definition}[Scalar Multiplication]
Given a scalar $\alpha \in \mathbb{R}$ and a matrix $\mat{A} \in \mathbb{R}^{m \times n}$, the product $\mat{B} = \alpha\mat{A}$ is:
\begin{equation}
    B_{ij} = \alpha A_{ij}
\end{equation}
\end{definition}

\subsection{Matrix Transpose}

The transpose is a fundamental operation that exchanges rows and columns.

\begin{definition}[Transpose]
The transpose of a matrix $\mat{A} \in \mathbb{R}^{m \times n}$ is a matrix $\mat{A}\transpose \in \mathbb{R}^{n \times m}$ where:
\begin{equation}
    (\mat{A}\transpose)_{ij} = A_{ji}
\end{equation}
\end{definition}

Properties of transpose:
\begin{align}
    (\mat{A}\transpose)\transpose &= \mat{A} \\
    (\mat{A} + \mat{B})\transpose &= \mat{A}\transpose + \mat{B}\transpose \\
    (\alpha\mat{A})\transpose &= \alpha\mat{A}\transpose \\
    (\mat{AB})\transpose &= \mat{B}\transpose\mat{A}\transpose
\end{align}

\subsection{Matrix Multiplication}

Matrix multiplication is central to neural network computations.

\begin{definition}[Matrix Multiplication]
Given $\mat{A} \in \mathbb{R}^{m \times n}$ and $\mat{B} \in \mathbb{R}^{n \times p}$, their product $\mat{C} = \mat{AB} \in \mathbb{R}^{m \times p}$ is defined as:
\begin{equation}
    C_{ij} = \sum_{k=1}^{n} A_{ik}B_{kj}
\end{equation}
\end{definition}

\begin{example}
Consider the multiplication:
\begin{equation}
    \begin{bmatrix} 1 & 2 \\ 3 & 4 \end{bmatrix}
    \begin{bmatrix} 5 & 6 \\ 7 & 8 \end{bmatrix}
    = \begin{bmatrix} 19 & 22 \\ 43 & 50 \end{bmatrix}
\end{equation}
where $C_{11} = 1 \cdot 5 + 2 \cdot 7 = 19$.
\end{example}

Important properties:
\begin{itemize}
    \item \textbf{Associative:} $(\mat{AB})\mat{C} = \mat{A}(\mat{BC})$
    \item \textbf{Distributive:} $\mat{A}(\mat{B} + \mat{C}) = \mat{AB} + \mat{AC}$
    \item \textbf{Not commutative:} Generally $\mat{AB} \neq \mat{BA}$
\end{itemize}

\subsection{Element-wise (Hadamard) Product}

The element-wise product is denoted by $\odot$ and operates on corresponding elements.

\begin{definition}[Hadamard Product]
Given $\mat{A}, \mat{B} \in \mathbb{R}^{m \times n}$, the Hadamard product $\mat{C} = \mat{A} \odot \mat{B}$ is:
\begin{equation}
    C_{ij} = A_{ij}B_{ij}
\end{equation}
\end{definition}

This operation is common in neural networks, particularly in activation functions and gating mechanisms.

\subsection{Matrix-Vector Products}

When multiplying a matrix by a vector, we can view it as a special case of matrix multiplication:

\begin{equation}
    \mat{A}\vect{x} = \vect{b}
\end{equation}

where $\mat{A} \in \mathbb{R}^{m \times n}$, $\vect{x} \in \mathbb{R}^n$, and $\vect{b} \in \mathbb{R}^m$.

This operation is fundamental in neural networks, where it represents the linear transformation:
\begin{equation}
    b_i = \sum_{j=1}^{n} A_{ij}x_j
\end{equation}

\subsection{Dot Product}

The dot product (or inner product) of two vectors is a special case of matrix multiplication:

\begin{definition}[Dot Product]
For vectors $\vect{x}, \vect{y} \in \mathbb{R}^n$, their dot product is:
\begin{equation}
    \vect{x} \cdot \vect{y} = \vect{x}\transpose\vect{y} = \sum_{i=1}^{n} x_i y_i
\end{equation}
\end{definition}

The dot product has geometric interpretation:
\begin{equation}
    \vect{x} \cdot \vect{y} = \norm{\vect{x}} \norm{\vect{y}} \cos\theta
\end{equation}
where $\theta$ is the angle between the vectors.

\subsection{Computational Complexity}

Understanding computational costs is important for efficient implementation:

\begin{itemize}
    \item Matrix-matrix multiplication $\mat{A} \in \mathbb{R}^{m \times n}$, $\mat{B} \in \mathbb{R}^{n \times p}$: $O(mnp)$ operations
    \item Matrix-vector multiplication: $O(mn)$ operations
    \item Element-wise operations: $O(mn)$ operations
\end{itemize}

These operations can be efficiently parallelized on modern hardware (GPUs, TPUs), which is one reason deep learning has become practical.

% Chapter 2, Section 3: Identity and Inverse Matrices

\section{Identity and Inverse Matrices}
\label{sec:identity-inverse}

Special matrices play important roles in linear algebra and deep learning. The identity matrix and matrix inverses are among the most fundamental.

\subsection{Identity Matrix}

The identity matrix is the matrix analog of the number 1.

\begin{definition}[Identity Matrix]
The identity matrix $\mat{I}_n \in \mathbb{R}^{n \times n}$ is a square matrix with ones on the diagonal and zeros elsewhere:
\begin{equation}
    \mat{I}_n = \begin{bmatrix}
        1 & 0 & \cdots & 0 \\
        0 & 1 & \cdots & 0 \\
        \vdots & \vdots & \ddots & \vdots \\
        0 & 0 & \cdots & 1
    \end{bmatrix}
\end{equation}
Formally, $(\mat{I}_n)_{ij} = \delta_{ij}$ where $\delta_{ij}$ is the Kronecker delta:
\begin{equation}
    \delta_{ij} = \begin{cases}
        1 & \text{if } i = j \\
        0 & \text{if } i \neq j
    \end{cases}
\end{equation}
\end{definition}

The key property of the identity matrix is:
\begin{equation}
    \mat{I}_n\mat{A} = \mat{A}\mat{I}_n = \mat{A}
\end{equation}
for any matrix $\mat{A} \in \mathbb{R}^{n \times n}$.

\subsection{Matrix Inverse}

The inverse of a matrix, when it exists, allows us to solve systems of linear equations.

\begin{definition}[Matrix Inverse]
A square matrix $\mat{A} \in \mathbb{R}^{n \times n}$ is \emph{invertible} (or \emph{non-singular}) if there exists a matrix $\mat{A}^{-1} \in \mathbb{R}^{n \times n}$ such that:
\begin{equation}
    \mat{A}^{-1}\mat{A} = \mat{A}\mat{A}^{-1} = \mat{I}_n
\end{equation}
\end{definition}

\begin{example}
The matrix $\mat{A} = \begin{bmatrix} 2 & 1 \\ 1 & 1 \end{bmatrix}$ has inverse:
\begin{equation}
    \mat{A}^{-1} = \begin{bmatrix} 1 & -1 \\ -1 & 2 \end{bmatrix}
\end{equation}
We can verify: $\mat{A}\mat{A}^{-1} = \begin{bmatrix} 1 & 0 \\ 0 & 1 \end{bmatrix} = \mat{I}_2$.
\end{example}

\subsection{Properties of Inverses}

If $\mat{A}$ and $\mat{B}$ are invertible, then:

\begin{align}
    (\mat{A}^{-1})^{-1} &= \mat{A} \\
    (\mat{AB})^{-1} &= \mat{B}^{-1}\mat{A}^{-1} \\
    (\mat{A}\transpose)^{-1} &= (\mat{A}^{-1})\transpose
\end{align}

\subsection{Solving Linear Systems}

The inverse allows us to solve systems of linear equations. Given $\mat{A}\vect{x} = \vect{b}$, if $\mat{A}$ is invertible:
\begin{equation}
    \vect{x} = \mat{A}^{-1}\vect{b}
\end{equation}

However, computing inverses is expensive ($O(n^3)$ for dense matrices) and numerically unstable. In practice, we often use more efficient methods like LU decomposition or iterative solvers.

\subsection{Conditions for Invertibility}

A matrix $\mat{A} \in \mathbb{R}^{n \times n}$ is invertible if and only if:
\begin{itemize}
    \item Its determinant is non-zero: $\det(\mat{A}) \neq 0$
    \item Its columns (and rows) are linearly independent
    \item It has full rank: $\text{rank}(\mat{A}) = n$
    \item Its null space contains only the zero vector
\end{itemize}

\subsection{Singular Matrices}

Matrices that are not invertible are called \emph{singular} or \emph{degenerate}.

\begin{example}
The matrix $\mat{A} = \begin{bmatrix} 1 & 2 \\ 2 & 4 \end{bmatrix}$ is singular because its rows are linearly dependent (the second row is twice the first). Its determinant is $\det(\mat{A}) = 4 - 4 = 0$.
\end{example}

Singular matrices arise in deep learning when:
\begin{itemize}
    \item Features are perfectly correlated
    \item The model is overparameterized
    \item Numerical precision issues occur
\end{itemize}

\subsection{Pseudo-inverse}

For non-square or singular matrices, we can use the Moore-Penrose pseudo-inverse $\mat{A}^+$, which provides a generalized notion of inversion. The pseudo-inverse is particularly useful in least squares problems and is discussed further in later chapters.

\subsection{Practical Considerations}

In deep learning implementations:
\begin{itemize}
    \item Avoid explicitly computing matrix inverses when possible
    \item Use numerically stable algorithms (e.g., QR decomposition, SVD)
    \item Add regularization to ensure invertibility (e.g., $(\mat{A}\transpose\mat{A} + \lambda\mat{I})^{-1}$)
    \item Leverage optimized linear algebra libraries (BLAS, LAPACK, cuBLAS)
\end{itemize}

% Chapter 2, Section 4: Linear Dependence and Span

\section{Linear Dependence and Span}
\label{sec:linear-dependence-span}

Understanding linear independence and span is crucial for analyzing the capacity and expressiveness of neural networks.

\subsection{Linear Combinations}

A \emph{linear combination} of vectors $\vect{v}_1, \vect{v}_2, \ldots, \vect{v}_n$ is any vector of the form:
\begin{equation}
    \vect{v} = a_1\vect{v}_1 + a_2\vect{v}_2 + \cdots + a_n\vect{v}_n
\end{equation}
where $a_1, a_2, \ldots, a_n$ are scalars called \emph{coefficients}.

\begin{example}
If $\vect{v}_1 = \begin{bmatrix} 1 \\ 0 \end{bmatrix}$ and $\vect{v}_2 = \begin{bmatrix} 0 \\ 1 \end{bmatrix}$, then any vector in $\mathbb{R}^2$ can be written as a linear combination:
\begin{equation}
    \begin{bmatrix} x \\ y \end{bmatrix} = x\vect{v}_1 + y\vect{v}_2
\end{equation}
\end{example}

\subsection{Span}

\begin{definition}[Span]
The \emph{span} of a set of vectors $\{\vect{v}_1, \vect{v}_2, \ldots, \vect{v}_n\}$ is the set of all possible linear combinations of these vectors:
\begin{equation}
    \text{span}(\{\vect{v}_1, \ldots, \vect{v}_n\}) = \left\{ \sum_{i=1}^{n} a_i\vect{v}_i \;\middle|\; a_i \in \mathbb{R} \right\}
\end{equation}
\end{definition}

The span defines all vectors that can be reached by scaling and adding the given vectors.

\begin{example}
In $\mathbb{R}^3$, the span of $\vect{v}_1 = \begin{bmatrix} 1 \\ 0 \\ 0 \end{bmatrix}$ and $\vect{v}_2 = \begin{bmatrix} 0 \\ 1 \\ 0 \end{bmatrix}$ is the $xy$-plane:
\begin{equation}
    \text{span}(\{\vect{v}_1, \vect{v}_2\}) = \left\{ \begin{bmatrix} x \\ y \\ 0 \end{bmatrix} \;\middle|\; x, y \in \mathbb{R} \right\}
\end{equation}
\end{example}

\subsection{Linear Independence}

\begin{definition}[Linear Independence]
A set of vectors $\{\vect{v}_1, \vect{v}_2, \ldots, \vect{v}_n\}$ is \emph{linearly independent} if no vector can be written as a linear combination of the others. Formally, the only solution to:
\begin{equation}
    a_1\vect{v}_1 + a_2\vect{v}_2 + \cdots + a_n\vect{v}_n = \vect{0}
\end{equation}
is $a_1 = a_2 = \cdots = a_n = 0$.
\end{definition}

If a set of vectors is not linearly independent, it is \emph{linearly dependent}.

\begin{example}[Linear Dependence]
The vectors $\vect{v}_1 = \begin{bmatrix} 1 \\ 2 \end{bmatrix}$, $\vect{v}_2 = \begin{bmatrix} 2 \\ 4 \end{bmatrix}$ are linearly dependent because $\vect{v}_2 = 2\vect{v}_1$.
\end{example}

\begin{example}[Linear Independence]
The standard basis vectors $\vect{e}_1 = \begin{bmatrix} 1 \\ 0 \end{bmatrix}$ and $\vect{e}_2 = \begin{bmatrix} 0 \\ 1 \end{bmatrix}$ are linearly independent.
\end{example}

\subsection{Basis}

\begin{definition}[Basis]
A \emph{basis} for a vector space $V$ is a set of linearly independent vectors that span $V$. Every vector in $V$ can be uniquely expressed as a linear combination of basis vectors.
\end{definition}

\begin{example}[Standard Basis]
The standard basis for $\mathbb{R}^3$ is:
\begin{equation}
    \vect{e}_1 = \begin{bmatrix} 1 \\ 0 \\ 0 \end{bmatrix}, \quad
    \vect{e}_2 = \begin{bmatrix} 0 \\ 1 \\ 0 \end{bmatrix}, \quad
    \vect{e}_3 = \begin{bmatrix} 0 \\ 0 \\ 1 \end{bmatrix}
\end{equation}
\end{example}

\subsection{Dimension and Rank}

\begin{definition}[Dimension]
The \emph{dimension} of a vector space is the number of vectors in any basis for that space. We write $\dim(V)$ for the dimension of space $V$.
\end{definition}

\begin{definition}[Rank]
The \emph{rank} of a matrix $\mat{A}$ is the dimension of the space spanned by its columns (column rank) or rows (row rank). For any matrix, column rank equals row rank, so we simply refer to ``the rank.''
\end{definition}

Properties of rank:
\begin{itemize}
    \item $\text{rank}(\mat{A}) \leq \min(m, n)$ for $\mat{A} \in \mathbb{R}^{m \times n}$
    \item $\text{rank}(\mat{AB}) \leq \min(\text{rank}(\mat{A}), \text{rank}(\mat{B}))$
    \item $\mat{A}$ is invertible if and only if $\text{rank}(\mat{A}) = n$ (full rank)
\end{itemize}

\subsection{Column Space and Null Space}

\begin{definition}[Column Space]
The \emph{column space} (or \emph{range}) of a matrix $\mat{A} \in \mathbb{R}^{m \times n}$ is the span of its columns:
\begin{equation}
    \text{Col}(\mat{A}) = \{\mat{A}\vect{x} \mid \vect{x} \in \mathbb{R}^n\}
\end{equation}
The dimension of the column space is the rank of $\mat{A}$.
\end{definition}

\begin{definition}[Null Space]
The \emph{null space} (or \emph{kernel}) of $\mat{A}$ is the set of all vectors that map to zero:
\begin{equation}
    \text{Null}(\mat{A}) = \{\vect{x} \in \mathbb{R}^n \mid \mat{A}\vect{x} = \vect{0}\}
\end{equation}
\end{definition}

\subsection{Relevance to Deep Learning}

These concepts are fundamental to understanding:
\begin{itemize}
    \item \textbf{Model Capacity:} The expressiveness of a layer depends on the rank of its weight matrix
    \item \textbf{Redundancy:} Linear dependence in features indicates redundant information
    \item \textbf{Dimensionality Reduction:} Methods like PCA seek low-dimensional representations
    \item \textbf{Network Design:} Understanding which transformations are possible with given architectures
\end{itemize}

% Chapter 2, Section 5: Norms

\section{Norms}
\label{sec:norms}

Norms are functions that measure the size or length of vectors. They are essential for regularization, optimization, and measuring distances in deep learning.

\subsection{Definition of a Norm}

\begin{definition}[Norm]
A function $f: \mathbb{R}^n \rightarrow \mathbb{R}$ is a norm if it satisfies the following properties for all $\vect{x}, \vect{y} \in \mathbb{R}^n$ and $\alpha \in \mathbb{R}$:
\begin{enumerate}
    \item \textbf{Non-negativity:} $f(\vect{x}) \geq 0$, with equality if and only if $\vect{x} = \vect{0}$
    \item \textbf{Homogeneity:} $f(\alpha\vect{x}) = |\alpha|f(\vect{x})$
    \item \textbf{Triangle inequality:} $f(\vect{x} + \vect{y}) \leq f(\vect{x}) + f(\vect{y})$
\end{enumerate}
\end{definition}

We typically denote norms using the notation $\norm{\vect{x}}$.

\subsection{$L^p$ Norms}

The most common family of norms are the $L^p$ norms.

\begin{definition}[$L^p$ Norm]
For $p \geq 1$, the $L^p$ norm of a vector $\vect{x} \in \mathbb{R}^n$ is:
\begin{equation}
    \norm{\vect{x}}_p = \left(\sum_{i=1}^{n} |x_i|^p\right)^{1/p}
\end{equation}
\end{definition}

\subsection{Common Norms}

\subsubsection{$L^1$ Norm (Manhattan Distance)}

The $L^1$ norm is the sum of absolute values:
\begin{equation}
    \norm{\vect{x}}_1 = \sum_{i=1}^{n} |x_i|
\end{equation}

\begin{example}
For $\vect{x} = \begin{bmatrix} 3 \\ -4 \\ 2 \end{bmatrix}$, we have $\norm{\vect{x}}_1 = 3 + 4 + 2 = 9$.
\end{example}

The $L^1$ norm is used in:
\begin{itemize}
    \item Lasso regularization (encourages sparsity)
    \item Robust statistics
    \item Compressed sensing
\end{itemize}

\subsubsection{$L^2$ Norm (Euclidean Distance)}

The $L^2$ norm is the most common norm, corresponding to Euclidean distance:
\begin{equation}
    \norm{\vect{x}}_2 = \sqrt{\sum_{i=1}^{n} x_i^2} = \sqrt{\vect{x}\transpose\vect{x}}
\end{equation}

\begin{example}
For $\vect{x} = \begin{bmatrix} 3 \\ -4 \\ 2 \end{bmatrix}$, we have $\norm{\vect{x}}_2 = \sqrt{9 + 16 + 4} = \sqrt{29} \approx 5.39$.
\end{example}

The $L^2$ norm is used in:
\begin{itemize}
    \item Ridge regularization (weight decay)
    \item Gradient descent
    \item Distance metrics
\end{itemize}

The squared $L^2$ norm is often used in optimization because it has simpler derivatives:
\begin{equation}
    \norm{\vect{x}}_2^2 = \vect{x}\transpose\vect{x} = \sum_{i=1}^{n} x_i^2
\end{equation}

\subsubsection{$L^\infty$ Norm (Maximum Norm)}

The $L^\infty$ norm is defined as:
\begin{equation}
    \norm{\vect{x}}_\infty = \max_i |x_i|
\end{equation}

This can be viewed as the limit of $L^p$ norms as $p \rightarrow \infty$.

\begin{example}
For $\vect{x} = \begin{bmatrix} 3 \\ -4 \\ 2 \end{bmatrix}$, we have $\norm{\vect{x}}_\infty = \max(3, 4, 2) = 4$.
\end{example}

\subsection{Frobenius Norm}

For matrices, the Frobenius norm is analogous to the $L^2$ norm for vectors.

\begin{definition}[Frobenius Norm]
For a matrix $\mat{A} \in \mathbb{R}^{m \times n}$:
\begin{equation}
    \norm{\mat{A}}_F = \sqrt{\sum_{i=1}^{m}\sum_{j=1}^{n} A_{ij}^2} = \sqrt{\text{trace}(\mat{A}\transpose\mat{A})}
\end{equation}
\end{definition}

The Frobenius norm is used for regularizing weight matrices in neural networks.

\subsection{Unit Vectors and Normalization}

A vector with unit norm ($\norm{\vect{x}} = 1$) is called a \emph{unit vector}.

\begin{definition}[Normalization]
To normalize a vector $\vect{x}$, we divide by its norm:
\begin{equation}
    \hat{\vect{x}} = \frac{\vect{x}}{\norm{\vect{x}}}
\end{equation}
resulting in a unit vector pointing in the same direction.
\end{definition}

Normalization is commonly used in deep learning:
\begin{itemize}
    \item Batch normalization
    \item Layer normalization
    \item Input feature scaling
    \item Weight normalization
\end{itemize}

\subsection{Distance Metrics}

Norms induce distance metrics. The distance between vectors $\vect{x}$ and $\vect{y}$ is:
\begin{equation}
    d(\vect{x}, \vect{y}) = \norm{\vect{x} - \vect{y}}
\end{equation}

Different norms lead to different notions of distance:
\begin{itemize}
    \item $L^1$: Manhattan distance (sum of coordinate differences)
    \item $L^2$: Euclidean distance (straight-line distance)
    \item $L^\infty$: Chebyshev distance (maximum coordinate difference)
\end{itemize}

\subsection{Regularization in Deep Learning}

Norms are central to regularization techniques:

\begin{itemize}
    \item \textbf{$L^1$ Regularization:} Adds $\lambda\norm{\vect{w}}_1$ to loss, promoting sparsity
    \item \textbf{$L^2$ Regularization:} Adds $\lambda\norm{\vect{w}}_2^2$ to loss, preventing large weights
    \item \textbf{Elastic Net:} Combines $L^1$ and $L^2$: $\lambda_1\norm{\vect{w}}_1 + \lambda_2\norm{\vect{w}}_2^2$
\end{itemize}

Understanding norms and their properties is essential for designing effective regularization strategies and analyzing model behavior.

% Chapter 2, Section 6: Eigendecomposition

\section{Eigendecomposition}
\label{sec:eigendecomposition}

Eigendecomposition is a powerful tool for understanding and analyzing linear transformations, with important applications in deep learning.

\subsection{Eigenvalues and Eigenvectors}

\begin{definition}[Eigenvector and Eigenvalue]
An \emph{eigenvector} of a square matrix $\mat{A} \in \mathbb{R}^{n \times n}$ is a non-zero vector $\vect{v}$ such that:
\begin{equation}
    \mat{A}\vect{v} = \lambda\vect{v}
\end{equation}
where $\lambda \in \mathbb{R}$ (or $\mathbb{C}$) is the corresponding \emph{eigenvalue}.
\end{definition}

The eigenvector's direction is preserved under the transformation $\mat{A}$, with only its magnitude scaled by $\lambda$.

\begin{example}
Consider $\mat{A} = \begin{bmatrix} 3 & 1 \\ 0 & 2 \end{bmatrix}$. We can verify that $\vect{v}_1 = \begin{bmatrix} 1 \\ 0 \end{bmatrix}$ is an eigenvector:
\begin{equation}
    \mat{A}\vect{v}_1 = \begin{bmatrix} 3 & 1 \\ 0 & 2 \end{bmatrix}\begin{bmatrix} 1 \\ 0 \end{bmatrix} = \begin{bmatrix} 3 \\ 0 \end{bmatrix} = 3\vect{v}_1
\end{equation}
So $\lambda_1 = 3$ is an eigenvalue.
\end{example}

\subsection{Finding Eigenvalues}

To find eigenvalues, we solve the \emph{characteristic equation}:
\begin{equation}
    \det(\mat{A} - \lambda\mat{I}) = 0
\end{equation}

This gives a polynomial of degree $n$ called the characteristic polynomial, which has $n$ roots (counting multiplicities) in $\mathbb{C}$.

\begin{example}
For $\mat{A} = \begin{bmatrix} 2 & 1 \\ 1 & 2 \end{bmatrix}$:
\begin{equation}
    \det\begin{bmatrix} 2-\lambda & 1 \\ 1 & 2-\lambda \end{bmatrix} = (2-\lambda)^2 - 1 = \lambda^2 - 4\lambda + 3 = 0
\end{equation}
Solving gives $\lambda_1 = 3$ and $\lambda_2 = 1$.
\end{example}

\subsection{Eigendecomposition}

If a matrix $\mat{A} \in \mathbb{R}^{n \times n}$ has $n$ linearly independent eigenvectors, it can be decomposed as:

\begin{equation}
    \mat{A} = \mat{V}\boldsymbol{\Lambda}\mat{V}^{-1}
\end{equation}

where:
\begin{itemize}
    \item $\mat{V}$ is the matrix whose columns are eigenvectors: $\mat{V} = [\vect{v}_1, \vect{v}_2, \ldots, \vect{v}_n]$
    \item $\boldsymbol{\Lambda}$ is a diagonal matrix of eigenvalues: $\boldsymbol{\Lambda} = \text{diag}(\lambda_1, \lambda_2, \ldots, \lambda_n)$
\end{itemize}

This is called the \emph{eigendecomposition} or \emph{spectral decomposition}.

\subsection{Symmetric Matrices}

Symmetric matrices have particularly nice properties.

\begin{theorem}[Spectral Theorem for Symmetric Matrices]
If $\mat{A}$ is a real symmetric matrix ($\mat{A} = \mat{A}\transpose$), then:
\begin{enumerate}
    \item All eigenvalues are real
    \item Eigenvectors corresponding to different eigenvalues are orthogonal
    \item $\mat{A}$ can be decomposed as:
    \begin{equation}
        \mat{A} = \mat{Q}\boldsymbol{\Lambda}\mat{Q}\transpose
    \end{equation}
    where $\mat{Q}$ is an orthogonal matrix ($\mat{Q}\transpose\mat{Q} = \mat{I}$) of eigenvectors.
\end{enumerate}
\end{theorem}

This decomposition is fundamental in many algorithms, including Principal Component Analysis (PCA).

\subsection{Properties of Eigenvalues}

For a matrix $\mat{A} \in \mathbb{R}^{n \times n}$:

\begin{itemize}
    \item $\text{trace}(\mat{A}) = \sum_{i=1}^{n} A_{ii} = \sum_{i=1}^{n} \lambda_i$
    \item $\det(\mat{A}) = \prod_{i=1}^{n} \lambda_i$
    \item If $\mat{A}$ is invertible, eigenvalues of $\mat{A}^{-1}$ are $1/\lambda_i$
    \item Eigenvalues of $\mat{A}^k$ are $\lambda_i^k$
\end{itemize}

\subsection{Positive Definite Matrices}

\begin{definition}[Positive Definite]
A symmetric matrix $\mat{A}$ is \emph{positive definite} if for all non-zero $\vect{x} \in \mathbb{R}^n$:
\begin{equation}
    \vect{x}\transpose\mat{A}\vect{x} > 0
\end{equation}
Equivalently, all eigenvalues of $\mat{A}$ are positive.
\end{definition}

\begin{definition}[Positive Semi-definite]
$\mat{A}$ is \emph{positive semi-definite} if $\vect{x}\transpose\mat{A}\vect{x} \geq 0$ for all $\vect{x}$, i.e., all eigenvalues are non-negative.
\end{definition}

Positive definite matrices are crucial in optimization, as they ensure that local minima are global minima for quadratic functions.

\subsection{Applications in Deep Learning}

Eigendecomposition has several important applications:

\begin{enumerate}
    \item \textbf{Principal Component Analysis (PCA):} Finds directions of maximum variance by computing eigenvectors of the covariance matrix.
    
    \item \textbf{Optimization:} The Hessian matrix's eigenvalues determine the curvature of the loss surface. Positive definite Hessians indicate convexity.
    
    \item \textbf{Spectral Normalization:} Constrains the largest eigenvalue of weight matrices to stabilize training of GANs.
    
    \item \textbf{Graph Neural Networks:} Graph Laplacian eigendecomposition defines spectral graph convolutions.
    
    \item \textbf{Understanding Dynamics:} Eigenvalues of recurrent weight matrices affect gradient flow and stability.
\end{enumerate}

\subsection{Computational Considerations}

Computing eigendecomposition:
\begin{itemize}
    \item Full eigendecomposition: $O(n^3)$ for dense matrices
    \item Power iteration for dominant eigenvector: $O(kn^2)$ for $k$ iterations
    \item Iterative methods (e.g., Lanczos) for sparse matrices
\end{itemize}

For large-scale deep learning, we often use:
\begin{itemize}
    \item Approximations (e.g., power iteration)
    \item Focus on top-$k$ eigenvalues/eigenvectors
    \item Specialized algorithms for specific structures (e.g., symmetric, sparse)
\end{itemize}

Understanding eigendecomposition provides insight into the geometric properties of linear transformations and is essential for many advanced deep learning techniques.



% Chapter 3: Probability and Information Theory
% Chapter 3: Probability and Information Theory

\chapter{Probability and Information Theory}
\label{chap:probability}

This chapter introduces fundamental concepts from probability theory and information theory that are essential for understanding machine learning and deep learning. Topics include probability distributions, conditional probability, expectation, variance, entropy, and mutual information.

% Chapter 3, Section 1: Probability Distributions

\section{Probability Distributions}
\label{sec:probability-distributions}

Probability theory provides a mathematical framework for quantifying uncertainty. In deep learning, we use probability distributions to model uncertainty in data, model parameters, and predictions.

\subsection{Discrete Probability Distributions}

A discrete random variable $X$ takes values from a countable set. The \textbf{probability mass function} (PMF) $P(X=x)$ assigns probabilities to each possible value:

\begin{equation}
P(X=x) \geq 0 \quad \text{for all } x
\end{equation}

\begin{equation}
\sum_{x} P(X=x) = 1
\end{equation}

\subsection{Continuous Probability Distributions}

A continuous random variable can take any value in a continuous range. We describe it using a \textbf{probability density function} (PDF) $p(x)$:

\begin{equation}
p(x) \geq 0 \quad \text{for all } x
\end{equation}

\begin{equation}
\int_{-\infty}^{\infty} p(x) \, dx = 1
\end{equation}

The probability that $X$ falls in an interval $[a, b]$ is:

\begin{equation}
P(a \leq X \leq b) = \int_a^b p(x) \, dx
\end{equation}

\subsection{Joint and Marginal Distributions}

For multiple random variables $X$ and $Y$, the \textbf{joint distribution} $P(X, Y)$ describes their combined behavior. The \textbf{marginal distribution} is obtained by summing (or integrating) over the other variable:

\begin{equation}
P(X=x) = \sum_{y} P(X=x, Y=y)
\end{equation}

For continuous variables:

\begin{equation}
p(x) = \int p(x, y) \, dy
\end{equation}

% Chapter 3, Section 2: Conditional Probability and Bayes' Rule

\section{Conditional Probability and Bayes' Rule}
\label{sec:conditional-probability}

\subsection{Conditional Probability}

The \textbf{conditional probability} of $X$ given $Y$ is:

\begin{equation}
P(X|Y) = \frac{P(X, Y)}{P(Y)}
\end{equation}

This quantifies how the probability of $X$ changes when we know the value of $Y$.

\subsection{Independence}

Two random variables $X$ and $Y$ are \textbf{independent} if:

\begin{equation}
P(X, Y) = P(X)P(Y)
\end{equation}

Equivalently, $P(X|Y) = P(X)$ and $P(Y|X) = P(Y)$.

\subsection{Bayes' Theorem}

\textbf{Bayes' theorem} is fundamental to probabilistic inference:

\begin{equation}
P(X|Y) = \frac{P(Y|X)P(X)}{P(Y)}
\end{equation}

In machine learning terminology:
\begin{itemize}
    \item $P(X)$ is the \textbf{prior} probability
    \item $P(Y|X)$ is the \textbf{likelihood}
    \item $P(X|Y)$ is the \textbf{posterior} probability
    \item $P(Y)$ is the \textbf{evidence} or marginal likelihood
\end{itemize}

\subsection{Application to Machine Learning}

Bayes' theorem forms the basis of:
\begin{itemize}
    \item Bayesian inference
    \item Naive Bayes classifiers
    \item Maximum a posteriori (MAP) estimation
    \item Bayesian neural networks
\end{itemize}

Given data $\mathcal{D}$ and model parameters $\theta$:

\begin{equation}
P(\theta|\mathcal{D}) = \frac{P(\mathcal{D}|\theta)P(\theta)}{P(\mathcal{D})}
\end{equation}

% Chapter 3, Section 3: Expectation, Variance, and Covariance

\section{Expectation, Variance, and Covariance}
\label{sec:expectation-variance}

\subsection{Expectation}

The \textbf{expected value} or \textbf{mean} of a function $f(x)$ with respect to distribution $P(x)$ is:

For discrete variables:
\begin{equation}
\mathbb{E}_{x \sim P}[f(x)] = \sum_{x} P(x) f(x)
\end{equation}

For continuous variables:
\begin{equation}
\mathbb{E}_{x \sim p}[f(x)] = \int p(x) f(x) \, dx
\end{equation}

\subsection{Variance}

The \textbf{variance} measures the spread of a distribution:

\begin{equation}
\text{Var}(X) = \mathbb{E}[(X - \mathbb{E}[X])^2] = \mathbb{E}[X^2] - (\mathbb{E}[X])^2
\end{equation}

The \textbf{standard deviation} is $\sigma = \sqrt{\text{Var}(X)}$.

\subsection{Covariance}

The \textbf{covariance} measures how two variables vary together:

\begin{equation}
\text{Cov}(X, Y) = \mathbb{E}[(X - \mathbb{E}[X])(Y - \mathbb{E}[Y])]
\end{equation}

Positive covariance indicates that $X$ and $Y$ tend to increase together, while negative covariance indicates they tend to vary in opposite directions.

\subsection{Correlation}

The \textbf{correlation coefficient} normalizes covariance:

\begin{equation}
\rho(X, Y) = \frac{\text{Cov}(X, Y)}{\sqrt{\text{Var}(X)\text{Var}(Y)}}
\end{equation}

Properties:
\begin{itemize}
    \item $-1 \leq \rho \leq 1$
    \item $|\rho| = 1$ indicates perfect linear relationship
    \item $\rho = 0$ indicates no linear relationship (but variables may still be dependent)
\end{itemize}

% Chapter 3, Section 4: Common Probability Distributions

\section{Common Probability Distributions}
\label{sec:common-distributions}

\subsection{Bernoulli Distribution}

Models a binary random variable (0 or 1):

\begin{equation}
P(X=1) = \phi, \quad P(X=0) = 1-\phi
\end{equation}

Used for binary classification problems.

\subsection{Categorical Distribution}

Generalizes Bernoulli to $k$ discrete outcomes. If $X$ can take values $\{1, 2, \ldots, k\}$:

\begin{equation}
P(X=i) = p_i \quad \text{where} \quad \sum_{i=1}^{k} p_i = 1
\end{equation}

\subsection{Gaussian (Normal) Distribution}

The most important continuous distribution in deep learning:

\begin{equation}
\mathcal{N}(x; \mu, \sigma^2) = \frac{1}{\sqrt{2\pi\sigma^2}} \exp\left(-\frac{(x-\mu)^2}{2\sigma^2}\right)
\end{equation}

Properties:
\begin{itemize}
    \item Mean: $\mu$
    \item Variance: $\sigma^2$
    \item Central limit theorem: sums of independent variables approach Gaussian
\end{itemize}

The multivariate Gaussian with mean vector $\boldsymbol{\mu}$ and covariance matrix $\boldsymbol{\Sigma}$ is:

\begin{equation}
\mathcal{N}(\vect{x}; \boldsymbol{\mu}, \boldsymbol{\Sigma}) = \frac{1}{\sqrt{(2\pi)^n |\boldsymbol{\Sigma}|}} \exp\left(-\frac{1}{2}(\vect{x}-\boldsymbol{\mu})^\top \boldsymbol{\Sigma}^{-1} (\vect{x}-\boldsymbol{\mu})\right)
\end{equation}

\subsection{Exponential Distribution}

Models the time between events in a Poisson process:

\begin{equation}
p(x; \lambda) = \lambda e^{-\lambda x} \quad \text{for } x \geq 0
\end{equation}

\subsection{Laplace Distribution}

Heavy-tailed alternative to Gaussian:

\begin{equation}
\text{Laplace}(x; \mu, b) = \frac{1}{2b} \exp\left(-\frac{|x-\mu|}{b}\right)
\end{equation}

Used in robust statistics and L1 regularization.

\subsection{Dirac Delta and Mixture Distributions}

The \textbf{Dirac delta} $\delta(x)$ concentrates all probability at a single point:

\begin{equation}
p(x) = \delta(x - \mu)
\end{equation}

\textbf{Mixture distributions} combine multiple distributions:

\begin{equation}
p(x) = \sum_{i=1}^{k} \alpha_i p_i(x), \quad \sum_{i=1}^{k} \alpha_i = 1
\end{equation}

Example: Gaussian Mixture Model (GMM).

% Chapter 3, Section 5: Information Theory Basics

\section{Information Theory Basics}
\label{sec:information-theory}

Information theory provides tools for quantifying information and uncertainty, which are crucial for understanding learning and compression.

\subsection{Self-Information}

The \textbf{self-information} or \textbf{surprisal} of an event $x$ is:

\begin{equation}
I(x) = -\log P(x)
\end{equation}

Rare events have high information content, while certain events have zero information.

\subsection{Entropy}

The \textbf{Shannon entropy} measures the expected information in a distribution:

\begin{equation}
H(X) = \mathbb{E}_{x \sim P}[I(x)] = -\sum_{x} P(x) \log P(x)
\end{equation}

For continuous distributions, we use \textbf{differential entropy}:

\begin{equation}
H(X) = -\int p(x) \log p(x) \, dx
\end{equation}

Entropy is maximized when all outcomes are equally likely.

\subsection{Cross-Entropy}

The \textbf{cross-entropy} between distributions $P$ and $Q$ is:

\begin{equation}
H(P, Q) = -\mathbb{E}_{x \sim P}[\log Q(x)] = -\sum_{x} P(x) \log Q(x)
\end{equation}

In deep learning, cross-entropy is commonly used as a loss function for classification.

\subsection{Kullback-Leibler Divergence}

The \textbf{KL divergence} measures how one distribution differs from another:

\begin{equation}
D_{KL}(P \| Q) = \mathbb{E}_{x \sim P}\left[\log \frac{P(x)}{Q(x)}\right] = \sum_{x} P(x) \log \frac{P(x)}{Q(x)}
\end{equation}

Properties:
\begin{itemize}
    \item $D_{KL}(P \| Q) \geq 0$ with equality if and only if $P = Q$
    \item Not symmetric: $D_{KL}(P \| Q) \neq D_{KL}(Q \| P)$
    \item Related to cross-entropy: $D_{KL}(P \| Q) = H(P, Q) - H(P)$
\end{itemize}

\subsection{Mutual Information}

The \textbf{mutual information} between $X$ and $Y$ quantifies how much knowing one reduces uncertainty about the other:

\begin{equation}
I(X; Y) = D_{KL}(P(X, Y) \| P(X)P(Y))
\end{equation}

Equivalently:

\begin{equation}
I(X; Y) = H(X) - H(X|Y) = H(Y) - H(Y|X)
\end{equation}

Mutual information is symmetric and measures the dependence between variables.

\subsection{Applications in Deep Learning}

Information theory concepts are used in:
\begin{itemize}
    \item Loss functions (cross-entropy loss)
    \item Model selection (AIC, BIC use information-theoretic principles)
    \item Variational inference (minimizing KL divergence)
    \item Information bottleneck theory
    \item Mutual information maximization in self-supervised learning
\end{itemize}



% Chapter 4: Numerical Computation
% Chapter 4: Numerical Computation

\chapter{Numerical Computation}
\label{chap:numerical-computation}

This chapter covers numerical methods and computational considerations essential for implementing deep learning algorithms. Topics include gradient-based optimization, numerical stability, and conditioning.

% Chapter 4, Section 1: Overflow and Underflow

\section{Overflow and Underflow}
\label{sec:overflow-underflow}

Computers represent real numbers with finite precision, typically using floating-point arithmetic. This leads to \textbf{rounding errors} that can accumulate and cause problems.

\subsection{Floating-Point Representation}

The IEEE 754 standard defines floating-point numbers. For 32-bit floats:
\begin{itemize}
    \item Smallest positive number: approximately $10^{-38}$
    \item Largest number: approximately $10^{38}$
    \item Machine epsilon: approximately $10^{-7}$
\end{itemize}

\subsection{Underflow}

\textbf{Underflow} occurs when numbers near zero are rounded to zero. This can be problematic when we need to compute ratios or logarithms. For example, the softmax function:

\begin{equation}
\text{softmax}(\vect{x})_i = \frac{\exp(x_i)}{\sum_j \exp(x_j)}
\end{equation}

can underflow if all $x_i$ are very negative.

\subsection{Overflow}

\textbf{Overflow} occurs when large numbers exceed representable values. In the softmax example, overflow can occur if some $x_i$ are very large.

\subsection{Numerical Stability}

To stabilize softmax, we use the identity:

\begin{equation}
\text{softmax}(\vect{x}) = \text{softmax}(\vect{x} - c)
\end{equation}

where $c = \max_i x_i$. This prevents both overflow and underflow.

Similarly, when computing $\log(\sum_i \exp(x_i))$, we use the \textbf{log-sum-exp} trick:

\begin{equation}
\log\left(\sum_i \exp(x_i)\right) = c + \log\left(\sum_i \exp(x_i - c)\right)
\end{equation}

\subsection{Other Numerical Issues}

\textbf{Catastrophic cancellation:} Loss of precision when subtracting nearly equal numbers.

\textbf{Accumulated rounding errors:} Small errors compound through many operations.

\textbf{Solutions:}
\begin{itemize}
    \item Use higher precision (64-bit floats)
    \item Algorithmic modifications (like log-sum-exp)
    \item Batch normalization
    \item Gradient clipping
\end{itemize}

% Chapter 4, Section 2: Gradient-Based Optimization

\section{Gradient-Based Optimization}
\label{sec:gradient-optimization}

Most deep learning algorithms involve optimization: finding parameters that minimize or maximize an objective function.

\subsection{Gradient Descent}

For a function $f(\vect{\theta})$, \textbf{gradient descent} updates parameters as:

\begin{equation}
\vect{\theta}_{t+1} = \vect{\theta}_t - \alpha \nabla_{\vect{\theta}} f(\vect{\theta}_t)
\end{equation}

where $\alpha > 0$ is the \textbf{learning rate}.

\subsection{Jacobian and Hessian Matrices}

The \textbf{Jacobian matrix} contains all first-order partial derivatives. For $\vect{f}: \mathbb{R}^n \to \mathbb{R}^m$:

\begin{equation}
\mat{J}_{ij} = \frac{\partial f_i}{\partial x_j}
\end{equation}

The \textbf{Hessian matrix} contains second-order derivatives:

\begin{equation}
\mat{H}_{ij} = \frac{\partial^2 f}{\partial x_i \partial x_j}
\end{equation}

The Hessian characterizes the local curvature of the function.

\subsection{Taylor Series Approximation}

Near point $\vect{x}_0$, we can approximate $f(\vect{x})$ using Taylor series:

\begin{equation}
f(\vect{x}) \approx f(\vect{x}_0) + (\vect{x} - \vect{x}_0)^\top \nabla f(\vect{x}_0) + \frac{1}{2}(\vect{x} - \vect{x}_0)^\top \mat{H}(\vect{x}_0) (\vect{x} - \vect{x}_0)
\end{equation}

This provides insight into optimization behavior.

\subsection{Critical Points}

At a \textbf{critical point}, $\nabla f(\vect{x}) = \boldsymbol{0}$. The Hessian determines the nature:
\begin{itemize}
    \item \textbf{Local minimum:} Hessian is positive definite
    \item \textbf{Local maximum:} Hessian is negative definite
    \item \textbf{Saddle point:} Hessian has both positive and negative eigenvalues
\end{itemize}

Deep learning often encounters saddle points rather than local minima in high dimensions.

\subsection{Directional Derivatives}

The directional derivative in direction $\vect{u}$ (with $\|\vect{u}\| = 1$) is:

\begin{equation}
\frac{\partial}{\partial \alpha} f(\vect{x} + \alpha \vect{u}) \bigg|_{\alpha=0} = \vect{u}^\top \nabla f(\vect{x})
\end{equation}

To minimize $f$, we move in the direction $\vect{u} = -\frac{\nabla f(\vect{x})}{\|\nabla f(\vect{x})\|}$.

% Chapter 4, Section 3: Constrained Optimization

\section{Constrained Optimization}
\label{sec:constrained-optimization}

Many problems require optimizing a function subject to constraints.

\subsection{Lagrange Multipliers}

For equality constraint $g(\vect{x}) = 0$, the \textbf{Lagrangian} is:

\begin{equation}
\mathcal{L}(\vect{x}, \lambda) = f(\vect{x}) + \lambda g(\vect{x})
\end{equation}

At the optimum, both:
\begin{equation}
\nabla_{\vect{x}} \mathcal{L} = \boldsymbol{0} \quad \text{and} \quad \frac{\partial \mathcal{L}}{\partial \lambda} = 0
\end{equation}

\subsection{Inequality Constraints}

For inequality constraint $g(\vect{x}) \leq 0$, we use the \textbf{Karush-Kuhn-Tucker (KKT)} conditions:

\begin{align}
\nabla_{\vect{x}} \mathcal{L} &= \boldsymbol{0} \\
\lambda &\geq 0 \\
\lambda g(\vect{x}) &= 0 \quad \text{(complementary slackness)} \\
g(\vect{x}) &\leq 0
\end{align}

\subsection{Projected Gradient Descent}

For constraints defining a set $\mathcal{C}$, \textbf{projected gradient descent} applies:

\begin{equation}
\vect{x}_{t+1} = \text{Proj}_{\mathcal{C}}\left(\vect{x}_t - \alpha \nabla f(\vect{x}_t)\right)
\end{equation}

where $\text{Proj}_{\mathcal{C}}$ projects onto the feasible set.

\subsection{Applications in Deep Learning}

Constrained optimization appears in:
\begin{itemize}
    \item Weight constraints (e.g., unit norm constraints)
    \item Projection to valid probability distributions
    \item Adversarial training with bounded perturbations
    \item Fairness constraints
\end{itemize}

% Chapter 4, Section 4: Numerical Stability and Conditioning

\section{Numerical Stability and Conditioning}
\label{sec:numerical-stability}

\subsection{Condition Number}

The \textbf{condition number} of matrix $\mat{A}$ is:

\begin{equation}
\kappa(\mat{A}) = \|\mat{A}\| \|\mat{A}^{-1}\|
\end{equation}

For symmetric matrices with eigenvalues $\lambda_i$:

\begin{equation}
\kappa(\mat{A}) = \frac{\max_i |\lambda_i|}{\min_i |\lambda_i|}
\end{equation}

High condition numbers indicate numerical instability: small changes in input lead to large changes in output.

\subsection{Ill-Conditioned Matrices}

In deep learning, ill-conditioned Hessians can make optimization difficult. This motivates techniques like:
\begin{itemize}
    \item Batch normalization
    \item Careful weight initialization
    \item Adaptive learning rate methods
    \item Preconditioning
\end{itemize}

\subsection{Gradient Checking}

To verify gradient computations, we use \textbf{finite differences}:

\begin{equation}
\frac{\partial f}{\partial \theta_i} \approx \frac{f(\theta_i + \epsilon) - f(\theta_i - \epsilon)}{2\epsilon}
\end{equation}

This is computationally expensive but useful for debugging.

\subsection{Numerical Precision Trade-offs}

\textbf{Mixed precision training:}
\begin{itemize}
    \item Store weights in FP32
    \item Compute activations/gradients in FP16
    \item Use loss scaling to prevent underflow
    \item 2-3x speedup with minimal accuracy loss
\end{itemize}

\subsection{Practical Tips}

\begin{itemize}
    \item Monitor gradient norms during training
    \item Use gradient clipping for RNNs
    \item Prefer numerically stable implementations (log-space computations)
    \item Be aware of precision limits in very deep networks
\end{itemize}

\label{sec:overflow-underflow}

Computers represent real numbers with finite precision, typically using floating-point arithmetic. This leads to \textbf{rounding errors} that can accumulate and cause problems.

\subsection{Floating-Point Representation}

The IEEE 754 standard defines floating-point numbers. For 32-bit floats:
\begin{itemize}
    \item Smallest positive number: approximately $10^{-38}$
    \item Largest number: approximately $10^{38}$
    \item Machine epsilon: approximately $10^{-7}$
\end{itemize}

\subsection{Underflow}

\textbf{Underflow} occurs when numbers near zero are rounded to zero. This can be problematic when we need to compute ratios or logarithms. For example, the softmax function:

\begin{equation}
\text{softmax}(\vect{x})_i = \frac{\exp(x_i)}{\sum_j \exp(x_j)}
\end{equation}

can underflow if all $x_i$ are very negative.

\subsection{Overflow}

\textbf{Overflow} occurs when large numbers exceed representable values. In the softmax example, overflow can occur if some $x_i$ are very large.

\subsection{Numerical Stability}

To stabilize softmax, we use the identity:

\begin{equation}
\text{softmax}(\vect{x}) = \text{softmax}(\vect{x} - c)
\end{equation}

where $c = \max_i x_i$. This prevents both overflow and underflow.

Similarly, when computing $\log(\sum_i \exp(x_i))$, we use the \textbf{log-sum-exp} trick:

\begin{equation}
\log\left(\sum_i \exp(x_i)\right) = c + \log\left(\sum_i \exp(x_i - c)\right)
\end{equation}

\section{Gradient-Based Optimization}
\label{sec:gradient-optimization}

Most deep learning algorithms involve optimization: finding parameters that minimize or maximize an objective function.

\subsection{Gradient Descent}

For a function $f(\vect{\theta})$, \textbf{gradient descent} updates parameters as:

\begin{equation}
\vect{\theta}_{t+1} = \vect{\theta}_t - \alpha \nabla_{\vect{\theta}} f(\vect{\theta}_t)
\end{equation}

where $\alpha > 0$ is the \textbf{learning rate}.

\subsection{Jacobian and Hessian Matrices}

The \textbf{Jacobian matrix} contains all first-order partial derivatives. For $\vect{f}: \mathbb{R}^n \to \mathbb{R}^m$:

\begin{equation}
\mat{J}_{ij} = \frac{\partial f_i}{\partial x_j}
\end{equation}

The \textbf{Hessian matrix} contains second-order derivatives:

\begin{equation}
\mat{H}_{ij} = \frac{\partial^2 f}{\partial x_i \partial x_j}
\end{equation}

The Hessian characterizes the local curvature of the function.

\subsection{Taylor Series Approximation}

Near point $\vect{x}_0$, we can approximate $f(\vect{x})$ using Taylor series:

\begin{equation}
f(\vect{x}) \approx f(\vect{x}_0) + (\vect{x} - \vect{x}_0)^\top \nabla f(\vect{x}_0) + \frac{1}{2}(\vect{x} - \vect{x}_0)^\top \mat{H}(\vect{x}_0) (\vect{x} - \vect{x}_0)
\end{equation}

This provides insight into optimization behavior.

\subsection{Critical Points}

At a \textbf{critical point}, $\nabla f(\vect{x}) = \boldsymbol{0}$. The Hessian determines the nature:
\begin{itemize}
    \item \textbf{Local minimum:} Hessian is positive definite
    \item \textbf{Local maximum:} Hessian is negative definite
    \item \textbf{Saddle point:} Hessian has both positive and negative eigenvalues
\end{itemize}

Deep learning often encounters saddle points rather than local minima in high dimensions.

\subsection{Directional Derivatives}

The directional derivative in direction $\vect{u}$ (with $\|\vect{u}\| = 1$) is:

\begin{equation}
\frac{\partial}{\partial \alpha} f(\vect{x} + \alpha \vect{u}) \bigg|_{\alpha=0} = \vect{u}^\top \nabla f(\vect{x})
\end{equation}

To minimize $f$, we move in the direction $\vect{u} = -\frac{\nabla f(\vect{x})}{\|\nabla f(\vect{x})\|}$.

\section{Constrained Optimization}
\label{sec:constrained-optimization}

Many problems require optimizing a function subject to constraints.

\subsection{Lagrange Multipliers}

For equality constraint $g(\vect{x}) = 0$, the \textbf{Lagrangian} is:

\begin{equation}
\mathcal{L}(\vect{x}, \lambda) = f(\vect{x}) + \lambda g(\vect{x})
\end{equation}

At the optimum, both:
\begin{equation}
\nabla_{\vect{x}} \mathcal{L} = \boldsymbol{0} \quad \text{and} \quad \frac{\partial \mathcal{L}}{\partial \lambda} = 0
\end{equation}

\subsection{Inequality Constraints}

For inequality constraint $g(\vect{x}) \leq 0$, we use the \textbf{Karush-Kuhn-Tucker (KKT)} conditions:

\begin{align}
\nabla_{\vect{x}} \mathcal{L} &= \boldsymbol{0} \\
\lambda &\geq 0 \\
\lambda g(\vect{x}) &= 0 \quad \text{(complementary slackness)} \\
g(\vect{x}) &\leq 0
\end{align}

\subsection{Projected Gradient Descent}

For constraints defining a set $\mathcal{C}$, \textbf{projected gradient descent} applies:

\begin{equation}
\vect{x}_{t+1} = \text{Proj}_{\mathcal{C}}\left(\vect{x}_t - \alpha \nabla f(\vect{x}_t)\right)
\end{equation}

where $\text{Proj}_{\mathcal{C}}$ projects onto the feasible set.

\section{Numerical Stability and Conditioning}
\label{sec:numerical-stability}

\subsection{Condition Number}

The \textbf{condition number} of matrix $\mat{A}$ is:

\begin{equation}
\kappa(\mat{A}) = \|\mat{A}\| \|\mat{A}^{-1}\|
\end{equation}

For symmetric matrices with eigenvalues $\lambda_i$:

\begin{equation}
\kappa(\mat{A}) = \frac{\max_i |\lambda_i|}{\min_i |\lambda_i|}
\end{equation}

High condition numbers indicate numerical instability: small changes in input lead to large changes in output.

\subsection{Ill-Conditioned Matrices}

In deep learning, ill-conditioned Hessians can make optimization difficult. This motivates techniques like:
\begin{itemize}
    \item Batch normalization
    \item Careful weight initialization
    \item Adaptive learning rate methods
    \item Preconditioning
\end{itemize}

\subsection{Gradient Checking}

To verify gradient computations, we use \textbf{finite differences}:

\begin{equation}
\frac{\partial f}{\partial \theta_i} \approx \frac{f(\theta_i + \epsilon) - f(\theta_i - \epsilon)}{2\epsilon}
\end{equation}

This is computationally expensive but useful for debugging.


% Chapter 5: Classical Machine Learning Algorithms
% Chapter 5: Classical Machine Learning Algorithms

\chapter{Classical Machine Learning Algorithms}
\label{chap:classical-ml}

This chapter reviews traditional machine learning methods that provide context and motivation for deep learning approaches. Understanding these classical algorithms helps appreciate the advantages and innovations of deep learning.

\section{Linear Regression}
\label{sec:linear-regression}

\textbf{Linear regression} models the relationship between input features and a continuous output.

\subsection{Model Formulation}

For input $\vect{x} \in \mathbb{R}^d$ and output $y \in \mathbb{R}$:

\begin{equation}
\hat{y} = \vect{w}^\top \vect{x} + b
\end{equation}

where $\vect{w}$ are weights and $b$ is the bias.

\subsection{Ordinary Least Squares}

The \textbf{mean squared error} (MSE) loss is:

\begin{equation}
L(\vect{w}, b) = \frac{1}{n} \sum_{i=1}^{n} (y^{(i)} - \hat{y}^{(i)})^2
\end{equation}

The closed-form solution (using matrix form with bias absorbed) is:

\begin{equation}
\vect{w}^* = (\mat{X}^\top \mat{X})^{-1} \mat{X}^\top \vect{y}
\end{equation}

\subsection{Regularized Regression}

\textbf{Ridge regression} (L2 regularization) adds a penalty:

\begin{equation}
L(\vect{w}) = \|\mat{X}\vect{w} - \vect{y}\|^2 + \lambda \|\vect{w}\|^2
\end{equation}

Solution:
\begin{equation}
\vect{w}^* = (\mat{X}^\top \mat{X} + \lambda \mat{I})^{-1} \mat{X}^\top \vect{y}
\end{equation}

\textbf{Lasso regression} (L1 regularization) promotes sparsity:

\begin{equation}
L(\vect{w}) = \|\mat{X}\vect{w} - \vect{y}\|^2 + \lambda \|\vect{w}\|_1
\end{equation}

\subsection{Gradient Descent Solution}

For large datasets, we use iterative optimization:

\begin{equation}
\vect{w}_{t+1} = \vect{w}_t - \alpha \frac{2}{n} \mat{X}^\top (\mat{X}\vect{w}_t - \vect{y})
\end{equation}

\section{Logistic Regression}
\label{sec:logistic-regression}

\textbf{Logistic regression} is used for binary classification.

\subsection{Binary Classification}

The model uses the sigmoid function:

\begin{equation}
\sigma(z) = \frac{1}{1 + e^{-z}}
\end{equation}

Prediction:
\begin{equation}
P(y=1|\vect{x}) = \sigma(\vect{w}^\top \vect{x} + b)
\end{equation}

\subsection{Cross-Entropy Loss}

The loss function (negative log-likelihood) is:

\begin{equation}
L(\vect{w}, b) = -\frac{1}{n} \sum_{i=1}^{n} \left[y^{(i)} \log \hat{y}^{(i)} + (1-y^{(i)}) \log(1-\hat{y}^{(i)})\right]
\end{equation}

\subsection{Multiclass Classification}

For $K$ classes, we use \textbf{softmax regression}:

\begin{equation}
P(y=k|\vect{x}) = \frac{\exp(\vect{w}_k^\top \vect{x})}{\sum_{j=1}^{K} \exp(\vect{w}_j^\top \vect{x})}
\end{equation}

The loss is the categorical cross-entropy:

\begin{equation}
L = -\frac{1}{n} \sum_{i=1}^{n} \sum_{k=1}^{K} y_k^{(i)} \log \hat{y}_k^{(i)}
\end{equation}

where $y_k^{(i)}$ is 1 if example $i$ belongs to class $k$, and 0 otherwise.

\section{Support Vector Machines}
\label{sec:svm}

\textbf{Support Vector Machines} (SVMs) find the maximum-margin hyperplane separating classes.

\subsection{Linear SVM}

For binary classification with labels $y \in \{-1, +1\}$, the decision boundary is:

\begin{equation}
\vect{w}^\top \vect{x} + b = 0
\end{equation}

The \textbf{margin} is $\frac{2}{\|\vect{w}\|}$. Maximizing the margin is equivalent to minimizing $\|\vect{w}\|^2$ subject to:

\begin{equation}
y^{(i)}(\vect{w}^\top \vect{x}^{(i)} + b) \geq 1 \quad \forall i
\end{equation}

\subsection{Soft Margin SVM}

To handle non-separable data, we introduce slack variables $\xi_i$:

\begin{equation}
\min_{\vect{w}, b, \boldsymbol{\xi}} \frac{1}{2}\|\vect{w}\|^2 + C \sum_{i=1}^{n} \xi_i
\end{equation}

subject to:
\begin{equation}
y^{(i)}(\vect{w}^\top \vect{x}^{(i)} + b) \geq 1 - \xi_i, \quad \xi_i \geq 0
\end{equation}

$C$ controls the trade-off between margin size and training errors.

\subsection{Kernel Trick}

For non-linear decision boundaries, we map inputs to a higher-dimensional space using a \textbf{kernel function} $k(\vect{x}, \vect{x}')$:

\textbf{Common kernels:}
\begin{itemize}
    \item \textbf{Linear:} $k(\vect{x}, \vect{x}') = \vect{x}^\top \vect{x}'$
    \item \textbf{Polynomial:} $k(\vect{x}, \vect{x}') = (\vect{x}^\top \vect{x}' + c)^d$
    \item \textbf{RBF (Gaussian):} $k(\vect{x}, \vect{x}') = \exp(-\gamma \|\vect{x} - \vect{x}'\|^2)$
\end{itemize}

The decision function becomes:
\begin{equation}
f(\vect{x}) = \sum_{i=1}^{n} \alpha_i y^{(i)} k(\vect{x}^{(i)}, \vect{x}) + b
\end{equation}

\section{Decision Trees and Ensemble Methods}
\label{sec:decision-trees}

\subsection{Decision Trees}

A \textbf{decision tree} recursively partitions the input space based on feature values.

\textbf{Splitting criteria:}
\begin{itemize}
    \item \textbf{Gini impurity:} $1 - \sum_{k} p_k^2$
    \item \textbf{Entropy:} $-\sum_{k} p_k \log p_k$
    \item \textbf{MSE} (for regression): variance of target values
\end{itemize}

where $p_k$ is the proportion of class $k$ examples in a node.

\subsection{Random Forests}

\textbf{Random forests} combine multiple decision trees trained on bootstrap samples with random feature subsets at each split.

Prediction is made by averaging (regression) or voting (classification):

\begin{equation}
\hat{y} = \frac{1}{B} \sum_{b=1}^{B} f_b(\vect{x})
\end{equation}

where $B$ is the number of trees.

\subsection{Gradient Boosting}

\textbf{Gradient boosting} builds an ensemble sequentially, where each tree corrects errors of the previous ensemble.

For iteration $m$:
\begin{enumerate}
    \item Compute residuals: $r_i^{(m)} = y^{(i)} - \hat{y}^{(m-1)}(\vect{x}^{(i)})$
    \item Fit tree $f_m$ to residuals
    \item Update: $\hat{y}^{(m)} = \hat{y}^{(m-1)} + \nu f_m$
\end{enumerate}

where $\nu$ is the learning rate.

\section{k-Nearest Neighbors}
\label{sec:knn}

\textbf{k-Nearest Neighbors} (k-NN) is a non-parametric instance-based method.

\subsection{Algorithm}

For a query point $\vect{x}$:
\begin{enumerate}
    \item Find the $k$ closest training examples
    \item For classification: return the majority class
    \item For regression: return the average of their values
\end{enumerate}

\subsection{Distance Metrics}

Common distance metrics:
\begin{itemize}
    \item \textbf{Euclidean:} $d(\vect{x}, \vect{x}') = \sqrt{\sum_i (x_i - x_i')^2}$
    \item \textbf{Manhattan:} $d(\vect{x}, \vect{x}') = \sum_i |x_i - x_i'|$
    \item \textbf{Minkowski:} $d(\vect{x}, \vect{x}') = \left(\sum_i |x_i - x_i'|^p\right)^{1/p}$
\end{itemize}

\subsection{Choosing k}

\begin{itemize}
    \item Small $k$: flexible decision boundary but sensitive to noise
    \item Large $k$: smoother decision boundary but may miss local structure
    \item Typically chosen by cross-validation
\end{itemize}

\subsection{Computational Considerations}

k-NN requires:
\begin{itemize}
    \item No training time (lazy learning)
    \item $O(n)$ prediction time for $n$ training examples
    \item Can be accelerated using KD-trees or ball trees
\end{itemize}

\section{Comparison with Deep Learning}
\label{sec:comparison}

Classical methods have limitations that deep learning addresses:

\begin{itemize}
    \item \textbf{Feature engineering:} Classical methods require manual feature design; deep learning learns features automatically
    \item \textbf{Scalability:} Deep learning scales better with data and model size
    \item \textbf{Representation:} Deep networks learn hierarchical representations
    \item \textbf{Flexibility:} Deep learning handles diverse data types (images, text, audio)
\end{itemize}

However, classical methods remain valuable:
\begin{itemize}
    \item Simpler to interpret and debug
    \item Effective for small to medium datasets
    \item Lower computational requirements
    \item Strong theoretical guarantees in some cases
\end{itemize}


% ======================================================================
% PART II: Practical Deep Networks
% ======================================================================
\part{Practical Deep Networks}

% Chapter 6: Deep Feedforward Networks
% Chapter 6: Deep Feedforward Networks

\chapter{Deep Feedforward Networks}
\label{chap:feedforward-networks}

This chapter introduces deep feedforward neural networks, also known as multilayer perceptrons (MLPs). These are the fundamental building blocks of deep learning.

\section{Introduction to Feedforward Networks}

\textit{This section will introduce the basic architecture and forward propagation.}

\section{Activation Functions}

\textit{This section will cover sigmoid, tanh, ReLU, and other activation functions.}

\section{Output Units and Loss Functions}

\textit{This section will discuss output layer design for different tasks.}

\section{Backpropagation}

\textit{This section will explain the backpropagation algorithm for computing gradients.}

\section{Universal Approximation}

\textit{This section will discuss the theoretical capacity of neural networks.}

\vspace{1em}
\noindent\textit{Note: Detailed content for this chapter will be added in future revisions.}


% Chapter 7: Regularization for Deep Learning
% Chapter 7: Regularization for Deep Learning

\chapter{Regularization for Deep Learning}
\label{chap:regularization}

This chapter explores techniques to improve generalization and prevent overfitting in deep neural networks. Regularization helps models perform well on unseen data.

\section{Parameter Norm Penalties}
\label{sec:parameter-penalties}

\textbf{Parameter norm penalties} constrain model capacity by penalizing large weights.

\subsection{L2 Regularization (Weight Decay)}

Add squared L2 norm of weights to the loss:

\begin{equation}
\tilde{L}(\vect{\theta}) = L(\vect{\theta}) + \frac{\lambda}{2} \|\vect{w}\|^2
\end{equation}

Gradient update becomes:
\begin{equation}
\vect{w} \leftarrow (1 - \alpha\lambda)\vect{w} - \alpha \nabla_{\vect{w}} L
\end{equation}

The factor $(1 - \alpha\lambda)$ causes "weight decay."

\subsection{L1 Regularization}

Add L1 norm:
\begin{equation}
\tilde{L}(\vect{\theta}) = L(\vect{\theta}) + \lambda \|\vect{w}\|_1
\end{equation}

L1 regularization:
\begin{itemize}
    \item Promotes sparsity (many weights become exactly zero)
    \item Useful for feature selection
    \item Gradient: $\text{sign}(\vect{w})$
\end{itemize}

\subsection{Elastic Net}

Combines L1 and L2:
\begin{equation}
\tilde{L}(\vect{\theta}) = L(\vect{\theta}) + \lambda_1 \|\vect{w}\|_1 + \lambda_2 \|\vect{w}\|^2
\end{equation}

\section{Dataset Augmentation}
\label{sec:data-augmentation}

\textbf{Data augmentation} artificially increases training set size by applying transformations that preserve labels.

\subsection{Image Augmentation}

Common transformations:
\begin{itemize}
    \item \textbf{Geometric:} rotation, translation, scaling, flipping, cropping
    \item \textbf{Color:} brightness, contrast, saturation adjustments
    \item \textbf{Noise:} Gaussian noise, blur
    \item \textbf{Cutout/Erasing:} randomly mask regions
    \item \textbf{Mixup:} blend pairs of images and labels
\end{itemize}

Example: horizontal flip
\begin{equation}
\vect{x}_{\text{aug}} = \text{flip}(\vect{x}), \quad y_{\text{aug}} = y
\end{equation}

\subsection{Text Augmentation}

For NLP:
\begin{itemize}
    \item Synonym replacement
    \item Random insertion/deletion
    \item Back-translation
    \item Paraphrasing
\end{itemize}

\subsection{Audio Augmentation}

For speech/audio:
\begin{itemize}
    \item Time stretching
    \item Pitch shifting
    \item Adding background noise
    \item SpecAugment (masking frequency/time regions)
\end{itemize}

\section{Early Stopping}
\label{sec:early-stopping}

\textbf{Early stopping} monitors validation performance and stops training when it begins to degrade.

\subsection{Algorithm}

\begin{enumerate}
    \item Train model and evaluate on validation set periodically
    \item Track best validation performance
    \item If no improvement for $p$ epochs (patience), stop
    \item Return model with best validation performance
\end{enumerate}

\subsection{Benefits}

\begin{itemize}
    \item Simple and effective
    \item Automatically determines training duration
    \item Acts as regularization without changing the model
    \item Reduces computational cost
\end{itemize}

\subsection{Considerations}

\begin{itemize}
    \item Requires separate validation set
    \item Choice of patience hyperparameter
    \item Can interact with learning rate schedules
\end{itemize}

\section{Dropout}
\label{sec:dropout}

\textbf{Dropout} randomly deactivates neurons during training, preventing co-adaptation.

\subsection{Training with Dropout}

At each training step, for each layer:
\begin{enumerate}
    \item Sample binary mask $\vect{m}$ with $P(m_i = 1) = p$
    \item Apply mask: $\vect{h} = \vect{m} \odot \vect{h}$
\end{enumerate}

Mathematically:
\begin{equation}
\vect{h}_{\text{dropout}} = \vect{m} \odot f(\mat{W}\vect{x} + \vect{b})
\end{equation}

where $m_i \sim \text{Bernoulli}(p)$.

\subsection{Inference}

At test time, scale outputs by dropout probability:
\begin{equation}
\vect{h}_{\text{test}} = p \cdot f(\mat{W}\vect{x} + \vect{b})
\end{equation}

Or equivalently, scale weights during training by $\frac{1}{p}$ (inverted dropout).

\subsection{Interpretation}

Dropout can be viewed as:
\begin{itemize}
    \item Training an ensemble of $2^n$ subnetworks
    \item Adding noise to hidden activations
    \item Approximate Bayesian inference
\end{itemize}

\subsection{Variants}

\textbf{DropConnect:} drop weights instead of activations

\textbf{Spatial Dropout:} drop entire feature maps in CNNs

\textbf{Variational Dropout:} use same mask across time steps in RNNs

\section{Batch Normalization}
\label{sec:batch-normalization}

\textbf{Batch normalization} normalizes layer inputs across the batch dimension.

\subsection{Algorithm}

For mini-batch $\mathcal{B}$ with activations $\vect{x}$:

\begin{align}
\mu_{\mathcal{B}} &= \frac{1}{|\mathcal{B}|} \sum_{i \in \mathcal{B}} x_i \\
\sigma^2_{\mathcal{B}} &= \frac{1}{|\mathcal{B}|} \sum_{i \in \mathcal{B}} (x_i - \mu_{\mathcal{B}})^2 \\
\hat{x}_i &= \frac{x_i - \mu_{\mathcal{B}}}{\sqrt{\sigma^2_{\mathcal{B}} + \epsilon}} \\
y_i &= \gamma \hat{x}_i + \beta
\end{align}

where $\gamma$ and $\beta$ are learnable parameters.

\subsection{Benefits}

\begin{itemize}
    \item Reduces internal covariate shift
    \item Allows higher learning rates
    \item Reduces sensitivity to initialization
    \item Acts as regularization
    \item Enables deeper networks
\end{itemize}

\subsection{Inference}

At test time, use running averages computed during training:
\begin{equation}
y = \gamma \frac{x - \mu_{\text{running}}}{\sqrt{\sigma^2_{\text{running}} + \epsilon}} + \beta
\end{equation}

\subsection{Variants}

\textbf{Layer Normalization:} normalize across features (useful for RNNs)

\textbf{Group Normalization:} normalize within groups of channels

\textbf{Instance Normalization:} normalize each sample independently (used in style transfer)

\section{Other Regularization Techniques}
\label{sec:other-regularization}

\subsection{Label Smoothing}

Replace hard targets with smoothed distributions:
\begin{equation}
y'_k = (1 - \epsilon) y_k + \frac{\epsilon}{K}
\end{equation}

Prevents overconfident predictions.

\subsection{Gradient Clipping}

Limit gradient magnitude to prevent exploding gradients:

\textbf{Clipping by value:}
\begin{equation}
g \leftarrow \max(\min(g, \theta), -\theta)
\end{equation}

\textbf{Clipping by norm:}
\begin{equation}
g \leftarrow \frac{g}{\max(1, \|g\| / \theta)}
\end{equation}

\subsection{Stochastic Depth}

Randomly skip layers during training (for very deep networks).

\subsection{Mixup}

Train on convex combinations of examples:
\begin{align}
\tilde{\vect{x}} &= \lambda \vect{x}_i + (1-\lambda) \vect{x}_j \\
\tilde{y} &= \lambda y_i + (1-\lambda) y_j
\end{align}

where $\lambda \sim \text{Beta}(\alpha, \alpha)$.

\subsection{Adversarial Training}

Add adversarially perturbed examples to training:
\begin{equation}
\vect{x}_{\text{adv}} = \vect{x} + \epsilon \cdot \text{sign}(\nabla_{\vect{x}} L(\vect{x}, y))
\end{equation}

Improves robustness and generalization.


% Chapter 8: Optimization for Training Deep Models
% Chapter 8: Optimization for Training Deep Models

\chapter{Optimization for Training Deep Models}
\label{chap:optimization}

This chapter covers optimization algorithms and strategies for training deep neural networks effectively. Modern optimizers go beyond basic gradient descent to accelerate convergence and improve performance.

\section{Gradient Descent Variants}
\label{sec:gd-variants}

\subsection{Batch Gradient Descent}

Computes gradient using entire training set:
\begin{equation}
\vect{\theta}_{t+1} = \vect{\theta}_t - \alpha \nabla_{\vect{\theta}} \frac{1}{n} \sum_{i=1}^{n} L(\vect{\theta}, \vect{x}^{(i)}, y^{(i)})
\end{equation}

Characteristics:
\begin{itemize}
    \item Guaranteed convergence to global minimum (convex) or local minimum (non-convex)
    \item Very slow for large datasets
    \item Deterministic updates
\end{itemize}

\subsection{Stochastic Gradient Descent (SGD)}

Uses a single random example per update:
\begin{equation}
\vect{\theta}_{t+1} = \vect{\theta}_t - \alpha \nabla_{\vect{\theta}} L(\vect{\theta}, \vect{x}^{(i)}, y^{(i)})
\end{equation}

Characteristics:
\begin{itemize}
    \item Much faster per iteration
    \item Noisy updates can escape shallow local minima
    \item High variance in updates
    \item May not converge to exact minimum
\end{itemize}

\subsection{Mini-Batch Gradient Descent}

Balances batch and stochastic approaches:
\begin{equation}
\vect{\theta}_{t+1} = \vect{\theta}_t - \alpha \nabla_{\vect{\theta}} \frac{1}{|\mathcal{B}|} \sum_{i \in \mathcal{B}} L(\vect{\theta}, \vect{x}^{(i)}, y^{(i)})
\end{equation}

where $\mathcal{B}$ is a mini-batch (typically 32-256 examples).

Benefits:
\begin{itemize}
    \item Reduces gradient variance
    \item Efficient GPU utilization
    \item Good balance of speed and stability
\end{itemize}

\section{Momentum-Based Methods}
\label{sec:momentum}

\subsection{Momentum}

Accumulates gradients over time:
\begin{align}
\vect{v}_t &= \beta \vect{v}_{t-1} - \alpha \nabla_{\vect{\theta}} L(\vect{\theta}_t) \\
\vect{\theta}_{t+1} &= \vect{\theta}_t + \vect{v}_t
\end{align}

where $\beta \in [0, 1)$ is the momentum coefficient (typically 0.9).

Benefits:
\begin{itemize}
    \item Accelerates convergence in relevant directions
    \item Dampens oscillations
    \item Helps escape local minima and saddle points
\end{itemize}

\subsection{Nesterov Accelerated Gradient (NAG)}

"Look-ahead" version of momentum:
\begin{align}
\vect{v}_t &= \beta \vect{v}_{t-1} - \alpha \nabla_{\vect{\theta}} L(\vect{\theta}_t + \beta \vect{v}_{t-1}) \\
\vect{\theta}_{t+1} &= \vect{\theta}_t + \vect{v}_t
\end{align}

Evaluates gradient at anticipated future position, often providing better updates.

\section{Adaptive Learning Rate Methods}
\label{sec:adaptive-methods}

\subsection{AdaGrad}

Adapts learning rate per parameter based on historical gradients:
\begin{align}
\vect{g}_t &= \nabla_{\vect{\theta}} L(\vect{\theta}_t) \\
\vect{r}_t &= \vect{r}_{t-1} + \vect{g}_t \odot \vect{g}_t \\
\vect{\theta}_{t+1} &= \vect{\theta}_t - \frac{\alpha}{\sqrt{\vect{r}_t + \epsilon}} \odot \vect{g}_t
\end{align}

where $\epsilon$ (e.g., $10^{-8}$) prevents division by zero.

\subsection{RMSProp}

Addresses AdaGrad's aggressive decay using exponential moving average:
\begin{align}
\vect{r}_t &= \rho \vect{r}_{t-1} + (1-\rho) \vect{g}_t \odot \vect{g}_t \\
\vect{\theta}_{t+1} &= \vect{\theta}_t - \frac{\alpha}{\sqrt{\vect{r}_t + \epsilon}} \odot \vect{g}_t
\end{align}

\subsection{Adam (Adaptive Moment Estimation)}

Combines momentum and adaptive learning rates:
\begin{align}
\vect{m}_t &= \beta_1 \vect{m}_{t-1} + (1-\beta_1) \vect{g}_t \\
\vect{v}_t &= \beta_2 \vect{v}_{t-1} + (1-\beta_2) \vect{g}_t \odot \vect{g}_t \\
\hat{\vect{m}}_t &= \frac{\vect{m}_t}{1 - \beta_1^t} \\
\hat{\vect{v}}_t &= \frac{\vect{v}_t}{1 - \beta_2^t} \\
\vect{\theta}_{t+1} &= \vect{\theta}_t - \frac{\alpha \hat{\vect{m}}_t}{\sqrt{\hat{\vect{v}}_t} + \epsilon}
\end{align}

Default hyperparameters: $\beta_1 = 0.9$, $\beta_2 = 0.999$, $\epsilon = 10^{-8}$, $\alpha = 0.001$.

\subsection{Learning Rate Schedules}

\textbf{Step Decay:}
\begin{equation}
\alpha_t = \alpha_0 \cdot \gamma^{\lfloor t / s \rfloor}
\end{equation}

\textbf{Exponential Decay:}
\begin{equation}
\alpha_t = \alpha_0 e^{-\lambda t}
\end{equation}

\textbf{Cosine Annealing:}
\begin{equation}
\alpha_t = \alpha_{\min} + \frac{1}{2}(\alpha_{\max} - \alpha_{\min})\left(1 + \cos\left(\frac{t}{T}\pi\right)\right)
\end{equation}

\section{Second-Order Methods}
\label{sec:second-order}

\subsection{Newton's Method}

Uses second-order Taylor expansion:
\begin{equation}
\vect{\theta}_{t+1} = \vect{\theta}_t - \mat{H}^{-1} \nabla_{\vect{\theta}} L(\vect{\theta}_t)
\end{equation}

where $\mat{H}$ is the Hessian matrix.

Challenges:
\begin{itemize}
    \item Computing Hessian is $O(n^2)$ in parameters
    \item Inverting Hessian is $O(n^3)$
    \item Infeasible for large neural networks
\end{itemize}

\subsection{Quasi-Newton Methods}

Approximate the Hessian inverse:

\textbf{L-BFGS:} maintains low-rank approximation of Hessian inverse
\begin{itemize}
    \item More efficient than full Newton's method
    \item Still expensive for very large models
    \item Used for smaller networks or specific applications
\end{itemize}

\subsection{Natural Gradient}

Uses Fisher information matrix instead of Hessian:
\begin{equation}
\vect{\theta}_{t+1} = \vect{\theta}_t - \alpha \mat{F}^{-1} \nabla_{\vect{\theta}} L(\vect{\theta}_t)
\end{equation}

Provides parameter updates invariant to reparameterization.

\section{Optimization Challenges}
\label{sec:challenges}

\subsection{Vanishing and Exploding Gradients}

In deep networks, gradients can become exponentially small or large.

\textbf{Vanishing gradients:}
\begin{itemize}
    \item Common with sigmoid/tanh activations
    \item Mitigated by ReLU, batch normalization, residual connections
\end{itemize}

\textbf{Exploding gradients:}
\begin{itemize}
    \item Common in RNNs
    \item Mitigated by gradient clipping, careful initialization
\end{itemize}

\textbf{Gradient clipping:}
\begin{equation}
\vect{g} \leftarrow \frac{\vect{g}}{\max(1, \|\vect{g}\| / \theta)}
\end{equation}

\subsection{Local Minima and Saddle Points}

In high dimensions, saddle points are more common than local minima.

Saddle points have:
\begin{itemize}
    \item Zero gradient
    \item Mixed curvature (positive and negative eigenvalues)
\end{itemize}

Momentum and noise help escape saddle points.

\subsection{Plateaus}

Flat regions with small gradients slow convergence. Adaptive methods and learning rate schedules help navigate plateaus.

\subsection{Practical Optimization Strategy}

\textbf{Recommended approach:}
\begin{enumerate}
    \item Start with Adam optimizer
    \item Learning rate: try 0.001, 0.0003, 0.0001
    \item Batch size: use 32-256
    \item Monitor training metrics
    \item Add learning rate schedule if needed
    \item Use gradient clipping if unstable
\end{enumerate}


% Chapter 9: Convolutional Networks
% Chapter 9: Convolutional Networks

\chapter{Convolutional Networks}
\label{chap:convolutional-networks}

This chapter introduces convolutional neural networks (CNNs), which are particularly effective for processing grid-structured data like images.

\section{The Convolution Operation}

\textit{This section will explain convolution and its properties.}

\section{Pooling}

\textit{This section will cover max pooling, average pooling, and other pooling operations.}

\section{CNN Architectures}

\textit{This section will discuss classic architectures: LeNet, AlexNet, VGG, ResNet, etc.}

\section{Applications of CNNs}

\textit{This section will explore image classification, object detection, and segmentation.}

\vspace{1em}
\noindent\textit{Note: Detailed content for this chapter will be added in future revisions.}


% Chapter 10: Sequence Modeling: Recurrent and Recursive Nets
% Chapter 10: Sequence Modeling: Recurrent and Recursive Nets

\chapter{Sequence Modeling: Recurrent and Recursive Nets}
\label{chap:sequence-modeling}

This chapter covers architectures designed for sequential and temporal data, including recurrent neural networks (RNNs) and their variants.

\section{Recurrent Neural Networks}
\label{sec:rnns}

\subsection{Motivation}

Sequential data has temporal dependencies:
\begin{itemize}
    \item Time series (stock prices, sensor readings)
    \item Text (words depend on previous words)
    \item Speech (phonemes form words)
    \item Video (frames over time)
\end{itemize}

Standard feedforward networks cannot capture these dependencies effectively.

\subsection{Basic RNN Architecture}

An RNN maintains a hidden state $\vect{h}_t$ that evolves over time:

\begin{align}
\vect{h}_t &= \sigma(\mat{W}_{hh} \vect{h}_{t-1} + \mat{W}_{xh} \vect{x}_t + \vect{b}_h) \\
\vect{y}_t &= \mat{W}_{hy} \vect{h}_t + \vect{b}_y
\end{align}

where $\vect{x}_t$ is input at time $t$, and $\sigma$ is typically tanh.

\subsection{Unfolding in Time}

RNNs can be "unrolled" into a feedforward network with shared weights across time steps:

\begin{equation}
\vect{h}_t = f(\vect{h}_{t-1}, \vect{x}_t; \vect{\theta})
\end{equation}

\subsection{Types of Sequences}

\textbf{One-to-many:} Single input, sequence output (e.g., image captioning)

\textbf{Many-to-one:} Sequence input, single output (e.g., sentiment classification)

\textbf{Many-to-many:} Sequence input and output (e.g., machine translation)

\section{Backpropagation Through Time}
\label{sec:bptt}

\subsection{BPTT Algorithm}

Gradients are computed by unrolling the network and applying backpropagation:

\begin{equation}
\frac{\partial L}{\partial \vect{h}_t} = \frac{\partial L}{\partial \vect{y}_t} \frac{\partial \vect{y}_t}{\partial \vect{h}_t} + \frac{\partial L}{\partial \vect{h}_{t+1}} \frac{\partial \vect{h}_{t+1}}{\partial \vect{h}_t}
\end{equation}

For weight matrix $\mat{W}$:
\begin{equation}
\frac{\partial L}{\partial \mat{W}} = \sum_{t=1}^{T} \frac{\partial L_t}{\partial \mat{W}}
\end{equation}

\subsection{Vanishing and Exploding Gradients}

Gradients can vanish or explode exponentially:

\begin{equation}
\frac{\partial \vect{h}_t}{\partial \vect{h}_k} = \prod_{i=k+1}^{t} \frac{\partial \vect{h}_i}{\partial \vect{h}_{i-1}} = \prod_{i=k+1}^{t} \mat{W}^\top \text{diag}(\sigma'(\vect{z}_i))
\end{equation}

If eigenvalues of $\mat{W}$ are:
\begin{itemize}
    \item $< 1$: gradients vanish
    \item $> 1$: gradients explode
\end{itemize}

\textbf{Solutions:}
\begin{itemize}
    \item Gradient clipping (for explosion)
    \item Careful initialization
    \item ReLU activation
    \item LSTM/GRU architectures
\end{itemize}

\subsection{Truncated BPTT}

For very long sequences, truncate gradient computation:
\begin{itemize}
    \item Only backpropagate through $k$ time steps
    \item Reduces memory and computation
    \item Trade-off: cannot capture long-term dependencies
\end{itemize}

\section{Long Short-Term Memory (LSTM)}
\label{sec:lstm}

\subsection{Architecture}

LSTM uses \textbf{gating mechanisms} to control information flow:

\begin{align}
\vect{f}_t &= \sigma(\mat{W}_f [\vect{h}_{t-1}, \vect{x}_t] + \vect{b}_f) \quad \text{(forget gate)} \\
\vect{i}_t &= \sigma(\mat{W}_i [\vect{h}_{t-1}, \vect{x}_t] + \vect{b}_i) \quad \text{(input gate)} \\
\tilde{\vect{c}}_t &= \tanh(\mat{W}_c [\vect{h}_{t-1}, \vect{x}_t] + \vect{b}_c) \quad \text{(candidate)} \\
\vect{c}_t &= \vect{f}_t \odot \vect{c}_{t-1} + \vect{i}_t \odot \tilde{\vect{c}}_t \quad \text{(cell state)} \\
\vect{o}_t &= \sigma(\mat{W}_o [\vect{h}_{t-1}, \vect{x}_t] + \vect{b}_o) \quad \text{(output gate)} \\
\vect{h}_t &= \vect{o}_t \odot \tanh(\vect{c}_t) \quad \text{(hidden state)}
\end{align}

\subsection{Key Ideas}

\textbf{Cell state} $\vect{c}_t$: Long-term memory
\begin{itemize}
    \item Information flows with minimal transformation
    \item Gates control what to remember/forget
\end{itemize}

\textbf{Forget gate} $\vect{f}_t$: Decides what to discard from cell state

\textbf{Input gate} $\vect{i}_t$: Decides what new information to store

\textbf{Output gate} $\vect{o}_t$: Decides what to output

\subsection{Advantages}

\begin{itemize}
    \item Addresses vanishing gradient problem
    \item Can learn long-term dependencies
    \item Gradients flow more easily through cell state
    \item Widely used for sequential tasks
\end{itemize}

\section{Gated Recurrent Units (GRU)}
\label{sec:gru}

\subsection{Architecture}

GRU simplifies LSTM with fewer parameters:

\begin{align}
\vect{z}_t &= \sigma(\mat{W}_z [\vect{h}_{t-1}, \vect{x}_t]) \quad \text{(update gate)} \\
\vect{r}_t &= \sigma(\mat{W}_r [\vect{h}_{t-1}, \vect{x}_t]) \quad \text{(reset gate)} \\
\tilde{\vect{h}}_t &= \tanh(\mat{W} [\vect{r}_t \odot \vect{h}_{t-1}, \vect{x}_t]) \quad \text{(candidate)} \\
\vect{h}_t &= (1 - \vect{z}_t) \odot \vect{h}_{t-1} + \vect{z}_t \odot \tilde{\vect{h}}_t
\end{align}

\subsection{Comparison with LSTM}

\textbf{GRU:}
\begin{itemize}
    \item Fewer parameters (faster to train)
    \item No separate cell state
    \item Often performs similarly to LSTM
\end{itemize}

\textbf{LSTM:}
\begin{itemize}
    \item More expressive (separate forget/input gates)
    \item Better for longer sequences in some cases
\end{itemize}

\section{Sequence-to-Sequence Models}
\label{sec:seq2seq}

\subsection{Encoder-Decoder Architecture}

For tasks like machine translation:

\textbf{Encoder:} Processes input sequence into fixed representation
\begin{equation}
\vect{c} = f(\vect{x}_1, \vect{x}_2, \ldots, \vect{x}_T)
\end{equation}

\textbf{Decoder:} Generates output sequence from representation
\begin{equation}
\vect{y}_t = g(\vect{y}_{t-1}, \vect{c}, \vect{s}_{t-1})
\end{equation}

\subsection{Attention Mechanism}

Standard seq2seq compresses entire input into fixed vector $\vect{c}$, causing information bottleneck.

\textbf{Attention} allows decoder to focus on relevant input parts:

\begin{align}
e_{ti} &= a(\vect{s}_{t-1}, \vect{h}_i) \quad \text{(alignment scores)} \\
\alpha_{ti} &= \frac{\exp(e_{ti})}{\sum_j \exp(e_{tj})} \quad \text{(attention weights)} \\
\vect{c}_t &= \sum_i \alpha_{ti} \vect{h}_i \quad \text{(context vector)}
\end{align}

Benefits:
\begin{itemize}
    \item Dynamic context for each output
    \item Better for long sequences
    \item Interpretable (visualize attention weights)
\end{itemize}

\subsection{Applications}

\begin{itemize}
    \item Machine translation
    \item Text summarization
    \item Question answering
    \item Image captioning (CNN encoder, RNN decoder)
    \item Speech recognition
\end{itemize}

\section{Advanced Topics}
\label{sec:rnn-advanced}

\subsection{Bidirectional RNNs}

Process sequence in both directions:
\begin{align}
\overrightarrow{\vect{h}}_t &= f(\vect{x}_t, \overrightarrow{\vect{h}}_{t-1}) \\
\overleftarrow{\vect{h}}_t &= f(\vect{x}_t, \overleftarrow{\vect{h}}_{t+1}) \\
\vect{h}_t &= [\overrightarrow{\vect{h}}_t; \overleftarrow{\vect{h}}_t]
\end{align}

Useful when future context is available.

\subsection{Deep RNNs}

Stack multiple RNN layers:
\begin{equation}
\vect{h}_t^{(l)} = f(\vect{h}_t^{(l-1)}, \vect{h}_{t-1}^{(l)})
\end{equation}

Each layer captures different levels of abstraction.

\subsection{Teacher Forcing}

During training, use ground truth as decoder input (not model's prediction):
\begin{itemize}
    \item Faster convergence
    \item Stable training
    \item May cause exposure bias at test time
\end{itemize}

\subsection{Beam Search}

For inference, maintain top-$k$ hypotheses:
\begin{itemize}
    \item Better than greedy decoding
    \item Trade-off between quality and speed
    \item Common beam size: 5-10
\end{itemize}


% Chapter 11: Practical Methodology
% Chapter 11: Practical Methodology

\chapter{Practical Methodology}
\label{chap:practical-methodology}

This chapter provides practical guidelines for successfully applying deep learning to real-world problems.


% Chapter 11, Section 1

\section{Performance Metrics}
\label{sec:performance-metrics}

\subsection{Classification Metrics}

\textbf{Accuracy:}
\begin{equation}
\text{Accuracy} = \frac{\text{Correct predictions}}{\text{Total predictions}}
\end{equation}

\textbf{Precision and Recall:}
\begin{align}
\text{Precision} &= \frac{\text{TP}}{\text{TP} + \text{FP}} \\
\text{Recall} &= \frac{\text{TP}}{\text{TP} + \text{FN}}
\end{align}

\textbf{F1 Score:} Harmonic mean of precision and recall
\begin{equation}
F_1 = 2 \cdot \frac{\text{Precision} \cdot \text{Recall}}{\text{Precision} + \text{Recall}}
\end{equation}

\textbf{ROC Curve and AUC:} Trade-off between true positive rate and false positive rate

\textbf{Confusion Matrix:} Visualizes prediction performance across classes

\subsection{Regression Metrics}

\textbf{Mean Squared Error (MSE):}
\begin{equation}
\text{MSE} = \frac{1}{n} \sum_{i=1}^{n} (y_i - \hat{y}_i)^2
\end{equation}

\textbf{Mean Absolute Error (MAE):}
\begin{equation}
\text{MAE} = \frac{1}{n} \sum_{i=1}^{n} |y_i - \hat{y}_i|
\end{equation}

\textbf{R-squared ($R^2$):} Proportion of variance explained

\subsection{NLP Metrics}

\textbf{BLEU:} For machine translation (measures n-gram overlap)

\textbf{ROUGE:} For summarization

\textbf{Perplexity:} For language models
\begin{equation}
\text{PPL} = \exp\left(-\frac{1}{N} \sum_{i=1}^{N} \log P(x_i)\right)
\end{equation}


% Chapter 11, Section 2

\section{Baseline Models and Debugging}
\label{sec:baselines-debugging}

\subsection{Establishing Baselines}

Start with simple baselines:
\begin{enumerate}
    \item \textbf{Random baseline:} Random predictions
    \item \textbf{Simple heuristics:} Rule-based systems
    \item \textbf{Classical ML:} Logistic regression, random forests
    \item \textbf{Simple neural networks:} Small architectures
\end{enumerate}

Compare deep learning improvements against these baselines.

\subsection{Debugging Strategy}

\textbf{Step 1: Overfit a small dataset}
\begin{itemize}
    \item Take 10-100 examples
    \item Turn off regularization
    \item If can't overfit, model has bugs
\end{itemize}

\textbf{Step 2: Check intermediate outputs}
\begin{itemize}
    \item Visualize activations
    \item Check gradient magnitudes
    \item Verify loss decreases on training set
\end{itemize}

\textbf{Step 3: Diagnose underfitting vs. overfitting}
\begin{itemize}
    \item \textbf{Underfitting:} Poor train performance $\to$ increase capacity
    \item \textbf{Overfitting:} Good train, poor validation $\to$ add regularization
\end{itemize}

\subsection{Common Issues}

\textbf{Vanishing/exploding gradients:}
\begin{itemize}
    \item Use batch normalization
    \item Gradient clipping
    \item Better initialization
\end{itemize}

\textbf{Dead ReLUs:}
\begin{itemize}
    \item Lower learning rate
    \item Use Leaky ReLU or ELU
\end{itemize}

\textbf{Loss not decreasing:}
\begin{itemize}
    \item Check learning rate (too high or too low)
    \item Verify gradient computation
    \item Check data preprocessing
\end{itemize}


% Chapter 11, Section 3

\section{Hyperparameter Tuning}
\label{sec:hyperparameter-tuning}

\subsection{Key Hyperparameters (Priority Order)}

\begin{enumerate}
    \item \textbf{Learning rate:} Most critical
    \item \textbf{Network architecture:} Layers, neurons
    \item \textbf{Batch size:} Affects training dynamics
    \item \textbf{Regularization:} Dropout, weight decay
    \item \textbf{Optimizer parameters:} Momentum, beta values
\end{enumerate}

\subsection{Search Strategies}

\textbf{Manual Search:}
\begin{itemize}
    \item Start with educated guesses
    \item Adjust based on results
    \item Time-consuming but insightful
\end{itemize}

\textbf{Grid Search:}
\begin{itemize}
    \item Try all combinations from predefined values
    \item Exhaustive but expensive
    \item Better for 2-3 hyperparameters
\end{itemize}

\textbf{Random Search:}
\begin{itemize}
    \item Sample hyperparameters randomly
    \item More efficient than grid search
    \item Better for high-dimensional spaces
\end{itemize}

\textbf{Bayesian Optimization:}
\begin{itemize}
    \item Model hyperparameter performance
    \item Choose next trials intelligently
    \item More sample-efficient
\end{itemize}

\subsection{Best Practices}

\begin{itemize}
    \item Use logarithmic scale for learning rate
    \item Try learning rates: 0.1, 0.01, 0.001, 0.0001
    \item Start with standard architectures
    \item Use validation set for selection
    \item Retrain with best hyperparameters on full train set
\end{itemize}


% Chapter 11, Section 4

\section{Data Preparation and Preprocessing}
\label{sec:data-preparation}

\subsection{Data Splitting}

\textbf{Train/Validation/Test split:}
\begin{itemize}
    \item Training: 60-80\%
    \item Validation: 10-20\%
    \item Test: 10-20\%
\end{itemize}

\textbf{Cross-validation:} For small datasets
\begin{itemize}
    \item k-fold cross-validation
    \item Stratified splits for imbalanced data
\end{itemize}

\subsection{Normalization}

\textbf{Min-Max Scaling:}
\begin{equation}
x' = \frac{x - x_{\min}}{x_{\max} - x_{\min}}
\end{equation}

\textbf{Standardization (Z-score):}
\begin{equation}
x' = \frac{x - \mu}{\sigma}
\end{equation}

Always compute statistics on training set only!

\subsection{Handling Imbalanced Data}

\begin{itemize}
    \item \textbf{Oversampling:} Duplicate minority class examples
    \item \textbf{Undersampling:} Remove majority class examples
    \item \textbf{SMOTE:} Synthetic minority oversampling
    \item \textbf{Class weights:} Penalize errors on minority class more
    \item \textbf{Focal loss:} Focus on hard examples
\end{itemize}

\subsection{Data Augmentation}

Generate additional training examples through transformations (see Chapter 7).


% Chapter 11, Section 5

\section{Production Considerations}
\label{sec:production}

\subsection{Model Deployment}

\begin{itemize}
    \item Model compression (pruning, quantization)
    \item Model serving infrastructure
    \item Latency requirements
    \item Batch vs. online inference
\end{itemize}

\subsection{Monitoring}

Track in production:
\begin{itemize}
    \item Prediction distribution shifts
    \item Model performance metrics
    \item System latency and throughput
    \item Error analysis
\end{itemize}

\subsection{Iterative Improvement}

\begin{enumerate}
    \item Deploy initial model
    \item Monitor performance
    \item Collect more data
    \item Retrain and improve
    \item A/B test new models
\end{enumerate}



% Chapter 12: Applications
% Chapter 12: Applications

\chapter{Applications}
\label{chap:applications}

This chapter showcases deep learning applications across various domains, demonstrating the breadth and impact of the field.


% Chapter 12, Section 1

\section{Computer Vision Applications}
\label{sec:cv-applications}

\subsection{Image Classification}

Assign labels to images:
\begin{itemize}
    \item \textbf{ImageNet:} 1000-class object recognition
    \item \textbf{Fine-grained classification:} Bird species, car models
    \item \textbf{Medical imaging:} Disease classification from X-rays, CT scans
\end{itemize}

\textbf{Architecture:} CNN backbone (ResNet, EfficientNet) + classification head

\subsection{Object Detection}

Locate and classify objects in images:
\begin{itemize}
    \item \textbf{Autonomous driving:} Pedestrians, vehicles, traffic signs
    \item \textbf{Surveillance:} Person detection and tracking
    \item \textbf{Retail:} Product recognition
\end{itemize}

\textbf{Methods:} YOLO, Faster R-CNN, RetinaNet

\subsection{Semantic Segmentation}

Classify every pixel:
\begin{itemize}
    \item \textbf{Autonomous driving:} Road, sidewalk, vehicle segmentation
    \item \textbf{Medical imaging:} Tumor segmentation, organ delineation
    \item \textbf{Satellite imagery:} Land use classification
\end{itemize}

\textbf{Architectures:} U-Net, DeepLab, Mask R-CNN

\subsection{Face Recognition}

Identify or verify individuals:
\begin{itemize}
    \item Security and access control
    \item Photo organization
    \item Payment authentication
\end{itemize}

\textbf{Approach:} Face detection + embedding (FaceNet, ArcFace) + similarity matching

\subsection{Image Generation and Manipulation}

\begin{itemize}
    \item \textbf{Style transfer:} Apply artistic styles
    \item \textbf{Super-resolution:} Enhance image quality
    \item \textbf{Inpainting:} Fill missing regions
    \item \textbf{Deepfakes:} Face swapping (ethical concerns)
\end{itemize}


% Chapter 12, Section 2

\section{Natural Language Processing}
\label{sec:nlp-applications}

\subsection{Text Classification}

Categorize text documents:
\begin{itemize}
    \item \textbf{Sentiment analysis:} Positive/negative reviews
    \item \textbf{Spam detection:} Email filtering
    \item \textbf{Topic classification:} News categorization
\end{itemize}

\textbf{Models:} BERT, RoBERTa, DistilBERT

\subsection{Machine Translation}

Translate between languages:
\begin{itemize}
    \item Google Translate, DeepL
    \item Sequence-to-sequence with attention
    \item Transformer models
\end{itemize}

\textbf{Architecture:} Encoder-decoder transformers

\subsection{Question Answering}

Answer questions based on context:
\begin{itemize}
    \item \textbf{Extractive QA:} Find answer span in text (SQuAD)
    \item \textbf{Open-domain QA:} Answer from large corpora
    \item \textbf{Visual QA:} Answer questions about images
\end{itemize}

\subsection{Language Models and Text Generation}

Generate coherent text:
\begin{itemize}
    \item GPT models for general text generation
    \item Code generation (GitHub Copilot)
    \item Chatbots and conversational AI
    \item Content creation
\end{itemize}

\subsection{Named Entity Recognition}

Extract entities from text:
\begin{itemize}
    \item Person, organization, location names
    \item Dates, quantities, technical terms
    \item Applications in information extraction
\end{itemize}


% Chapter 12, Section 3

\section{Speech Recognition and Synthesis}
\label{sec:speech-applications}

\subsection{Automatic Speech Recognition (ASR)}

Convert speech to text:
\begin{itemize}
    \item Virtual assistants (Siri, Alexa, Google Assistant)
    \item Transcription services
    \item Voice commands
\end{itemize}

\textbf{Architectures:} Deep Speech, Listen Attend and Spell, Wav2Vec

\subsection{Text-to-Speech (TTS)}

Generate natural-sounding speech:
\begin{itemize}
    \item WaveNet, Tacotron
    \item Voice cloning
    \item Accessibility tools
\end{itemize}

\subsection{Speaker Recognition}

Identify speakers:
\begin{itemize}
    \item Voice biometrics
    \item Speaker diarization (who spoke when)
\end{itemize}


% Chapter 12, Section 4

\section{Healthcare and Medical Imaging}
\label{sec:healthcare-applications}

\subsection{Medical Image Analysis}

\textbf{Disease detection:}
\begin{itemize}
    \item Cancer detection in mammograms, CT scans
    \item Diabetic retinopathy from retinal images
    \item Pneumonia detection from chest X-rays
\end{itemize}

\textbf{Segmentation:}
\begin{itemize}
    \item Tumor boundary delineation
    \item Organ segmentation for surgical planning
\end{itemize}

\subsection{Drug Discovery}

\begin{itemize}
    \item Predicting molecular properties
    \item Protein structure prediction (AlphaFold)
    \item Drug-target interaction prediction
\end{itemize}

\subsection{Clinical Decision Support}

\begin{itemize}
    \item Diagnosis assistance
    \item Treatment recommendation
    \item Risk prediction (readmission, mortality)
\end{itemize}

\subsection{Genomics}

\begin{itemize}
    \item DNA sequence analysis
    \item Variant calling
    \item Gene expression prediction
\end{itemize}


% Chapter 12, Section 5

\section{Reinforcement Learning Applications}
\label{sec:rl-applications}

\subsection{Game Playing}

Superhuman performance:
\begin{itemize}
    \item \textbf{AlphaGo:} Defeated world champion in Go
    \item \textbf{AlphaZero:} Mastered chess, shogi, and Go
    \item \textbf{OpenAI Five:} Dota 2
    \item \textbf{AlphaStar:} StarCraft II
\end{itemize}

\subsection{Robotics}

\begin{itemize}
    \item \textbf{Manipulation:} Grasping, assembly
    \item \textbf{Navigation:} Autonomous movement
    \item \textbf{Locomotion:} Walking, running
\end{itemize}

\subsection{Autonomous Vehicles}

\begin{itemize}
    \item Path planning and decision making
    \item Combined with perception (CV)
    \item Safety-critical systems
\end{itemize}

\subsection{Recommendation Systems}

\begin{itemize}
    \item Netflix, YouTube content recommendations
    \item E-commerce product suggestions
    \item Personalized news feeds
\end{itemize}

\subsection{Resource Management}

\begin{itemize}
    \item Data center cooling optimization (DeepMind)
    \item Traffic light control
    \item Energy grid optimization
\end{itemize}


% Chapter 12, Section 6

\section{Other Applications}
\label{sec:other-applications}

\subsection{Finance}

\begin{itemize}
    \item Algorithmic trading
    \item Fraud detection
    \item Credit risk assessment
    \item Market prediction
\end{itemize}

\subsection{Scientific Research}

\begin{itemize}
    \item \textbf{Physics:} Particle classification, gravitational wave detection
    \item \textbf{Climate science:} Weather prediction, climate modeling
    \item \textbf{Astronomy:} Galaxy classification, exoplanet detection
\end{itemize}

\subsection{Agriculture}

\begin{itemize}
    \item Crop disease detection
    \item Yield prediction
    \item Precision agriculture
\end{itemize}

\subsection{Manufacturing}

\begin{itemize}
    \item Quality control and defect detection
    \item Predictive maintenance
    \item Supply chain optimization
\end{itemize}


% Chapter 12, Section 7

\section{Ethical Considerations}
\label{sec:ethics}

Deep learning applications raise important concerns:

\begin{itemize}
    \item \textbf{Bias and fairness:} Models may perpetuate societal biases
    \item \textbf{Privacy:} Data collection and usage concerns
    \item \textbf{Transparency:} "Black box" nature of deep models
    \item \textbf{Security:} Adversarial attacks and model manipulation
    \item \textbf{Job displacement:} Automation impact on employment
    \item \textbf{Environmental impact:} Energy consumption of large models
\end{itemize}

Responsible AI development requires addressing these challenges.



% ======================================================================
% PART III: Deep Learning Research
% ======================================================================
\part{Deep Learning Research}

% Chapter 13: Linear Factor Models
% Chapter 13: Linear Factor Models

\chapter{Linear Factor Models}
\label{chap:linear-factor-models}

This chapter introduces probabilistic models with linear structure, which form the foundation for many unsupervised learning methods.


% Chapter 13, Section 1

\section{Probabilistic PCA}
\label{sec:prob-pca}

\subsection{Principal Component Analysis Review}

PCA finds orthogonal directions of maximum variance:
\begin{equation}
\vect{z} = \mat{W}^\top (\vect{x} - \boldsymbol{\mu})
\end{equation}

where $\mat{W}$ contains principal components (eigenvectors of covariance matrix).

\subsection{Probabilistic Formulation}

Model observations as:
\begin{align}
\vect{z} &\sim \mathcal{N}(\boldsymbol{0}, \mat{I}) \\
\vect{x} | \vect{z} &\sim \mathcal{N}(\mat{W}\vect{z} + \boldsymbol{\mu}, \sigma^2 \mat{I})
\end{align}

Marginalizing over $\vect{z}$:
\begin{equation}
\vect{x} \sim \mathcal{N}(\boldsymbol{\mu}, \mat{W}\mat{W}^\top + \sigma^2 \mat{I})
\end{equation}

\subsection{Learning}

Maximize likelihood using EM algorithm:
\begin{itemize}
    \item \textbf{E-step:} Compute $p(\vect{z}|\vect{x})$
    \item \textbf{M-step:} Update $\mat{W}$, $\boldsymbol{\mu}$, $\sigma^2$
\end{itemize}

As $\sigma^2 \to 0$, recovers standard PCA.


% Chapter 13, Section 2

\section{Factor Analysis}
\label{sec:factor-analysis}

Similar to probabilistic PCA but with diagonal noise covariance:
\begin{equation}
\vect{x} | \vect{z} \sim \mathcal{N}(\mat{W}\vect{z} + \boldsymbol{\mu}, \boldsymbol{\Psi})
\end{equation}

where $\boldsymbol{\Psi}$ is diagonal. Each observed dimension has its own noise variance.

\textbf{Applications:} Psychology, social sciences, finance


% Chapter 13, Section 3

\section{Independent Component Analysis}
\label{sec:ica}

\subsection{Objective}

Find independent sources from linear mixtures:
\begin{equation}
\vect{x} = \mat{A}\vect{s}
\end{equation}

where $\vect{s}$ contains independent sources.

\subsection{Non-Gaussianity}

ICA exploits that independent signals are typically non-Gaussian.

\textbf{Applications:}
\begin{itemize}
    \item Blind source separation (cocktail party problem)
    \item Signal processing
    \item Feature extraction
\end{itemize}


% Chapter 13, Section 4

\section{Sparse Coding}
\label{sec:sparse-coding}

Learn overcomplete dictionary where data has sparse representation:
\begin{equation}
\min_{\mat{D}, \vect{z}} \|\vect{x} - \mat{D}\vect{z}\|^2 + \lambda \|\vect{z}\|_1
\end{equation}

\textbf{Applications:}
\begin{itemize}
    \item Image denoising
    \item Feature learning
    \item Compression
\end{itemize}



% Chapter 14: Autoencoders
% Chapter 14: Autoencoders

\chapter{Autoencoders}
\label{chap:autoencoders}

This chapter explores autoencoders, neural networks designed for unsupervised learning through data reconstruction.

\section{Undercomplete Autoencoders}
\label{sec:undercomplete-ae}

\subsection{Architecture}

An autoencoder consists of:
\begin{itemize}
    \item \textbf{Encoder:} $\vect{h} = f(\vect{x})$ maps input to latent representation
    \item \textbf{Decoder:} $\hat{\vect{x}} = g(\vect{h})$ reconstructs from latent code
\end{itemize}

\subsection{Training Objective}

Minimize reconstruction error:
\begin{equation}
L = \|\vect{x} - g(f(\vect{x}))\|^2
\end{equation}

or more generally:
\begin{equation}
L = -\log p(\vect{x} | g(f(\vect{x})))
\end{equation}

\subsection{Undercomplete Constraint}

If $\dim(\vect{h}) < \dim(\vect{x})$, the autoencoder learns compressed representation.

Acts as dimensionality reduction (similar to PCA but non-linear).

\section{Regularized Autoencoders}
\label{sec:regularized-ae}

\subsection{Sparse Autoencoders}

Add sparsity penalty on hidden activations:
\begin{equation}
L = \|\vect{x} - \hat{\vect{x}}\|^2 + \lambda \sum_j |h_j|
\end{equation}

Encourages learning of sparse, interpretable features.

\subsection{Denoising Autoencoders (DAE)}

Train to reconstruct clean input from corrupted version:
\begin{enumerate}
    \item Corrupt input: $\tilde{\vect{x}} \sim q(\tilde{\vect{x}}|\vect{x})$
    \item Encode corrupted input: $\vect{h} = f(\tilde{\vect{x}})$
    \item Decode and reconstruct: $\hat{\vect{x}} = g(\vect{h})$
    \item Minimize: $L = \|\vect{x} - \hat{\vect{x}}\|^2$
\end{enumerate}

\textbf{Corruption types:}
\begin{itemize}
    \item Additive Gaussian noise
    \item Masking (randomly set inputs to zero)
    \item Salt-and-pepper noise
\end{itemize}

Learns robust representations.

\subsection{Contractive Autoencoders (CAE)}

Add penalty on Jacobian of encoder:
\begin{equation}
L = \|\vect{x} - \hat{\vect{x}}\|^2 + \lambda \left\|\frac{\partial f(\vect{x})}{\partial \vect{x}}\right\|_F^2
\end{equation}

Encourages locally contractive mappings (robust to small perturbations).

\section{Variational Autoencoders}
\label{sec:vae}

\subsection{Probabilistic Framework}

VAE is a generative model:
\begin{align}
p(\vect{x}) &= \int p(\vect{x}|\vect{z}) p(\vect{z}) d\vect{z} \\
p(\vect{z}) &= \mathcal{N}(\boldsymbol{0}, \mat{I}) \\
p(\vect{x}|\vect{z}) &= \mathcal{N}(\vect{x}; \boldsymbol{\mu}_{\theta}(\vect{z}), \boldsymbol{\sigma}^2_{\theta}(\vect{z})\mat{I})
\end{align}

\subsection{Evidence Lower Bound (ELBO)}

Cannot directly maximize $\log p(\vect{x})$. Instead maximize ELBO:
\begin{equation}
\mathcal{L} = \mathbb{E}_{q(\vect{z}|\vect{x})}[\log p(\vect{x}|\vect{z})] - D_{KL}(q(\vect{z}|\vect{x}) \| p(\vect{z}))
\end{equation}

where $q(\vect{z}|\vect{x}) = \mathcal{N}(\vect{z}; \boldsymbol{\mu}_{\phi}(\vect{x}), \boldsymbol{\sigma}^2_{\phi}(\vect{x})\mat{I})$ is the encoder.

\subsection{Reparameterization Trick}

To backpropagate through sampling:
\begin{equation}
\vect{z} = \boldsymbol{\mu}_{\phi}(\vect{x}) + \boldsymbol{\sigma}_{\phi}(\vect{x}) \odot \boldsymbol{\epsilon}, \quad \boldsymbol{\epsilon} \sim \mathcal{N}(\boldsymbol{0}, \mat{I})
\end{equation}

Enables end-to-end gradient-based training.

\subsection{Generation}

Sample from prior $\vect{z} \sim \mathcal{N}(\boldsymbol{0}, \mat{I})$ and decode to generate new data.

\section{Applications of Autoencoders}
\label{sec:ae-applications}

\subsection{Dimensionality Reduction}

Learn compact representations for:
\begin{itemize}
    \item Visualization (like t-SNE, UMAP)
    \item Preprocessing for downstream tasks
    \item Feature extraction
\end{itemize}

\subsection{Anomaly Detection}

High reconstruction error indicates anomalies:
\begin{itemize}
    \item Fraud detection
    \item Manufacturing defects
    \item Network intrusion detection
\end{itemize}

\subsection{Denoising}

DAEs remove noise from:
\begin{itemize}
    \item Images
    \item Audio signals
    \item Sensor data
\end{itemize}

\subsection{Data Generation}

VAEs generate new samples:
\begin{itemize}
    \item Image synthesis
    \item Data augmentation
    \item Molecular design
\end{itemize}

\subsection{Representation Learning}

Pre-train encoders for:
\begin{itemize}
    \item Transfer learning
    \item Semi-supervised learning
    \item Metric learning
\end{itemize}


% Chapter 15: Representation Learning
% Chapter 15: Representation Learning

\chapter{Representation Learning}
\label{chap:representation-learning}

This chapter discusses the central challenge of deep learning: learning meaningful representations from data.

\section{What Makes a Good Representation?}

\textit{This section will explore properties of effective representations.}

\section{Transfer Learning and Domain Adaptation}

\textit{This section will cover using learned representations across tasks and domains.}

\section{Self-Supervised Learning}

\textit{This section will introduce techniques for learning without explicit labels.}

\section{Contrastive Learning}

\textit{This section will discuss modern approaches to representation learning.}

\vspace{1em}
\noindent\textit{Note: Detailed content for this chapter will be added in future revisions.}


% Chapter 16: Structured Probabilistic Models for Deep Learning
% Chapter 16: Structured Probabilistic Models for Deep Learning

\chapter{Structured Probabilistic Models for Deep Learning}
\label{chap:structured-probabilistic-models}

This chapter covers graphical models and their integration with deep learning.

\section{Graphical Models}

\textit{This section will introduce Bayesian networks and Markov random fields.}

\section{Inference in Graphical Models}

\textit{This section will discuss exact and approximate inference methods.}

\section{Deep Learning and Structured Models}

\textit{This section will explore the intersection of neural networks and graphical models.}

\vspace{1em}
\noindent\textit{Note: Detailed content for this chapter will be added in future revisions.}


% Chapter 17: Monte Carlo Methods
% Chapter 17: Monte Carlo Methods

\chapter{Monte Carlo Methods}
\label{chap:monte-carlo}

This chapter introduces sampling-based approaches for probabilistic inference and learning.

\section{Sampling and Monte Carlo Estimators}

\textit{This section will introduce basic Monte Carlo methods and variance reduction.}

\section{Markov Chain Monte Carlo}

\textit{This section will cover Metropolis-Hastings, Gibbs sampling, and related methods.}

\section{Importance Sampling}

\textit{This section will discuss importance sampling and its applications.}

\section{Applications in Deep Learning}

\textit{This section will explore how sampling methods are used in modern deep learning.}

\vspace{1em}
\noindent\textit{Note: Detailed content for this chapter will be added in future revisions.}


% Chapter 18: Confronting the Partition Function
% Chapter 18: Confronting the Partition Function

\chapter{Confronting the Partition Function}
\label{chap:partition-function}

This chapter addresses computational challenges in probabilistic models arising from intractable partition functions.

\section{The Partition Function Problem}

\textit{This section will explain why partition functions are challenging.}

\section{Contrastive Divergence}

\textit{This section will introduce approximations for gradient computation.}

\section{Noise-Contrastive Estimation}

\textit{This section will discuss NCE and related techniques.}

\vspace{1em}
\noindent\textit{Note: Detailed content for this chapter will be added in future revisions.}


% Chapter 19: Approximate Inference
% Chapter 19: Approximate Inference

\chapter{Approximate Inference}
\label{chap:approximate-inference}

This chapter explores methods for tractable inference in complex probabilistic models.


% Chapter 19, Section 1

\section{Variational Inference}
\label{sec:variational-inference}

\subsection{Evidence Lower Bound (ELBO)}

For latent variable model with intractable posterior $p(\vect{z}|\vect{x})$, approximate with $q(\vect{z})$:

\begin{align}
\log p(\vect{x}) &= \mathbb{E}_{q(\vect{z})}[\log p(\vect{x})] \\
&= \mathbb{E}_{q(\vect{z})}\left[\log \frac{p(\vect{x}, \vect{z})}{p(\vect{z}|\vect{x})}\right] \\
&= \mathbb{E}_{q(\vect{z})}\left[\log \frac{p(\vect{x}, \vect{z})}{q(\vect{z})}\right] + D_{KL}(q(\vect{z}) \| p(\vect{z}|\vect{x})) \\
&\geq \mathbb{E}_{q(\vect{z})}\left[\log \frac{p(\vect{x}, \vect{z})}{q(\vect{z})}\right] = \mathcal{L}(q)
\end{align}

Maximizing $\mathcal{L}(q)$ minimizes $D_{KL}(q(\vect{z}) \| p(\vect{z}|\vect{x}))$.

\subsection{Variational Family}

Choose tractable family of distributions:

\textbf{Mean field:} Fully factorized
\begin{equation}
q(\vect{z}) = \prod_{i=1}^{n} q_i(z_i)
\end{equation}

\textbf{Structured:} Allow some dependencies
\begin{equation}
q(\vect{z}) = \prod_{c} q_c(\vect{z}_c)
\end{equation}

Trade-off between expressiveness and tractability.

\subsection{Coordinate Ascent VI}

Optimize each factor iteratively:
\begin{equation}
q_j^*(z_j) \propto \exp\left(\mathbb{E}_{q_{-j}}[\log p(\vect{z}, \vect{x})]\right)
\end{equation}

Guaranteed to converge to local optimum of ELBO.

\subsection{Stochastic Variational Inference}

Use stochastic gradients for scalability:
\begin{itemize}
    \item Mini-batch data
    \item Monte Carlo estimation of expectations
    \item Reparameterization trick for low variance
\end{itemize}


% Chapter 19, Section 2

\section{Mean Field Approximation}
\label{sec:mean-field}

\subsection{Fully Factorized Approximation}

Assume all variables independent:
\begin{equation}
q(\vect{z}) = \prod_{i=1}^{n} q_i(z_i)
\end{equation}

\subsection{Update Equations}

For each variable:
\begin{equation}
\log q_j^*(z_j) = \mathbb{E}_{i \neq j}[\log p(\vect{z}, \vect{x})] + \text{const}
\end{equation}

Iterate until convergence.

\subsection{Properties}

\begin{itemize}
    \item Underestimates variance (overconfident)
    \item Computationally efficient
    \item Often good approximation in practice
\end{itemize}


% Chapter 19, Section 3

\section{Loopy Belief Propagation}
\label{sec:loopy-bp}

\subsection{Message Passing}

On graphical models, pass messages between nodes:
\begin{equation}
m_{i \to j}(x_j) = \sum_{x_i} \psi(x_i, x_j) \psi(x_i) \prod_{k \in N(i) \setminus j} m_{k \to i}(x_i)
\end{equation}

\subsection{Beliefs}

Compute marginals from messages:
\begin{equation}
b_i(x_i) \propto \psi(x_i) \prod_{j \in N(i)} m_{j \to i}(x_i)
\end{equation}

\subsection{Exact on Trees}

For tree-structured graphs, converges to exact marginals.

\subsection{Loopy Graphs}

On graphs with cycles:
\begin{itemize}
    \item May not converge
    \item Often gives good approximations
    \item Used in error-correcting codes, computer vision
\end{itemize}


% Chapter 19, Section 4

\section{Expectation Propagation}
\label{sec:ep}

Approximates each factor with simpler distribution:
\begin{equation}
p(\vect{x}) = \frac{1}{Z} \prod_i f_i(\vect{x}) \approx \frac{1}{Z} \prod_i \tilde{f}_i(\vect{x})
\end{equation}

Iteratively refine approximations to match moments.

Better than mean field for multi-modal posteriors.



% Chapter 20: Deep Generative Models
% Chapter 20: Deep Generative Models

\chapter{Deep Generative Models}
\label{chap:deep-generative-models}

This chapter examines modern approaches to generating new data samples using deep learning.


% Chapter 20, Section 1

\section{Variational Autoencoders (VAEs)}
\label{sec:vaes}

(See also Chapter 14 for detailed VAE coverage.)

\subsection{Recap}

VAE learns latent representation $\vect{z}$ and decoder $p_{\theta}(\vect{x}|\vect{z})$:
\begin{equation}
\max_{\theta, \phi} \mathbb{E}_{q_{\phi}(\vect{z}|\vect{x})}[\log p_{\theta}(\vect{x}|\vect{z})] - D_{KL}(q_{\phi}(\vect{z}|\vect{x}) \| p(\vect{z}))
\end{equation}

\subsection{Conditional VAEs}

Generate conditioned on class or attributes:
\begin{equation}
\max \mathbb{E}_{q(\vect{z}|\vect{x}, y)}[\log p(\vect{x}|\vect{z}, y)] - D_{KL}(q(\vect{z}|\vect{x}, y) \| p(\vect{z}))
\end{equation}

\subsection{Disentangled Representations}

\textbf{$\beta$-VAE:} Increase KL weight for disentanglement
\begin{equation}
\mathcal{L} = \mathbb{E}_{q}[\log p(\vect{x}|\vect{z})] - \beta D_{KL}(q(\vect{z}|\vect{x}) \| p(\vect{z}))
\end{equation}


% Chapter 20, Section 2

\section{Generative Adversarial Networks (GANs)}
\label{sec:gans}

\subsection{Core Idea}

Two networks compete:
\begin{itemize}
    \item \textbf{Generator} $G$: Creates fake samples from noise
    \item \textbf{Discriminator} $D$: Distinguishes real from fake
\end{itemize}

\subsection{Objective}

Minimax game:
\begin{equation}
\min_G \max_D \mathbb{E}_{\vect{x} \sim p_{\text{data}}}[\log D(\vect{x})] + \mathbb{E}_{\vect{z} \sim p(\vect{z})}[\log(1 - D(G(\vect{z})))]
\end{equation}

\subsection{Training Procedure}

Alternate updates:
\begin{enumerate}
    \item \textbf{Update D:} Maximize discrimination
    \begin{equation}
    \max_D \mathbb{E}_{\vect{x}}[\log D(\vect{x})] + \mathbb{E}_{\vect{z}}[\log(1 - D(G(\vect{z})))]
    \end{equation}
    
    \item \textbf{Update G:} Minimize discrimination (or maximize $\log D(G(\vect{z}))$)
    \begin{equation}
    \min_G \mathbb{E}_{\vect{z}}[\log(1 - D(G(\vect{z})))]
    \end{equation}
\end{enumerate}

\subsection{Training Challenges}

\textbf{Mode collapse:} Generator produces limited variety

\textbf{Training instability:} Oscillations, non-convergence

\textbf{Vanishing gradients:} When discriminator too strong

\subsection{GAN Variants}

\textbf{DCGAN:} Deep Convolutional GAN with architectural guidelines

\textbf{WGAN:} Wasserstein GAN with improved training stability

\textbf{StyleGAN:} High-quality image generation with style control

\textbf{Conditional GAN:} Generate from class labels

\textbf{CycleGAN:} Unpaired image-to-image translation


% Chapter 20, Section 3

\section{Normalizing Flows}
\label{sec:normalizing-flows}

\subsection{Key Idea}

Transform simple distribution (e.g., Gaussian) through invertible mappings:
\begin{equation}
\vect{x} = f_{\theta}(\vect{z}), \quad \vect{z} \sim p_z(\vect{z})
\end{equation}

\subsection{Change of Variables}

Density transforms as:
\begin{equation}
p_x(\vect{x}) = p_z(f^{-1}(\vect{x})) \left|\det \frac{\partial f^{-1}}{\partial \vect{x}}\right|
\end{equation}

or equivalently:
\begin{equation}
\log p_x(\vect{x}) = \log p_z(\vect{z}) - \log\left|\det \frac{\partial f}{\partial \vect{z}}\right|
\end{equation}

\subsection{Requirements}

Function $f$ must be:
\begin{itemize}
    \item Invertible
    \item Have tractable Jacobian determinant
\end{itemize}

\subsection{Flow Architectures}

\textbf{Coupling layers:} Split dimensions and transform half conditioned on other half

\textbf{Autoregressive flows:} Each dimension depends on previous ones

\textbf{Continuous normalizing flows:} Use neural ODEs

\subsection{Advantages}

\begin{itemize}
    \item Exact likelihood computation
    \item Exact sampling
    \item Stable training (no adversarial dynamics)
\end{itemize}


% Chapter 20, Section 4

\section{Diffusion Models}
\label{sec:diffusion-models}

\subsection{Forward Process}

Gradually add noise over $T$ steps:
\begin{equation}
q(\vect{x}_t|\vect{x}_{t-1}) = \mathcal{N}(\vect{x}_t; \sqrt{1-\beta_t} \vect{x}_{t-1}, \beta_t \mat{I})
\end{equation}

Eventually $\vect{x}_T \approx \mathcal{N}(\boldsymbol{0}, \mat{I})$.

\subsection{Reverse Process}

Learn to denoise (reverse diffusion):
\begin{equation}
p_{\theta}(\vect{x}_{t-1}|\vect{x}_t) = \mathcal{N}(\vect{x}_{t-1}; \boldsymbol{\mu}_{\theta}(\vect{x}_t, t), \boldsymbol{\Sigma}_{\theta}(\vect{x}_t, t))
\end{equation}

\subsection{Training}

Predict noise $\boldsymbol{\epsilon}_{\theta}(\vect{x}_t, t)$ at each step:
\begin{equation}
\mathcal{L} = \mathbb{E}_{t, \vect{x}_0, \boldsymbol{\epsilon}}\left[\|\boldsymbol{\epsilon} - \boldsymbol{\epsilon}_{\theta}(\vect{x}_t, t)\|^2\right]
\end{equation}

\subsection{Sampling}

Start from noise and iteratively denoise:
\begin{equation}
\vect{x}_{t-1} = \frac{1}{\sqrt{\alpha_t}}\left(\vect{x}_t - \frac{1-\alpha_t}{\sqrt{1-\bar{\alpha}_t}}\boldsymbol{\epsilon}_{\theta}(\vect{x}_t, t)\right) + \sigma_t \vect{z}
\end{equation}

\subsection{Advantages}

\begin{itemize}
    \item High-quality generation (DALL-E 2, Stable Diffusion, Midjourney)
    \item Stable training
    \item Strong theoretical foundations
    \item Can condition on text, images, etc.
\end{itemize}


% Chapter 20, Section 5

\section{Applications and Future Directions}
\label{sec:generative-applications}

\subsection{Current Applications}

\textbf{Image generation:}
\begin{itemize}
    \item Text-to-image (DALL-E, Stable Diffusion)
    \item Image editing and inpainting
    \item Super-resolution
    \item Style transfer
\end{itemize}

\textbf{Text generation:}
\begin{itemize}
    \item Large language models (GPT family)
    \item Code generation
    \item Creative writing
\end{itemize}

\textbf{Audio/speech:}
\begin{itemize}
    \item Text-to-speech
    \item Music generation
    \item Voice conversion
\end{itemize}

\textbf{Video:}
\begin{itemize}
    \item Video prediction
    \item Video synthesis
    \item Animation
\end{itemize}

\textbf{Scientific applications:}
\begin{itemize}
    \item Molecule design
    \item Protein structure prediction
    \item Materials discovery
\end{itemize}

\subsection{Future Directions}

\begin{itemize}
    \item \textbf{Controllability:} Fine-grained control over generation
    \item \textbf{Efficiency:} Faster sampling and smaller models
    \item \textbf{Multi-modal:} Unified models across modalities
    \item \textbf{Reasoning:} Incorporating logical reasoning
    \item \textbf{Safety:} Preventing harmful content generation
    \item \textbf{Evaluation:} Better metrics for generation quality
\end{itemize}

\subsection{Societal Impact}

Generative models raise important considerations:
\begin{itemize}
    \item Copyright and intellectual property
    \item Misinformation and deepfakes
    \item Job displacement in creative fields
    \item Environmental cost of large-scale training
    \item Equitable access to technology
\end{itemize}

Responsible development requires addressing these challenges while advancing capabilities.



% --- Back Matter ---
\backmatter

% Glossary
\printglossary[title={Glossary},toctitle={Glossary}]

% Index
\printindex

% Bibliography
\printbibliography[heading=bibintoc,title={Bibliography}]

\end{document}
