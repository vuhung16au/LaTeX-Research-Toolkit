% Chapter 4: Numerical Computation

\chapter{Numerical Computation}
\label{chap:numerical-computation}

This chapter covers numerical methods and computational considerations essential for implementing deep learning algorithms. Topics include gradient-based optimization, numerical stability, and conditioning.

% Chapter 4, Section 1: Overflow and Underflow

\section{Overflow and Underflow}
\label{sec:overflow-underflow}

Computers represent real numbers with finite precision, typically using floating-point arithmetic. This leads to \textbf{rounding errors} that can accumulate and cause problems.

\subsection{Floating-Point Representation}

The IEEE 754 standard defines floating-point numbers. For 32-bit floats:
\begin{itemize}
    \item Smallest positive number: approximately $10^{-38}$
    \item Largest number: approximately $10^{38}$
    \item Machine epsilon: approximately $10^{-7}$
\end{itemize}

\subsection{Underflow}

\textbf{Underflow} occurs when numbers near zero are rounded to zero. This can be problematic when we need to compute ratios or logarithms. For example, the softmax function:

\begin{equation}
\text{softmax}(\vect{x})_i = \frac{\exp(x_i)}{\sum_j \exp(x_j)}
\end{equation}

can underflow if all $x_i$ are very negative.

\subsection{Overflow}

\textbf{Overflow} occurs when large numbers exceed representable values. In the softmax example, overflow can occur if some $x_i$ are very large.

\subsection{Numerical Stability}

To stabilize softmax, we use the identity:

\begin{equation}
\text{softmax}(\vect{x}) = \text{softmax}(\vect{x} - c)
\end{equation}

where $c = \max_i x_i$. This prevents both overflow and underflow.

Similarly, when computing $\log(\sum_i \exp(x_i))$, we use the \textbf{log-sum-exp} trick:

\begin{equation}
\log\left(\sum_i \exp(x_i)\right) = c + \log\left(\sum_i \exp(x_i - c)\right)
\end{equation}

\subsection{Other Numerical Issues}

\textbf{Catastrophic cancellation:} Loss of precision when subtracting nearly equal numbers.

\textbf{Accumulated rounding errors:} Small errors compound through many operations.

\textbf{Solutions:}
\begin{itemize}
    \item Use higher precision (64-bit floats)
    \item Algorithmic modifications (like log-sum-exp)
    \item Batch normalization
    \item Gradient clipping
\end{itemize}

% Chapter 4, Section 2: Gradient-Based Optimization

\section{Gradient-Based Optimization}
\label{sec:gradient-optimization}

Most deep learning algorithms involve optimization: finding parameters that minimize or maximize an objective function.

\subsection{Gradient Descent}

For a function $f(\vect{\theta})$, \textbf{gradient descent} updates parameters as:

\begin{equation}
\vect{\theta}_{t+1} = \vect{\theta}_t - \alpha \nabla_{\vect{\theta}} f(\vect{\theta}_t)
\end{equation}

where $\alpha > 0$ is the \textbf{learning rate}.

\subsection{Jacobian and Hessian Matrices}

The \textbf{Jacobian matrix} contains all first-order partial derivatives. For $\vect{f}: \mathbb{R}^n \to \mathbb{R}^m$:

\begin{equation}
\mat{J}_{ij} = \frac{\partial f_i}{\partial x_j}
\end{equation}

The \textbf{Hessian matrix} contains second-order derivatives:

\begin{equation}
\mat{H}_{ij} = \frac{\partial^2 f}{\partial x_i \partial x_j}
\end{equation}

The Hessian characterizes the local curvature of the function.

\subsection{Taylor Series Approximation}

Near point $\vect{x}_0$, we can approximate $f(\vect{x})$ using Taylor series:

\begin{equation}
f(\vect{x}) \approx f(\vect{x}_0) + (\vect{x} - \vect{x}_0)^\top \nabla f(\vect{x}_0) + \frac{1}{2}(\vect{x} - \vect{x}_0)^\top \mat{H}(\vect{x}_0) (\vect{x} - \vect{x}_0)
\end{equation}

This provides insight into optimization behavior.

\subsection{Critical Points}

At a \textbf{critical point}, $\nabla f(\vect{x}) = \boldsymbol{0}$. The Hessian determines the nature:
\begin{itemize}
    \item \textbf{Local minimum:} Hessian is positive definite
    \item \textbf{Local maximum:} Hessian is negative definite
    \item \textbf{Saddle point:} Hessian has both positive and negative eigenvalues
\end{itemize}

Deep learning often encounters saddle points rather than local minima in high dimensions.

\subsection{Directional Derivatives}

The directional derivative in direction $\vect{u}$ (with $\|\vect{u}\| = 1$) is:

\begin{equation}
\frac{\partial}{\partial \alpha} f(\vect{x} + \alpha \vect{u}) \bigg|_{\alpha=0} = \vect{u}^\top \nabla f(\vect{x})
\end{equation}

To minimize $f$, we move in the direction $\vect{u} = -\frac{\nabla f(\vect{x})}{\|\nabla f(\vect{x})\|}$.

% Chapter 4, Section 3: Constrained Optimization

\section{Constrained Optimization}
\label{sec:constrained-optimization}

Many problems require optimizing a function subject to constraints.

\subsection{Lagrange Multipliers}

For equality constraint $g(\vect{x}) = 0$, the \textbf{Lagrangian} is:

\begin{equation}
\mathcal{L}(\vect{x}, \lambda) = f(\vect{x}) + \lambda g(\vect{x})
\end{equation}

At the optimum, both:
\begin{equation}
\nabla_{\vect{x}} \mathcal{L} = \boldsymbol{0} \quad \text{and} \quad \frac{\partial \mathcal{L}}{\partial \lambda} = 0
\end{equation}

\subsection{Inequality Constraints}

For inequality constraint $g(\vect{x}) \leq 0$, we use the \textbf{Karush-Kuhn-Tucker (KKT)} conditions:

\begin{align}
\nabla_{\vect{x}} \mathcal{L} &= \boldsymbol{0} \\
\lambda &\geq 0 \\
\lambda g(\vect{x}) &= 0 \quad \text{(complementary slackness)} \\
g(\vect{x}) &\leq 0
\end{align}

\subsection{Projected Gradient Descent}

For constraints defining a set $\mathcal{C}$, \textbf{projected gradient descent} applies:

\begin{equation}
\vect{x}_{t+1} = \text{Proj}_{\mathcal{C}}\left(\vect{x}_t - \alpha \nabla f(\vect{x}_t)\right)
\end{equation}

where $\text{Proj}_{\mathcal{C}}$ projects onto the feasible set.

\subsection{Applications in Deep Learning}

Constrained optimization appears in:
\begin{itemize}
    \item Weight constraints (e.g., unit norm constraints)
    \item Projection to valid probability distributions
    \item Adversarial training with bounded perturbations
    \item Fairness constraints
\end{itemize}

% Chapter 4, Section 4: Numerical Stability and Conditioning

\section{Numerical Stability and Conditioning}
\label{sec:numerical-stability}

\subsection{Condition Number}

The \textbf{condition number} of matrix $\mat{A}$ is:

\begin{equation}
\kappa(\mat{A}) = \|\mat{A}\| \|\mat{A}^{-1}\|
\end{equation}

For symmetric matrices with eigenvalues $\lambda_i$:

\begin{equation}
\kappa(\mat{A}) = \frac{\max_i |\lambda_i|}{\min_i |\lambda_i|}
\end{equation}

High condition numbers indicate numerical instability: small changes in input lead to large changes in output.

\subsection{Ill-Conditioned Matrices}

In deep learning, ill-conditioned Hessians can make optimization difficult. This motivates techniques like:
\begin{itemize}
    \item Batch normalization
    \item Careful weight initialization
    \item Adaptive learning rate methods
    \item Preconditioning
\end{itemize}

\subsection{Gradient Checking}

To verify gradient computations, we use \textbf{finite differences}:

\begin{equation}
\frac{\partial f}{\partial \theta_i} \approx \frac{f(\theta_i + \epsilon) - f(\theta_i - \epsilon)}{2\epsilon}
\end{equation}

This is computationally expensive but useful for debugging.

\subsection{Numerical Precision Trade-offs}

\textbf{Mixed precision training:}
\begin{itemize}
    \item Store weights in FP32
    \item Compute activations/gradients in FP16
    \item Use loss scaling to prevent underflow
    \item 2-3x speedup with minimal accuracy loss
\end{itemize}

\subsection{Practical Tips}

\begin{itemize}
    \item Monitor gradient norms during training
    \item Use gradient clipping for RNNs
    \item Prefer numerically stable implementations (log-space computations)
    \item Be aware of precision limits in very deep networks
\end{itemize}

\label{sec:overflow-underflow}

Computers represent real numbers with finite precision, typically using floating-point arithmetic. This leads to \textbf{rounding errors} that can accumulate and cause problems.

\subsection{Floating-Point Representation}

The IEEE 754 standard defines floating-point numbers. For 32-bit floats:
\begin{itemize}
    \item Smallest positive number: approximately $10^{-38}$
    \item Largest number: approximately $10^{38}$
    \item Machine epsilon: approximately $10^{-7}$
\end{itemize}

\subsection{Underflow}

\textbf{Underflow} occurs when numbers near zero are rounded to zero. This can be problematic when we need to compute ratios or logarithms. For example, the softmax function:

\begin{equation}
\text{softmax}(\vect{x})_i = \frac{\exp(x_i)}{\sum_j \exp(x_j)}
\end{equation}

can underflow if all $x_i$ are very negative.

\subsection{Overflow}

\textbf{Overflow} occurs when large numbers exceed representable values. In the softmax example, overflow can occur if some $x_i$ are very large.

\subsection{Numerical Stability}

To stabilize softmax, we use the identity:

\begin{equation}
\text{softmax}(\vect{x}) = \text{softmax}(\vect{x} - c)
\end{equation}

where $c = \max_i x_i$. This prevents both overflow and underflow.

Similarly, when computing $\log(\sum_i \exp(x_i))$, we use the \textbf{log-sum-exp} trick:

\begin{equation}
\log\left(\sum_i \exp(x_i)\right) = c + \log\left(\sum_i \exp(x_i - c)\right)
\end{equation}

\section{Gradient-Based Optimization}
\label{sec:gradient-optimization}

Most deep learning algorithms involve optimization: finding parameters that minimize or maximize an objective function.

\subsection{Gradient Descent}

For a function $f(\vect{\theta})$, \textbf{gradient descent} updates parameters as:

\begin{equation}
\vect{\theta}_{t+1} = \vect{\theta}_t - \alpha \nabla_{\vect{\theta}} f(\vect{\theta}_t)
\end{equation}

where $\alpha > 0$ is the \textbf{learning rate}.

\subsection{Jacobian and Hessian Matrices}

The \textbf{Jacobian matrix} contains all first-order partial derivatives. For $\vect{f}: \mathbb{R}^n \to \mathbb{R}^m$:

\begin{equation}
\mat{J}_{ij} = \frac{\partial f_i}{\partial x_j}
\end{equation}

The \textbf{Hessian matrix} contains second-order derivatives:

\begin{equation}
\mat{H}_{ij} = \frac{\partial^2 f}{\partial x_i \partial x_j}
\end{equation}

The Hessian characterizes the local curvature of the function.

\subsection{Taylor Series Approximation}

Near point $\vect{x}_0$, we can approximate $f(\vect{x})$ using Taylor series:

\begin{equation}
f(\vect{x}) \approx f(\vect{x}_0) + (\vect{x} - \vect{x}_0)^\top \nabla f(\vect{x}_0) + \frac{1}{2}(\vect{x} - \vect{x}_0)^\top \mat{H}(\vect{x}_0) (\vect{x} - \vect{x}_0)
\end{equation}

This provides insight into optimization behavior.

\subsection{Critical Points}

At a \textbf{critical point}, $\nabla f(\vect{x}) = \boldsymbol{0}$. The Hessian determines the nature:
\begin{itemize}
    \item \textbf{Local minimum:} Hessian is positive definite
    \item \textbf{Local maximum:} Hessian is negative definite
    \item \textbf{Saddle point:} Hessian has both positive and negative eigenvalues
\end{itemize}

Deep learning often encounters saddle points rather than local minima in high dimensions.

\subsection{Directional Derivatives}

The directional derivative in direction $\vect{u}$ (with $\|\vect{u}\| = 1$) is:

\begin{equation}
\frac{\partial}{\partial \alpha} f(\vect{x} + \alpha \vect{u}) \bigg|_{\alpha=0} = \vect{u}^\top \nabla f(\vect{x})
\end{equation}

To minimize $f$, we move in the direction $\vect{u} = -\frac{\nabla f(\vect{x})}{\|\nabla f(\vect{x})\|}$.

\section{Constrained Optimization}
\label{sec:constrained-optimization}

Many problems require optimizing a function subject to constraints.

\subsection{Lagrange Multipliers}

For equality constraint $g(\vect{x}) = 0$, the \textbf{Lagrangian} is:

\begin{equation}
\mathcal{L}(\vect{x}, \lambda) = f(\vect{x}) + \lambda g(\vect{x})
\end{equation}

At the optimum, both:
\begin{equation}
\nabla_{\vect{x}} \mathcal{L} = \boldsymbol{0} \quad \text{and} \quad \frac{\partial \mathcal{L}}{\partial \lambda} = 0
\end{equation}

\subsection{Inequality Constraints}

For inequality constraint $g(\vect{x}) \leq 0$, we use the \textbf{Karush-Kuhn-Tucker (KKT)} conditions:

\begin{align}
\nabla_{\vect{x}} \mathcal{L} &= \boldsymbol{0} \\
\lambda &\geq 0 \\
\lambda g(\vect{x}) &= 0 \quad \text{(complementary slackness)} \\
g(\vect{x}) &\leq 0
\end{align}

\subsection{Projected Gradient Descent}

For constraints defining a set $\mathcal{C}$, \textbf{projected gradient descent} applies:

\begin{equation}
\vect{x}_{t+1} = \text{Proj}_{\mathcal{C}}\left(\vect{x}_t - \alpha \nabla f(\vect{x}_t)\right)
\end{equation}

where $\text{Proj}_{\mathcal{C}}$ projects onto the feasible set.

\section{Numerical Stability and Conditioning}
\label{sec:numerical-stability}

\subsection{Condition Number}

The \textbf{condition number} of matrix $\mat{A}$ is:

\begin{equation}
\kappa(\mat{A}) = \|\mat{A}\| \|\mat{A}^{-1}\|
\end{equation}

For symmetric matrices with eigenvalues $\lambda_i$:

\begin{equation}
\kappa(\mat{A}) = \frac{\max_i |\lambda_i|}{\min_i |\lambda_i|}
\end{equation}

High condition numbers indicate numerical instability: small changes in input lead to large changes in output.

\subsection{Ill-Conditioned Matrices}

In deep learning, ill-conditioned Hessians can make optimization difficult. This motivates techniques like:
\begin{itemize}
    \item Batch normalization
    \item Careful weight initialization
    \item Adaptive learning rate methods
    \item Preconditioning
\end{itemize}

\subsection{Gradient Checking}

To verify gradient computations, we use \textbf{finite differences}:

\begin{equation}
\frac{\partial f}{\partial \theta_i} \approx \frac{f(\theta_i + \epsilon) - f(\theta_i - \epsilon)}{2\epsilon}
\end{equation}

This is computationally expensive but useful for debugging.
