% Chapter 15, Section 2

\section{Transfer Learning and Domain Adaptation}
\label{sec:transfer-learning}

\subsection{Transfer Learning}

Leverage knowledge from source task to improve target task:

\textbf{Feature extraction:}
\begin{enumerate}
    \item Pre-train on large dataset (e.g., ImageNet)
    \item Freeze convolutional layers
    \item Train only final classification layers on target task
\end{enumerate}

\textbf{Fine-tuning:}
\begin{enumerate}
    \item Start with pre-trained model
    \item Continue training on target task with lower learning rate
    \item Optionally freeze early layers
\end{enumerate}

\subsection{Domain Adaptation}

Adapt model when training (source) and test (target) distributions differ.

\textbf{Approaches:}
\begin{itemize}
    \item \textbf{Domain-adversarial training:} Learn domain-invariant features
    \item \textbf{Self-training:} Use confident predictions on target domain
    \item \textbf{Multi-task learning:} Joint training on both domains
\end{itemize}

\subsection{Few-Shot Learning}

Learn from few examples per class:
\begin{itemize}
    \item \textbf{Meta-learning:} Learn to learn quickly (MAML)
    \item \textbf{Prototypical networks:} Learn metric space
    \item \textbf{Matching networks:} Attention-based comparison
\end{itemize}

