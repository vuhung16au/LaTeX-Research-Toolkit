% Chapter 7, Section 2

\section{Dataset Augmentation}
\label{sec:data-augmentation}

\textbf{Data augmentation} artificially increases training set size by applying transformations that preserve labels.

\subsection{Image Augmentation}

Common transformations:
\begin{itemize}
    \item \textbf{Geometric:} rotation, translation, scaling, flipping, cropping
    \item \textbf{Color:} brightness, contrast, saturation adjustments
    \item \textbf{Noise:} Gaussian noise, blur
    \item \textbf{Cutout/Erasing:} randomly mask regions
    \item \textbf{Mixup:} blend pairs of images and labels
\end{itemize}

Example: horizontal flip
\begin{equation}
\vect{x}_{\text{aug}} = \text{flip}(\vect{x}), \quad y_{\text{aug}} = y
\end{equation}

\subsection{Text Augmentation}

For NLP:
\begin{itemize}
    \item Synonym replacement
    \item Random insertion/deletion
    \item Back-translation
    \item Paraphrasing
\end{itemize}

\subsection{Audio Augmentation}

For speech/audio:
\begin{itemize}
    \item Time stretching
    \item Pitch shifting
    \item Adding background noise
    \item SpecAugment (masking frequency/time regions)
\end{itemize}

