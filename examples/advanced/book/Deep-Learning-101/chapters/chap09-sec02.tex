% Chapter 9, Section 2

\section{Pooling}
\label{sec:pooling}

\textbf{Pooling} reduces spatial dimensions and provides translation invariance.

\subsection{Max Pooling}

Takes maximum value in each pooling region:
\begin{equation}
\text{MaxPool}(i,j) = \max_{m,n \in \mathcal{R}_{ij}} I(m,n)
\end{equation}

Common: $2 \times 2$ max pooling with stride 2 (halves spatial dimensions).

\subsection{Average Pooling}

Computes average:
\begin{equation}
\text{AvgPool}(i,j) = \frac{1}{|\mathcal{R}_{ij}|} \sum_{m,n \in \mathcal{R}_{ij}} I(m,n)
\end{equation}

\subsection{Global Pooling}

Pools over entire spatial dimensions:
\begin{itemize}
    \item \textbf{Global Average Pooling (GAP):} average over all spatial locations
    \item \textbf{Global Max Pooling:} maximum over all spatial locations
\end{itemize}

Useful for reducing parameters before fully connected layers.

\subsection{Alternative: Strided Convolutions}

Modern architectures sometimes replace pooling with strided convolutions to learn downsampling.

