% Chapter 19, Section 1

\section{Variational Inference}
\label{sec:variational-inference}

\subsection{Evidence Lower Bound (ELBO)}

For latent variable model with intractable posterior $p(\vect{z}|\vect{x})$, approximate with $q(\vect{z})$:

\begin{align}
\log p(\vect{x}) &= \mathbb{E}_{q(\vect{z})}[\log p(\vect{x})] \\
&= \mathbb{E}_{q(\vect{z})}\left[\log \frac{p(\vect{x}, \vect{z})}{p(\vect{z}|\vect{x})}\right] \\
&= \mathbb{E}_{q(\vect{z})}\left[\log \frac{p(\vect{x}, \vect{z})}{q(\vect{z})}\right] + D_{KL}(q(\vect{z}) \| p(\vect{z}|\vect{x})) \\
&\geq \mathbb{E}_{q(\vect{z})}\left[\log \frac{p(\vect{x}, \vect{z})}{q(\vect{z})}\right] = \mathcal{L}(q)
\end{align}

Maximizing $\mathcal{L}(q)$ minimizes $D_{KL}(q(\vect{z}) \| p(\vect{z}|\vect{x}))$.

\subsection{Variational Family}

Choose tractable family of distributions:

\textbf{Mean field:} Fully factorized
\begin{equation}
q(\vect{z}) = \prod_{i=1}^{n} q_i(z_i)
\end{equation}

\textbf{Structured:} Allow some dependencies
\begin{equation}
q(\vect{z}) = \prod_{c} q_c(\vect{z}_c)
\end{equation}

Trade-off between expressiveness and tractability.

\subsection{Coordinate Ascent VI}

Optimize each factor iteratively:
\begin{equation}
q_j^*(z_j) \propto \exp\left(\mathbb{E}_{q_{-j}}[\log p(\vect{z}, \vect{x})]\right)
\end{equation}

Guaranteed to converge to local optimum of ELBO.

\subsection{Stochastic Variational Inference}

Use stochastic gradients for scalability:
\begin{itemize}
    \item Mini-batch data
    \item Monte Carlo estimation of expectations
    \item Reparameterization trick for low variance
\end{itemize}

