% Chapter 9, Section 1

\section{The Convolution Operation}
\label{sec:convolution}

\subsection{Definition}

The \textbf{convolution} operation applies a filter (kernel) across an input:

For discrete 2D convolution:
\begin{equation}
S(i,j) = (I * K)(i,j) = \sum_m \sum_n I(i-m, j-n) K(m, n)
\end{equation}

where $I$ is the input and $K$ is the kernel.

In practice, we often use \textbf{cross-correlation}:
\begin{equation}
S(i,j) = (I * K)(i,j) = \sum_m \sum_n I(i+m, j+n) K(m, n)
\end{equation}

\subsection{Properties}

\textbf{Parameter sharing:} Same kernel applied across spatial locations
\begin{itemize}
    \item Reduces parameters compared to fully connected layers
    \item Enables translation equivariance
\end{itemize}

\textbf{Local connectivity:} Each output depends on local input region
\begin{itemize}
    \item Exploits spatial locality in images
    \item Hierarchically builds complex features
\end{itemize}

\subsection{Multi-Channel Convolution}

For input with $C_{\text{in}}$ channels and $C_{\text{out}}$ output channels:
\begin{equation}
S_{c_{\text{out}}}(i,j) = \sum_{c_{\text{in}}=1}^{C_{\text{in}}} (I_{c_{\text{in}}} * K_{c_{\text{out}}, c_{\text{in}}})(i,j) + b_{c_{\text{out}}}
\end{equation}

\subsection{Hyperparameters}

\textbf{Kernel size:} Typically $3 \times 3$ or $5 \times 5$

\textbf{Stride:} Step size for sliding kernel (stride $s$):
\begin{equation}
\text{Output size} = \left\lfloor \frac{n - k}{s} \right\rfloor + 1
\end{equation}

\textbf{Padding:} Add zeros around input
\begin{itemize}
    \item \textbf{Valid:} no padding
    \item \textbf{Same:} padding to preserve spatial size
    \item \textbf{Full:} maximum padding
\end{itemize}

For "same" padding with stride 1:
\begin{equation}
p = \left\lfloor \frac{k-1}{2} \right\rfloor
\end{equation}

