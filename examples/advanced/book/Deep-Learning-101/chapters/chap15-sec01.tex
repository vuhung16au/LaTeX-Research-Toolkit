% Chapter 15, Section 1

\section{What Makes a Good Representation?}
\label{sec:good-representations}

\subsection{Desirable Properties}

\textbf{Disentanglement:} Different factors of variation are separated
\begin{itemize}
    \item Changes in one dimension affect one factor
    \item Easier interpretation and manipulation
\end{itemize}

\textbf{Invariance:} Representation unchanged under irrelevant transformations
\begin{itemize}
    \item Translation, rotation invariance for objects
    \item Speaker invariance for speech content
\end{itemize}

\textbf{Smoothness:} Similar inputs have similar representations
\begin{itemize}
    \item Enables generalization
    \item Supports interpolation
\end{itemize}

\textbf{Sparsity:} Few features active for each input
\begin{itemize}
    \item Computational efficiency
    \item Interpretability
\end{itemize}

\subsection{Manifold Hypothesis}

Natural data lies on low-dimensional manifolds embedded in high-dimensional space.

Deep learning learns to:
\begin{itemize}
    \item Discover the manifold structure
    \item Map data to meaningful coordinates on manifold
\end{itemize}

