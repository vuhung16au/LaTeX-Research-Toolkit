% Chapter 20, Section 2

\section{Generative Adversarial Networks (GANs)}
\label{sec:gans}

\subsection{Core Idea}

Two networks compete:
\begin{itemize}
    \item \textbf{Generator} $G$: Creates fake samples from noise
    \item \textbf{Discriminator} $D$: Distinguishes real from fake
\end{itemize}

\subsection{Objective}

Minimax game:
\begin{equation}
\min_G \max_D \mathbb{E}_{\vect{x} \sim p_{\text{data}}}[\log D(\vect{x})] + \mathbb{E}_{\vect{z} \sim p(\vect{z})}[\log(1 - D(G(\vect{z})))]
\end{equation}

\subsection{Training Procedure}

Alternate updates:
\begin{enumerate}
    \item \textbf{Update D:} Maximize discrimination
    \begin{equation}
    \max_D \mathbb{E}_{\vect{x}}[\log D(\vect{x})] + \mathbb{E}_{\vect{z}}[\log(1 - D(G(\vect{z})))]
    \end{equation}
    
    \item \textbf{Update G:} Minimize discrimination (or maximize $\log D(G(\vect{z}))$)
    \begin{equation}
    \min_G \mathbb{E}_{\vect{z}}[\log(1 - D(G(\vect{z})))]
    \end{equation}
\end{enumerate}

\subsection{Training Challenges}

\textbf{Mode collapse:} Generator produces limited variety

\textbf{Training instability:} Oscillations, non-convergence

\textbf{Vanishing gradients:} When discriminator too strong

\subsection{GAN Variants}

\textbf{DCGAN:} Deep Convolutional GAN with architectural guidelines

\textbf{WGAN:} Wasserstein GAN with improved training stability

\textbf{StyleGAN:} High-quality image generation with style control

\textbf{Conditional GAN:} Generate from class labels

\textbf{CycleGAN:} Unpaired image-to-image translation

