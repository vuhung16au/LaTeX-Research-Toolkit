\documentclass[11pt]{article}

% --- Packages used in the Preamble ---
\usepackage{amsmath}    % For math environments like align
\usepackage{amsthm}     % For theorem environments (10:22)
\usepackage{minted}     % For including syntax-highlighted code (13:48)
\usepackage{tcolorbox}  % For highlighting text (11:40)
\usepackage{witharrows} % For annotating equations (11:40)
\usepackage{cleveref}   % For smart referencing (12:31)
\usepackage{xifthen}    % For extended conditionals (1:58)
\usepackage{ifthen}     % For \ifthenelse and \equal
\usepackage{xcolor}     % For colors and \colorbox

% --- 0:28 New Commands ---
% Define a custom command for a partial derivative (takes 2 arguments: top and bottom)
\newcommand{\PD}[2]{\frac{\partial #1}{\partial #2}}

% --- 1:58 Conditional Formatting ---
% Define a variable to toggle solutions on/off.
% If \conditionalformat is defined as 'S', solutions are shown.
\newcommand{\conditionalformat}{S} % Change 'S' to empty {} to hide solutions

% Define the solution command (takes 1 argument: the solution text)
\newcommand{\solution}[1]{%
    \textbf{Solution:}~% space after label
    \ifthenelse{\equal{\conditionalformat}{S}}{#1}{There is no solution.}%
}

% Theorem setup
\newtheorem{theorem}{Theorem}

\title{My Favorite \\LaTeX\\ Tricks to Save Time}
\author{Vu's Efficiency Toolkit}
\date{\today}

\begin{document}
\maketitle

\section{Custom Commands and Conditional Formatting}

\subsection{New Commands Example (0:28)}
The partial derivative of $Z$ with respect to $X$ is written compactly using our custom command:
$$\PD{Z}{X}$$

\subsection{Conditional Toggle (1:58)}
\noindent \textbf{Question 1:} Find the derivative of $y=x^2$.

\solution{The derivative is $\PD{y}{x} = 2x$.}

\section{Code Inclusion and File Management}

\subsection{Including Code with Minted (13:48)}
Using the \texttt{minted} package provides superior syntax highlighting for code blocks.
\begin{listing}[h]
\begin{minted}[
    linenos,
    frame=single,
    bgcolor=gray!10
]{python}
# GNN Initialization for your PhD project
class GNNLayer(MessagePassing):
    def forward(self, x, edge_index):
        # 1. Message Passing (Propagation)
        return self.propagate(edge_index, x=x)
\end{minted}
\caption{Python GNN function using the \texttt{minted} package.}
\label{lst:python_gnn}
\end{listing}

\subsection{Using Multiple Files (5:37)}
% This demonstrates the concept of including files
% To use \includeonly, move the command BEFORE \begin{document}.
This content comes from the main file.
\subsection*{Sub-section One (Input Example)}
Content from the first sub-file. This was included using \texttt{\textbackslash input}.


 % \input does NOT start a new page
\subsection*{Sub-section Two (Include Example)}
Content from the second sub-file. This was added using \texttt{\textbackslash include}, so it starts on a new page.


 % use input to avoid include nesting restrictions

\section{Academic Styling and Referencing}

\subsection{Be Semantic (9:25)}
\textbf{The Semantic Approach:} It is better to use \texttt{\textbackslash emph\{word\}} (emphasis) than \texttt{\textbackslash textit\{word\}} (italic).
\begin{quote}
    In normal text, \emph{emphasis} looks like \textit{italics}.
\end{quote}
However, inside a \texttt{theorem} environment, the formatting changes automatically:
\begin{theorem}
    If the text is already italicized, \emph{emphasis} becomes non-italicized for contrast.
\end{theorem}
This focuses on the \emph{meaning}, not the raw formatting.

\subsection{Annotating Equations (11:40)}
We can use \texttt{\textbackslash underbrace} and \texttt{tcolorbox} to draw attention to specific parts.
\begin{equation}
    E = \underbrace{m \cdot c^2}_{\text{\colorbox{yellow}{Energy Term}}} +
    \underbrace{p \cdot c}_{\text{Momentum Term}}
    \label{eq:einstein}
\end{equation}

% Example WithArrows usage (optional demonstration)
\[
\begin{WithArrows}
    E = m c^2 + p c & \Arrow{Interpreting the components} \\
    = \underbrace{m c^2}_{\text{rest energy}} + \underbrace{p c}_{\text{momentum term}} &
\end{WithArrows}
\]

\subsection{Clever Referencing (12:31)}
\textbf{Cleveref:} We can reference multiple items in a compact, smart way.

The custom GNN code is located in \cref{lst:python_gnn}.
We can also reference the equation and the listing together: \cref{eq:einstein,lst:python_gnn}.
Notice how \texttt{\textbackslash cref} automatically inserts labels and handles pluralization.

% --- File Structure Refresh (For the include/input commands to work) ---
% These files exist separately in this folder:
% 1. subsection_one.tex
% 2. subsection_two.tex

\end{document}


