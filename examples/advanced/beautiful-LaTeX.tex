\documentclass[11pt]{article}

% --- 0:47 Margins and geometry ---
\usepackage[
    top=1.5in,
    bottom=1in,
    left=1in,
    right=2in % Demonstrating an oversized margin for margin notes
]{geometry}

% --- 0:46 Columns and minipages ---
\usepackage{multicol} % For multi-column sections
\usepackage{graphicx} % For including images
\usepackage{lipsum} % For generating placeholder text
\usepackage{parskip} % For adjusting paragraph spacing (Video: 9:31)
\usepackage{tcolorbox} % For highlighting text (Video: 11:40)
% \usepackage{witharrows} % For annotating equations (Video: 11:40) - Commented out as not standard
\usepackage{amsmath} % For math environments like align
\usepackage{minted} % For including syntax-highlighted code (Video: 13:48)
\usepackage{xcolor}

% Define a light background color used by minted example
\definecolor{mintedbg}{RGB}{245,245,245}

\setlength{\parindent}{0pt} % Set indent to 0pt as discussed in video

\begin{document}

\title{The Beautiful \LaTeX\ Document: A Style Refresher}
\author{Vu's Academic Toolkit}
\date{\today}
\maketitle

\section{Layout and Columns (0:46)}

This section demonstrates how to manage columns and place content side-by-side using \texttt{multicol} and \texttt{minipage}.

\subsection{Multicol Environment}

% Example of using the multicol environment
\begin{multicols}{2}
    \section*{Two-Column Section} % Using section* to avoid numbering
    \noindent\hrulefill % Horizontal line for separation
    \lipsum[1-2]
    \columnbreak % Force content split into the next column
    \lipsum[3]
\end{multicols}

\subsection{Minipage Environment (2:29)}
% Minipage example using widths and hspace

\noindent
\begin{minipage}[t]{0.6\linewidth} % 60% of the text width
    \centering
    \textbf{Minipage 1 (60\% Width)}
    \hrulefill
    \lipsum[4][1-5] % First 5 sentences of paragraph 4
\end{minipage}%
\hspace{10pt}% Small horizontal space (3:40)
\begin{minipage}[t]{0.3\linewidth} % 30% of the text width
    \centering
    \textbf{Minipage 2 (30\% Width)}
    \hrulefill
    \lipsum[5][1-5]
\end{minipage}

\section{Margins and Margin Notes (4:47)}

\lipsum[6] % Text to provide context for the margin note

This paragraph is aligned with the margin note, demonstrating the \texttt{geometry} package's effect on margins, which can be useful for adding extra comments or short summaries. \marginpar{\textbf{Margin Par:} This text appears in the oversized margin, demonstrating the $\mathbf{2in}$ right margin set by the \texttt{geometry} package.}
The document body remains readable while the note is placed aside.

\section{Whitespace and Alignment (6:56)}

\subsection{Math Mode Spacing (7:10)}
% Demonstrate spaces in math mode
Let the initial conditions be:
\begin{align*}
    y'(0) &= 1 \quad \text{and} \quad y(0) = 0 \\
    z &= x \! + \! y % Negative space for testing
\end{align*}
Here, \texttt{\\quad} gives a large standardized space, and \texttt{\\!} gives negative space.

\subsection{Horizontal Fill (8:21)}
\noindent This text is pushed left\hfill and this text is pushed right using \texttt{\textbackslash hfill}.

\section{Annotating Equations (11:40)}

We can annotate the steps of a derivative using braces and arrows.

\begin{align}
    \label{eq:derivative}
    \frac{dy}{dx} &= \underbrace{\left(\frac{\partial y}{\partial t}\right)}_{\text{Temporal Change}} \frac{dt}{dx} \! + \! \underbrace{\left(\frac{\partial y}{\partial z}\right)}_{\text{Spatial Gradient}} \frac{dz}{dx}
\end{align}

We take the Laplace transform of the equation $\mathbf{Y}$ to $\mathbf{Y}(\mathbf{s})$:
% Note: witharrows package not available, using standard align instead
\begin{align}
    \frac{d^2y}{dt^2} + y = \mathbf{f}(t) \quad &\text{(Original equation)}\\
    s^2\mathbf{Y}(s) + \mathbf{Y}(s) = \mathbf{F}(s) \quad &\text{(After Laplace Transform)}
\end{align}

\section{Including Code with Minted (13:48)}

The \texttt{minted} package allows for professional, syntax-highlighted inclusion of code, which is essential for your IT proposal.

\begin{listing}[h]
\begin{minted}[
    linenos,       % Line numbers
    frame=lines,   % Framed box
    bgcolor=mintedbg % Light background color
]{python}
# Example Python for GNN initialization
import torch_geometric.nn as tgnn

class GNN_Transformer(tgnn.MessagePassing):
    def __init__(self, in_channels, out_channels):
        super(GNN_Transformer, self).__init__(aggr='add')
        self.lin = torch.nn.Linear(in_channels, out_channels)

    def forward(self, x, edge_index):
        # 1. Message Passing
        return self.propagate(edge_index, x=x)
\end{minted}
\caption{Python implementation using the \texttt{minted} package.}
\label{lst:python_example}
\end{listing}

\end{document}


