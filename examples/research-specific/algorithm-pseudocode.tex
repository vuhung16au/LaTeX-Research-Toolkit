\documentclass{article}
\usepackage[utf8]{inputenc}
\usepackage{amsmath,amsfonts,amssymb}
\usepackage{listings}
\usepackage{xcolor}

% Define style for pseudocode
\lstdefinestyle{pseudocode}{
    basicstyle=\ttfamily\small,
    keywordstyle=\color{blue}\bfseries,
    commentstyle=\color{green!60!black},
    stringstyle=\color{red},
    numbers=left,
    numberstyle=\tiny\color{gray},
    stepnumber=1,
    numbersep=5pt,
    backgroundcolor=\color{gray!10},
    frame=single,
    tabsize=2,
    captionpos=b,
    breaklines=true,
    breakatwhitespace=false,
    escapeinside={(*@}{@*)},
    morekeywords={Input,Output,Initialize,For,To,Return,If,ElsIf,Else,EndIf,EndFor,Repeat,Until,State,Comment},
    mathescape=true
}

\title{Algorithm Pseudocode in LaTeX}
\author{Your Name}
\date{\today}

\begin{document}

\maketitle

\section{Basic Algorithm}
Here's a basic algorithm example:

\begin{lstlisting}[style=pseudocode,caption={Basic Algorithm},label=alg:basic]
Input: Input data $x_1, x_2, \ldots, x_n$
Output: Output result $y$
Initialize: $y = 0$
For i = 1 to n:
    $y = y + x_i$
Return $y$
\end{lstlisting}

\section{Complex Algorithm}
Here's a more complex algorithm with multiple steps:

\begin{lstlisting}[style=pseudocode,caption={Complex Algorithm},label=alg:complex]
Input: Input matrix $A$, vector $b$, tolerance $\epsilon$
Output: Solution vector $x$
Initialize: $x^{(0)} = 0$
$k = 0$
Repeat:
    $k = k + 1$
    For i = 1 to n:
        $x_i^{(k)} = \frac{1}{a_{ii}}(b_i - \sum_{j=1}^{i-1} a_{ij}x_j^{(k)} - \sum_{j=i+1}^{n} a_{ij}x_j^{(k-1)})$
Until: $\|x^{(k)} - x^{(k-1)}\| < \epsilon$
Return: $x^{(k)}$
\end{lstlisting}

\section{Algorithm with Comments}
Here's an algorithm with detailed comments:

\begin{lstlisting}[style=pseudocode,caption={Algorithm with Comments},label=alg:comments]
Input: Input data $D = \{d_1, d_2, \ldots, d_n\}$
Output: Clustered data $C = \{c_1, c_2, \ldots, c_k\}$
Comment: Initialize centroids randomly
$c_1, c_2, \ldots, c_k = \text{random\_centroids}()$
Comment: Main clustering loop
Repeat:
    Comment: Assign each point to nearest centroid
    For i = 1 to n:
        $j = \arg\min_{j} \|d_i - c_j\|^2$
        Assign $d_i$ to cluster $j$
    Comment: Update centroids
    For j = 1 to k:
        $c_j = \frac{1}{|C_j|} \sum_{d_i \in C_j} d_i$
Until: convergence
Return: $C$
\end{lstlisting}

\section{Algorithm with Conditions}
Here's an algorithm with conditional statements:

\begin{lstlisting}[style=pseudocode,caption={Conditional Algorithm},label=alg:conditional]
Input: Input value $x$, threshold $t$
Output: Processed value $y$
If $x > t$:
    $y = x^2$
    Comment: Apply square transformation
ElsIf $x > 0$:
    $y = x$
    Comment: Keep original value
Else:
    $y = 0$
    Comment: Set to zero for negative values
EndIf
Return $y$
\end{lstlisting}

\section{Algorithm with Loops}
Here's an algorithm with nested loops:

\begin{lstlisting}[style=pseudocode,caption={Nested Loop Algorithm},label=alg:nested]
Input: Input matrix $A$ of size $n \times n$
Output: Result matrix $B$ of size $n \times n$
For i = 1 to n:
    For j = 1 to n:
        $B_{ij} = 0$
        For k = 1 to n:
            $B_{ij} = B_{ij} + A_{ik} \cdot A_{kj}$
Return $B$
\end{lstlisting}

\section{Algorithm References}
As shown in Algorithm~\ref{alg:basic}, the basic approach is straightforward.

The complex algorithm (Algorithm~\ref{alg:complex}) demonstrates advanced techniques.

Algorithm~\ref{alg:conditional} shows how to handle different cases.

\section{Best Practices for Algorithm Pseudocode}
\begin{itemize}
    \item Use clear, descriptive variable names
    \item Include input and output specifications
    \item Add comments for complex steps
    \item Use consistent formatting
    \item Test the algorithm logic
    \item Keep algorithms focused and readable
\end{itemize}

\end{document}