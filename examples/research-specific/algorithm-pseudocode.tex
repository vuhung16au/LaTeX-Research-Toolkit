\documentclass{article}
\usepackage[utf8]{inputenc}
\usepackage{amsmath,amsfonts,amssymb}
\usepackage{algorithm}
\usepackage{algorithmic}
\usepackage{algpseudocode}

\title{Algorithm Pseudocode in LaTeX}
\author{Your Name}
\date{\today}

\begin{document}

\maketitle

\section{Basic Algorithm}
Here's a basic algorithm example:

\begin{algorithm}
\caption{Basic Algorithm}
\label{alg:basic}
\begin{algorithmic}[1]
\REQUIRE Input data $x_1, x_2, \ldots, x_n$
\ENSURE Output result $y$
\STATE Initialize $y = 0$
\FOR{$i = 1$ to $n$}
    \STATE $y = y + x_i$
\ENDFOR
\RETURN $y$
\end{algorithmic}
\end{algorithm}

\section{Complex Algorithm}
Here's a more complex algorithm with multiple steps:

\begin{algorithm}
\caption{Complex Algorithm}
\label{alg:complex}
\begin{algorithmic}[1]
\REQUIRE Input matrix $A$, vector $b$, tolerance $\epsilon$
\ENSURE Solution vector $x$
\STATE Initialize $x^{(0)} = 0$
\STATE $k = 0$
\REPEAT
    \STATE $k = k + 1$
    \FOR{$i = 1$ to $n$}
        \STATE $x_i^{(k)} = \frac{1}{a_{ii}}\left(b_i - \sum_{j=1}^{i-1} a_{ij}x_j^{(k)} - \sum_{j=i+1}^{n} a_{ij}x_j^{(k-1)}\right)$
    \ENDFOR
\UNTIL{$\|x^{(k)} - x^{(k-1)}\| < \epsilon$}
\RETURN $x^{(k)}$
\end{algorithmic}
\end{algorithm}

\section{Algorithm with Comments}
Here's an algorithm with detailed comments:

\begin{algorithm}
\caption{Algorithm with Comments}
\label{alg:comments}
\begin{algorithmic}[1]
\REQUIRE Input data $D = \{d_1, d_2, \ldots, d_n\}$
\ENSURE Clustered data $C = \{c_1, c_2, \ldots, c_k\}$
\STATE \COMMENT{Initialize centroids randomly}
\STATE $c_1, c_2, \ldots, c_k = \text{random\_centroids}()$
\STATE \COMMENT{Main clustering loop}
\REPEAT
    \STATE \COMMENT{Assign each point to nearest centroid}
    \FOR{$i = 1$ to $n$}
        \STATE $j = \arg\min_{j} \|d_i - c_j\|^2$
        \STATE Assign $d_i$ to cluster $j$
    \ENDFOR
    \STATE \COMMENT{Update centroids}
    \FOR{$j = 1$ to $k$}
        \STATE $c_j = \frac{1}{|C_j|} \sum_{d_i \in C_j} d_i$
    \ENDFOR
\UNTIL{convergence}
\RETURN $C$
\end{algorithmic}
\end{algorithm}

\section{Algorithm with Conditions}
Here's an algorithm with conditional statements:

\begin{algorithm}
\caption{Conditional Algorithm}
\label{alg:conditional}
\begin{algorithmic}[1]
\REQUIRE Input value $x$, threshold $t$
\ENSURE Processed value $y$
\IF{$x > t$}
    \STATE $y = x^2$
    \STATE \COMMENT{Apply square transformation}
\ELSIF{$x > 0$}
    \STATE $y = x$
    \STATE \COMMENT{Keep original value}
\ELSE
    \STATE $y = 0$
    \STATE \COMMENT{Set to zero for negative values}
\ENDIF
\RETURN $y$
\end{algorithmic}
\end{algorithm}

\section{Algorithm with Loops}
Here's an algorithm with nested loops:

\begin{algorithm}
\caption{Nested Loop Algorithm}
\label{alg:nested}
\begin{algorithmic}[1]
\REQUIRE Input matrix $A$ of size $n \times n$
\ENSURE Result matrix $B$ of size $n \times n$
\FOR{$i = 1$ to $n$}
    \FOR{$j = 1$ to $n$}
        \STATE $B_{ij} = 0$
        \FOR{$k = 1$ to $n$}
            \STATE $B_{ij} = B_{ij} + A_{ik} \cdot A_{kj}$
        \ENDFOR
    \ENDFOR
\ENDFOR
\RETURN $B$
\end{algorithmic}
\end{algorithm}

\section{Algorithm References}
As shown in Algorithm~\ref{alg:basic}, the basic approach is straightforward.

The complex algorithm (Algorithm~\ref{alg:complex}) demonstrates advanced techniques.

Algorithm~\ref{alg:conditional} shows how to handle different cases.

\section{Best Practices for Algorithm Pseudocode}
\begin{itemize}
    \item Use clear, descriptive variable names
    \item Include input and output specifications
    \item Add comments for complex steps
    \item Use consistent formatting
    \item Test the algorithm logic
    \item Keep algorithms focused and readable
\end{itemize}

\end{document}
