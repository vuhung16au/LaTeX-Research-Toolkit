\documentclass{article}
\usepackage[utf8]{inputenc}
\usepackage{amsmath,amsfonts,amssymb}
\usepackage{booktabs}
\usepackage{array}
\usepackage{multirow}
\usepackage{rotating}
\usepackage{graphicx}

\title{Data Tables in LaTeX}
\author{Your Name}
\date{\today}

\begin{document}

\maketitle

\section{Basic Data Table}
Here's a basic data table with experimental results:

\begin{table}[h]
\centering
\begin{tabular}{|l|c|c|c|}
\hline
\textbf{Method} & \textbf{Accuracy} & \textbf{Precision} & \textbf{Recall} \\
\hline
Baseline & 85.2\% & 82.1\% & 88.3\% \\
Method A & 89.5\% & 87.3\% & 91.7\% \\
Method B & 92.1\% & 90.8\% & 93.4\% \\
\hline
\end{tabular}
\caption{Performance comparison of different methods}
\label{tab:performance}
\end{table}

\section{Professional Data Table}
Here's a more professional table using booktabs:

\begin{table}[h]
\centering
\begin{tabular}{lccc}
\toprule
Method & Accuracy & Precision & Recall \\
\midrule
Baseline & 85.2\% & 82.1\% & 88.3\% \\
Method A & 89.5\% & 87.3\% & 91.7\% \\
Method B & 92.1\% & 90.8\% & 93.4\% \\
\bottomrule
\end{tabular}
\caption{Performance comparison using booktabs}
\label{tab:booktabs}
\end{table}

\section{Statistical Results Table}
Here's a table with statistical results including standard deviations:

\begin{table}[h]
\centering
\begin{tabular}{lcc}
\toprule
\textbf{Variable} & \textbf{Mean ± SD} & \textbf{Range} \\
\midrule
Age & 35.2 ± 8.7 & 18-65 \\
Height (cm) & 170.5 ± 9.2 & 150-190 \\
Weight (kg) & 68.3 ± 12.1 & 45-95 \\
BMI & 23.4 ± 3.2 & 18.5-30.0 \\
\bottomrule
\end{tabular}
\caption{Descriptive statistics of study participants}
\label{tab:stats}
\end{table}

\section{Multi-row Table}
Here's a table with multi-row entries:

\begin{table}[h]
\centering
\begin{tabular}{|l|c|c|}
\hline
\textbf{Category} & \textbf{Subcategory} & \textbf{Value} \\
\hline
\multirow{3}{*}{Group A} & Subcategory 1 & 10.5 \\
& Subcategory 2 & 15.2 \\
& Subcategory 3 & 8.7 \\
\hline
\multirow{2}{*}{Group B} & Subcategory 1 & 12.3 \\
& Subcategory 2 & 18.9 \\
\hline
\end{tabular}
\caption{Multi-row table example}
\label{tab:multirow}
\end{table}

\section{Rotated Table}
Here's a table with rotated text for better fit:

\begin{table}[h]
\centering
\begin{sideways}
\begin{tabular}{|l|c|c|c|c|}
\hline
\textbf{Method} & \textbf{Dataset 1} & \textbf{Dataset 2} & \textbf{Dataset 3} & \textbf{Average} \\
\hline
Method A & 85.2\% & 87.1\% & 83.5\% & 85.3\% \\
Method B & 89.5\% & 91.2\% & 88.7\% & 89.8\% \\
Method C & 92.1\% & 93.8\% & 90.4\% & 92.1\% \\
\hline
\end{tabular}
\end{sideways}
\caption{Rotated table for better space utilization}
\label{tab:rotated}
\end{table}

\section{Long Table}
Here's a long table that spans multiple pages:

\begin{table}[h]
\centering
\begin{tabular}{|l|c|c|c|c|c|}
\hline
\textbf{ID} & \textbf{Name} & \textbf{Age} & \textbf{Score} & \textbf{Grade} & \textbf{Status} \\
\hline
001 & Alice Johnson & 25 & 95 & A & Pass \\
002 & Bob Smith & 30 & 87 & B & Pass \\
003 & Carol Davis & 28 & 92 & A & Pass \\
004 & David Wilson & 35 & 78 & C & Pass \\
005 & Eve Brown & 22 & 96 & A & Pass \\
006 & Frank Miller & 40 & 85 & B & Pass \\
007 & Grace Lee & 27 & 89 & B & Pass \\
008 & Henry Taylor & 33 & 91 & A & Pass \\
009 & Irene Clark & 29 & 83 & B & Pass \\
010 & Jack Anderson & 31 & 88 & B & Pass \\
\hline
\end{tabular}
\caption{Student performance data}
\label{tab:long}
\end{table}

\section{Table with Mathematical Expressions}
Here's a table with mathematical expressions:

\begin{table}[h]
\centering
\begin{tabular}{|l|c|c|}
\hline
\textbf{Function} & \textbf{Formula} & \textbf{Value} \\
\hline
Linear & $f(x) = ax + b$ & $f(2) = 2a + b$ \\
Quadratic & $f(x) = ax^2 + bx + c$ & $f(2) = 4a + 2b + c$ \\
Exponential & $f(x) = ae^{bx}$ & $f(2) = ae^{2b}$ \\
\hline
\end{tabular}
\caption{Mathematical functions and their values}
\label{tab:math}
\end{table}

\section{Table References}
As shown in Table~\ref{tab:performance}, Method B achieves the highest accuracy.

The statistical results in Table~\ref{tab:stats} show the distribution of participant characteristics.

Table~\ref{tab:booktabs} demonstrates the professional appearance achievable with the booktabs package.

\section{Best Practices for Data Tables}
\begin{itemize}
    \item Use clear, descriptive column headers
    \item Include units of measurement
    \item Align numbers by decimal point
    \item Use consistent formatting
    \item Include appropriate captions
    \item Reference tables in your text
    \item Consider using booktabs for professional appearance
    \item Use multi-row tables for grouped data
    \item Rotate tables when necessary for space
\end{itemize}

\end{document}
